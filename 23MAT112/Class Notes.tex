\documentclass[twoside]{report}
%%%%%%%%%%%%%%%%%%%%%%%%%%%%%%%%%%%%%%%%%%%%%%%%%%%%%%%%%%%%%%%%%%%%%%%%%%%%%%%
%                                Basic Packages                               %
%%%%%%%%%%%%%%%%%%%%%%%%%%%%%%%%%%%%%%%%%%%%%%%%%%%%%%%%%%%%%%%%%%%%%%%%%%%%%%%

% Gives us multiple colors.
\usepackage[dvipsnames,pdftex]{xcolor}
% Lets us style link colors.
\usepackage{hyperref}
% Lets us import images and graphics.
\usepackage{graphicx}
% Lets us use figures in floating environments.
\usepackage{float}
% Lets us create multiple columns.
\usepackage{multicol}
% Gives us better math syntax.
\usepackage{amsmath,amsfonts,mathtools,amsthm,amssymb}
% Lets us strikethrough text.
\usepackage{cancel}
% Lets us edit the caption of a figure.
\usepackage{caption}
% Lets us import pdf directly in our tex code.
\usepackage{pdfpages}
% Lets us do algorithm stuff.
\usepackage[ruled,vlined,linesnumbered]{algorithm2e}
% Use a smiley face for our qed symbol.
\usepackage{tikzsymbols}
\renewcommand\qedsymbol{$\Laughey$}

\def\class{article}


%%%%%%%%%%%%%%%%%%%%%%%%%%%%%%%%%%%%%%%%%%%%%%%%%%%%%%%%%%%%%%%%%%%%%%%%%%%%%%%
%                                Basic Settings                               %
%%%%%%%%%%%%%%%%%%%%%%%%%%%%%%%%%%%%%%%%%%%%%%%%%%%%%%%%%%%%%%%%%%%%%%%%%%%%%%%

%%%%%%%%%%%%%
%  Symbols  %
%%%%%%%%%%%%%

\let\implies\Rightarrow
\let\impliedby\Leftarrow
\let\iff\Leftrightarrow
\let\epsilon\varepsilon

%%%%%%%%%%%%
%  Tables  %
%%%%%%%%%%%%

\setlength{\tabcolsep}{5pt}
\renewcommand\arraystretch{1.5}

%%%%%%%%%%%%%%
%  SI Unitx  %
%%%%%%%%%%%%%%

\usepackage{siunitx}
\sisetup{locale = FR}

%%%%%%%%%%
%  TikZ  %
%%%%%%%%%%

\usepackage[framemethod=TikZ]{mdframed}
\usepackage{tikz}
\usepackage{tikz-cd}
\usepackage{tikzsymbols}

\usetikzlibrary{intersections, angles, quotes, calc, positioning}
\usetikzlibrary{arrows.meta}

\tikzset{
  force/.style={thick, {Circle[length=2pt]}-stealth, shorten <=-1pt}
}

%%%%%%%%%%%%%%%
%  PGF Plots  %
%%%%%%%%%%%%%%%

\usepackage{pgfplots}
\pgfplotsset{compat=1.13}

%%%%%%%%%%%%%%%%%%%%%%%
%  Center Title Page  %
%%%%%%%%%%%%%%%%%%%%%%%

\usepackage{titling}
\renewcommand\maketitlehooka{\null\mbox{}\vfill}
\renewcommand\maketitlehookd{\vfill\null}

%%%%%%%%%%%%%%%%%%%%%%%%%%%%%%%%%%%%%%%%%%%%%%%%%%%%%%%
%  Create a grey background in the middle of the PDF  %
%%%%%%%%%%%%%%%%%%%%%%%%%%%%%%%%%%%%%%%%%%%%%%%%%%%%%%%

\usepackage{eso-pic}
\newcommand\definegraybackground{
  \definecolor{reallylightgray}{HTML}{FAFAFA}
  \AddToShipoutPicture{
    \ifthenelse{\isodd{\thepage}}{
      \AtPageLowerLeft{
        \put(\LenToUnit{\dimexpr\paperwidth-222pt},0){
          \color{reallylightgray}\rule{222pt}{297mm}
        }
      }
    }
    {
      \AtPageLowerLeft{
        \color{reallylightgray}\rule{222pt}{297mm}
      }
    }
  }
}

%%%%%%%%%%%%%%%%%%%%%%%%
%  Modify Links Color  %
%%%%%%%%%%%%%%%%%%%%%%%%

\hypersetup{
  % Enable highlighting links.
  colorlinks,
  % Change the color of links to blue.
  linkcolor=blue,
  % Change the color of citations to black.
  citecolor={black},
  % Change the color of url's to blue with some black.
  urlcolor={blue!80!black}
}

%%%%%%%%%%%%%%%%%%
% Fix WrapFigure %
%%%%%%%%%%%%%%%%%%

\newcommand{\wrapfill}{\par\ifnum\value{WF@wrappedlines}>0
    \parskip=0pt
    \addtocounter{WF@wrappedlines}{-1}%
    \null\vspace{\arabic{WF@wrappedlines}\baselineskip}%
    \WFclear
\fi}

%%%%%%%%%%%%%%%%%
% Multi Columns %
%%%%%%%%%%%%%%%%%

\let\multicolmulticols\multicols
\let\endmulticolmulticols\endmulticols

\RenewDocumentEnvironment{multicols}{mO{}}
{%
  \ifnum#1=1
    #2%
  \else % More than 1 column
    \multicolmulticols{#1}[#2]
  \fi
}
{%
  \ifnum#1=1
\else % More than 1 column
  \endmulticolmulticols
\fi
}

\newlength{\thickarrayrulewidth}
\setlength{\thickarrayrulewidth}{5\arrayrulewidth}


%%%%%%%%%%%%%%%%%%%%%%%%%%%%%%%%%%%%%%%%%%%%%%%%%%%%%%%%%%%%%%%%%%%%%%%%%%%%%%%
%                           School Specific Commands                          %
%%%%%%%%%%%%%%%%%%%%%%%%%%%%%%%%%%%%%%%%%%%%%%%%%%%%%%%%%%%%%%%%%%%%%%%%%%%%%%%

%%%%%%%%%%%%%%%%%%%%%%%%%%%
%  Initiate New Counters  %
%%%%%%%%%%%%%%%%%%%%%%%%%%%

\newcounter{lecturecounter}

%%%%%%%%%%%%%%%%%%%%%%%%%%
%  Helpful New Commands  %
%%%%%%%%%%%%%%%%%%%%%%%%%%

\makeatletter

\newcommand\resetcounters{
  % Reset the counters for subsection, subsubsection and the definition
  % all the custom environments.
  \setcounter{subsection}{0}
  \setcounter{subsubsection}{0}
  \setcounter{paragraph}{0}
  \setcounter{subparagraph}{0}
  \setcounter{theorem}{0}
  \setcounter{claim}{0}
  \setcounter{corollary}{0}
  \setcounter{lemma}{0}
  \setcounter{exercise}{0}

  \@ifclasswith\class{nocolor}{
    \setcounter{definition}{0}
  }{}
}

%%%%%%%%%%%%%%%%%%%%%
%  Lecture Command  %
%%%%%%%%%%%%%%%%%%%%%

\usepackage{xifthen}

% EXAMPLE:
% 1. \lesson{Oct 17 2022 Mon (08:46:48)}{Lecture Title}
% 2. \lesson[4]{Oct 17 2022 Mon (08:46:48)}{Lecture Title}
% 3. \lesson{Oct 17 2022 Mon (08:46:48)}{}
% 4. \lesson[4]{Oct 17 2022 Mon (08:46:48)}{}
% Parameters:
% 1. (Optional) Lesson number.
% 2. Time and date of lecture.
% 3. Lecture Title.
\def\@lesson{}
\newcommand\lesson[3][\arabic{lecturecounter}]{
  % Add 1 to the lecture counter.
  \addtocounter{lecturecounter}{1}

  % Set the section number to the lecture counter.
  \setcounter{section}{#1}
  \renewcommand\thesubsection{#1.\arabic{subsection}}

  % Reset the counters.
  \resetcounters

  % Check if user passed the lecture title or not.
  \ifthenelse{\isempty{#3}}{
    \def\@lesson{Lecture \arabic{lecturecounter}}
  }{
    \def\@lesson{Lecture \arabic{lecturecounter}: #3}
  }

  % Display the information like the following:
  %                                                  Oct 17 2022 Mon (08:49:10)
  % ---------------------------------------------------------------------------
  % Lecture 1: Lecture Title
  \hfill\small{#2}
  \hrule
  \vspace*{-0.3cm}
  \section*{\@lesson}
  \addcontentsline{toc}{section}{\@lesson}
}

%%%%%%%%%%%%%%%%%%%%
%  Import Figures  %
%%%%%%%%%%%%%%%%%%%%

\usepackage{import}
\pdfminorversion=7

% EXAMPLE:
% 1. \incfig{limit-graph}
% 2. \incfig[0.4]{limit-graph}
% Parameters:
% 1. The figure name. It should be located in figures/NAME.tex_pdf.
% 2. (Optional) The width of the figure. Example: 0.5, 0.35.
\newcommand\incfig[2][1]{%
  \def\svgwidth{#1\columnwidth}
  \import{./figures/}{#2.pdf_tex}
}

\begingroup\expandafter\expandafter\expandafter\endgroup
\expandafter\ifx\csname pdfsuppresswarningpagegroup\endcsname\relax
\else
  \pdfsuppresswarningpagegroup=1\relax
\fi

%%%%%%%%%%%%%%%%%
% Fancy Headers %
%%%%%%%%%%%%%%%%%

\usepackage{fancyhdr}

% Force a new page.

\newcommand\forcenewpage{\clearpage\mbox{~}\clearpage\newpage}
\newcommand\createintro{
  \pagestyle{fancy}
  \fancyhead{}
  \fancyhead[C]{23PHY114}
  \fancyfoot[L]{Adithya Nair}
  \fancyfoot[R]{AID23002}
  % Create a new page.
}
  \newpage
\makeatother

%%%%%%%%%%%%%%%%%%%%%%%%%%%%%%%%%%%%%%%%%%%%%%%%%%%%%%%%%%%%%%%%%%%%%%%%%%%%%%%
%                               Custom Commands                               %
%%%%%%%%%%%%%%%%%%%%%%%%%%%%%%%%%%%%%%%%%%%%%%%%%%%%%%%%%%%%%%%%%%%%%%%%%%%%%%%

%%%%%%%%%%%%
%  Circle  %
%%%%%%%%%%%%

\newcommand*\circled[1]{\tikz[baseline=(char.base)]{
  \node[shape=circle,draw,inner sep=1pt] (char) {#1};}
}

%%%%%%%%%%%%%%%%%%%
%  Todo Commands  %
%%%%%%%%%%%%%%%%%%%

\usepackage{xargs}
\usepackage[colorinlistoftodos]{todonotes}

\makeatletter

\@ifclasswith\class{working}{
  \newcommandx\unsure[2][1=]{\todo[linecolor=red,backgroundcolor=red!25,bordercolor=red,#1]{#2}}
  \newcommandx\change[2][1=]{\todo[linecolor=blue,backgroundcolor=blue!25,bordercolor=blue,#1]{#2}}
  \newcommandx\info[2][1=]{\todo[linecolor=OliveGreen,backgroundcolor=OliveGreen!25,bordercolor=OliveGreen,#1]{#2}}
  \newcommandx\improvement[2][1=]{\todo[linecolor=Plum,backgroundcolor=Plum!25,bordercolor=Plum,#1]{#2}}

  \newcommand\listnotes{
    \newpage
    \listoftodos[Notes]
  }
}{
  \newcommandx\unsure[2][1=]{}
  \newcommandx\change[2][1=]{}
  \newcommandx\info[2][1=]{}
  \newcommandx\improvement[2][1=]{}

  \newcommand\listnotes{}
}

\makeatother

%%%%%%%%%%%%%
%  Correct  %
%%%%%%%%%%%%%

% EXAMPLE:
% 1. \correct{INCORRECT}{CORRECT}
% Parameters:
% 1. The incorrect statement.
% 2. The correct statement.
\definecolor{correct}{HTML}{009900}
\newcommand\correct[2]{{\color{red}{#1 }}\ensuremath{\to}{\color{correct}{ #2}}}


%%%%%%%%%%%%%%%%%%%%%%%%%%%%%%%%%%%%%%%%%%%%%%%%%%%%%%%%%%%%%%%%%%%%%%%%%%%%%%%
%                                 Environments                                %
%%%%%%%%%%%%%%%%%%%%%%%%%%%%%%%%%%%%%%%%%%%%%%%%%%%%%%%%%%%%%%%%%%%%%%%%%%%%%%%

\usepackage{varwidth}
\usepackage{thmtools}
\usepackage[most,many,breakable]{tcolorbox}

\tcbuselibrary{theorems,skins,hooks}
\usetikzlibrary{arrows,calc,shadows.blur}

%%%%%%%%%%%%%%%%%%%
%  Define Colors  %
%%%%%%%%%%%%%%%%%%%

\definecolor{myblue}{RGB}{45, 111, 177}
\definecolor{mygreen}{RGB}{56, 140, 70}
\definecolor{myred}{RGB}{199, 68, 64}
\definecolor{mypurple}{RGB}{197, 92, 212}

\definecolor{definition}{HTML}{228b22}
\definecolor{theorem}{HTML}{00007B}
\definecolor{example}{HTML}{2A7F7F}
\definecolor{definition}{HTML}{228b22}
\definecolor{prop}{HTML}{191971}
\definecolor{lemma}{HTML}{983b0f}
\definecolor{exercise}{HTML}{88D6D1}

\colorlet{definition}{mygreen!85!black}
\colorlet{claim}{mygreen!85!black}
\colorlet{corollary}{mypurple!85!black}
\colorlet{proof}{theorem}

%%%%%%%%%%%%%%%%%%%%%%%%%%%%%%%%%%%%%%%%%%%%%%%%%%%%%%%%%
%  Create Environments Styles Based on Given Parameter  %
%%%%%%%%%%%%%%%%%%%%%%%%%%%%%%%%%%%%%%%%%%%%%%%%%%%%%%%%%

\mdfsetup{skipabove=1em,skipbelow=0em}

%%%%%%%%%%%%%%%%%%%%%%
%  Helpful Commands  %
%%%%%%%%%%%%%%%%%%%%%%

% EXAMPLE:
% 1. \createnewtheoremstyle{thmdefinitionbox}{}{}
% 2. \createnewtheoremstyle{thmtheorembox}{}{}
% 3. \createnewtheoremstyle{thmproofbox}{qed=\qedsymbol}{
%       rightline=false, topline=false, bottomline=false
%    }
% Parameters:
% 1. Theorem name.
% 2. Any extra parameters to pass directly to declaretheoremstyle.
% 3. Any extra parameters to pass directly to mdframed.
\newcommand\createnewtheoremstyle[3]{
  \declaretheoremstyle[
  headfont=\bfseries\sffamily, bodyfont=\normalfont, #2,
  mdframed={
    #3,
  },
  ]{#1}
}

% EXAMPLE:
% 1. \createnewcoloredtheoremstyle{thmdefinitionbox}{definition}{}{}
% 2. \createnewcoloredtheoremstyle{thmexamplebox}{example}{}{
%       rightline=true, leftline=true, topline=true, bottomline=true
%     }
% 3. \createnewcoloredtheoremstyle{thmproofbox}{proof}{qed=\qedsymbol}{backgroundcolor=white}
% Parameters:
% 1. Theorem name.
% 2. Color of theorem.
% 3. Any extra parameters to pass directly to declaretheoremstyle.
% 4. Any extra parameters to pass directly to mdframed.
\newcommand\createnewcoloredtheoremstyle[4]{
  \declaretheoremstyle[
  headfont=\bfseries\sffamily\color{#2}, bodyfont=\normalfont, #3,
  mdframed={
    linewidth=2pt,
    rightline=false, leftline=true, topline=false, bottomline=false,
    linecolor=#2, backgroundcolor=#2!5, #4,
  },
  ]{#1}
}

%%%%%%%%%%%%%%%%%%%%%%%%%%%%%%%%%%%
%  Create the Environment Styles  %
%%%%%%%%%%%%%%%%%%%%%%%%%%%%%%%%%%%

\makeatletter
\@ifclasswith\class{nocolor}{
  % Environments without color.

  \createnewtheoremstyle{thmdefinitionbox}{}{}
  \createnewtheoremstyle{thmtheorembox}{}{}
  \createnewtheoremstyle{thmexamplebox}{}{}
  \createnewtheoremstyle{thmclaimbox}{}{}
  \createnewtheoremstyle{thmcorollarybox}{}{}
  \createnewtheoremstyle{thmpropbox}{}{}
  \createnewtheoremstyle{thmlemmabox}{}{}
  \createnewtheoremstyle{thmexercisebox}{}{}
  \createnewtheoremstyle{thmdefinitionbox}{}{}
  \createnewtheoremstyle{thmquestionbox}{}{}
  \createnewtheoremstyle{thmsolutionbox}{}{}

  \createnewtheoremstyle{thmproofbox}{qed=\qedsymbol}{}
  \createnewtheoremstyle{thmexplanationbox}{}{}
}{
  % Environments with color.

  \createnewcoloredtheoremstyle{thmdefinitionbox}{definition}{}{}
  \createnewcoloredtheoremstyle{thmtheorembox}{theorem}{}{}
  \createnewcoloredtheoremstyle{thmexamplebox}{example}{}{
    rightline=true, leftline=true, topline=true, bottomline=true
  }
  \createnewcoloredtheoremstyle{thmclaimbox}{claim}{}{}
  \createnewcoloredtheoremstyle{thmcorollarybox}{corollary}{}{}
  \createnewcoloredtheoremstyle{thmpropbox}{prop}{}{}
  \createnewcoloredtheoremstyle{thmlemmabox}{lemma}{}{}
  \createnewcoloredtheoremstyle{thmexercisebox}{exercise}{}{}

  \createnewcoloredtheoremstyle{thmproofbox}{proof}{qed=\qedsymbol}{backgroundcolor=white}
  \createnewcoloredtheoremstyle{thmexplanationbox}{example}{qed=\qedsymbol}{backgroundcolor=white}
}
\makeatother

%%%%%%%%%%%%%%%%%%%%%%%%%%%%%
%  Create the Environments  %
%%%%%%%%%%%%%%%%%%%%%%%%%%%%%

\declaretheorem[numberwithin=section, style=thmtheorembox,     name=Theorem]{theorem}
\declaretheorem[numbered=no,          style=thmexamplebox,     name=Example]{example}
\declaretheorem[numberwithin=section, style=thmclaimbox,       name=Claim]{claim}
\declaretheorem[numberwithin=section, style=thmcorollarybox,   name=Corollary]{corollary}
\declaretheorem[numberwithin=section, style=thmpropbox,        name=Proposition]{prop}
\declaretheorem[numberwithin=section, style=thmlemmabox,       name=Lemma]{lemma}
\declaretheorem[numberwithin=section, style=thmexercisebox,    name=Exercise]{exercise}
\declaretheorem[numbered=no,          style=thmproofbox,       name=Proof]{replacementproof}
\declaretheorem[numbered=no,          style=thmexplanationbox, name=Proof]{expl}

\makeatletter
\@ifclasswith\class{nocolor}{
  % Environments without color.

  \newtheorem*{note}{Note}

  \declaretheorem[numberwithin=section, style=thmdefinitionbox, name=Definition]{definition}
  \declaretheorem[numberwithin=section, style=thmquestionbox,   name=Question]{question}
  \declaretheorem[numberwithin=section, style=thmsolutionbox,   name=Solution]{solution}
}{
  % Environments with color.

  \newtcbtheorem[number within=section]{Definition}{Definition}{
    enhanced,
    before skip=2mm,
    after skip=2mm,
    colback=red!5,
    colframe=red!80!black,
    colbacktitle=red!75!black,
    boxrule=0.5mm,
    attach boxed title to top left={
      xshift=1cm,
      yshift*=1mm-\tcboxedtitleheight
    },
    varwidth boxed title*=-3cm,
    boxed title style={
      interior engine=empty,
      frame code={
        \path[fill=tcbcolback]
        ([yshift=-1mm,xshift=-1mm]frame.north west)
        arc[start angle=0,end angle=180,radius=1mm]
        ([yshift=-1mm,xshift=1mm]frame.north east)
        arc[start angle=180,end angle=0,radius=1mm];
        \path[left color=tcbcolback!60!black,right color=tcbcolback!60!black,
        middle color=tcbcolback!80!black]
        ([xshift=-2mm]frame.north west) -- ([xshift=2mm]frame.north east)
        [rounded corners=1mm]-- ([xshift=1mm,yshift=-1mm]frame.north east)
        -- (frame.south east) -- (frame.south west)
        -- ([xshift=-1mm,yshift=-1mm]frame.north west)
        [sharp corners]-- cycle;
      },
    },
    fonttitle=\bfseries,
    title={#2},
    #1
  }{def}

  \NewDocumentEnvironment{definition}{O{}O{}}
    {\begin{Definition}{#1}{#2}}{\end{Definition}}

  \newtcolorbox{note}[1][]{%
    enhanced jigsaw,
    colback=gray!20!white,%
    colframe=gray!80!black,
    size=small,
    boxrule=1pt,
    title=\textbf{Note:-},
    halign title=flush center,
    coltitle=black,
    breakable,
    drop shadow=black!50!white,
    attach boxed title to top left={xshift=1cm,yshift=-\tcboxedtitleheight/2,yshifttext=-\tcboxedtitleheight/2},
    minipage boxed title=1.5cm,
    boxed title style={%
      colback=white,
      size=fbox,
      boxrule=1pt,
      boxsep=2pt,
      underlay={%
        \coordinate (dotA) at ($(interior.west) + (-0.5pt,0)$);
        \coordinate (dotB) at ($(interior.east) + (0.5pt,0)$);
        \begin{scope}
          \clip (interior.north west) rectangle ([xshift=3ex]interior.east);
          \filldraw [white, blur shadow={shadow opacity=60, shadow yshift=-.75ex}, rounded corners=2pt] (interior.north west) rectangle (interior.south east);
        \end{scope}
        \begin{scope}[gray!80!black]
          \fill (dotA) circle (2pt);
          \fill (dotB) circle (2pt);
        \end{scope}
      },
    },
    #1,
  }

  \newtcbtheorem{Question}{Question}{enhanced,
    breakable,
    colback=white,
    colframe=myblue!80!black,
    attach boxed title to top left={yshift*=-\tcboxedtitleheight},
    fonttitle=\bfseries,
    title=\textbf{Question:-},
    boxed title size=title,
    boxed title style={%
      sharp corners,
      rounded corners=northwest,
      colback=tcbcolframe,
      boxrule=0pt,
    },
    underlay boxed title={%
      \path[fill=tcbcolframe] (title.south west)--(title.south east)
      to[out=0, in=180] ([xshift=5mm]title.east)--
      (title.center-|frame.east)
      [rounded corners=\kvtcb@arc] |-
      (frame.north) -| cycle;
    },
    #1
  }{def}

  \NewDocumentEnvironment{question}{O{}O{}}
  {\begin{Question}{#1}{#2}}{\end{Question}}

  \newtcolorbox{Solution}{enhanced,
    breakable,
    colback=white,
    colframe=mygreen!80!black,
    attach boxed title to top left={yshift*=-\tcboxedtitleheight},
    title=\textbf{Solution:-},
    boxed title size=title,
    boxed title style={%
      sharp corners,
      rounded corners=northwest,
      colback=tcbcolframe,
      boxrule=0pt,
    },
    underlay boxed title={%
      \path[fill=tcbcolframe] (title.south west)--(title.south east)
      to[out=0, in=180] ([xshift=5mm]title.east)--
      (title.center-|frame.east)
      [rounded corners=\kvtcb@arc] |-
      (frame.north) -| cycle;
    },
  }

  \NewDocumentEnvironment{solution}{O{}O{}}
  {\vspace{-10pt}\begin{Solution}{#1}{#2}}{\end{Solution}}
}
\makeatother

%%%%%%%%%%%%%%%%%%%%%%%%%%%%
%  Edit Proof Environment  %
%%%%%%%%%%%%%%%%%%%%%%%%%%%%

\renewenvironment{proof}[1][\proofname]{\vspace{-10pt}\begin{replacementproof}}{\end{replacementproof}}
\newenvironment{explanation}[1][\proofname]{\vspace{-10pt}\begin{expl}}{\end{expl}}

\theoremstyle{definition}

\newtheorem*{notation}{Notation}
\newtheorem*{previouslyseen}{As previously seen}
\newtheorem*{problem}{Problem}
\newtheorem*{observe}{Observe}
\newtheorem*{property}{Property}
\newtheorem*{intuition}{Intuition}

%%%%%%%%%%%%%%%%%%%%%%%%%%%%%%%%%%%%%%%%%%%%%%%%%%%%%%%%%%%%%%%
%                 Code Highlighting                           %
%%%%%%%%%%%%%%%%%%%%%%%%%%%%%%%%%%%%%%%%%%%%%%%%%%%%%%%%%%%%%%%
\usepackage{listings}
\lstset{
language=Octave,
backgroundcolor=\color{white},   % choose the background color; you must add \usepackage{color} or \usepackage{xcolor}
basicstyle=\footnotesize\ttfamily,        % the size of the fonts that are used for the code
breakatwhitespace=false,         % sets if automatic breaks should only happen at whitespace
breaklines=true,                 % sets automatic line breaking
captionpos=b,                    % sets the caption-position to bottom
commentstyle=\color{gray},    % comment style
%escapeinside={\%*}{*)},          % if you want to add LaTeX within your code
extendedchars=true,            % lets you use non-ASCII characters; for 8-bits encodings only, does not work with UTF-8
frame=single,                    % adds a frame around the code
% frameround=fttt,
keepspaces=true,                 % keeps spaces in text, useful for keeping indentation of code (possibly needs columns=flexible)
columns=flexible,
classoffset=0,
keywordstyle=\color{RoyalBlue},       % keyword style
deletekeywords={function,endfunction, if,endif},
classoffset=1,
morekeywords={function,endfunction, if,endif},
keywordstyle=\bf\color{Red},       % keyword style
classoffset=2,
morekeywords={persistent},            % if you want to add more keywords to the set
keywordstyle=\bf\color{ForestGreen},       % keyword style
classoffset=0,
literate=
{/}{{{\color{Mahogany}/}}}1
{*}{{{\color{Mahogany}*}}}1
{.*}{{{\color{Mahogany}.*}}}2
{+}{{{\color{Mahogany}+{}}}}1
{=}{{{\bf\color{Mahogany}=}}}1
{-}{{{\color{Mahogany}-}}}1
{[}{{{\bf\color{RedOrange}[}}}1
{]}{{{\bf\color{RedOrange}]}}}1
{ç}{{\c{c}}}1 % Cedilha
{á}{{\'{a}}}1 % Acentos agudos
{é}{{\'{e}}}1
{í}{{\'{i}}}1
{ó}{{\'{o}}}1
{ú}{{\'{u}}}1
{â}{{\^{a}}}1 % Acentos circunflexos
{ê}{{\^{e}}}1
{î}{{\^{i}}}1
{ô}{{\^{o}}}1
{û}{{\^{u}}}1
{à}{{\`{a}}}1 % Acentos graves
{è}{{\`{e}}}1
{ì}{{\`{i}}}1
{ò}{{\`{o}}}1
{ù}{{\`{u}}}1
{ã}{{\~{a}}}1 % Tils
{ẽ}{{\~{e}}}1
{ĩ}{{\~{i}}}1
{õ}{{\~{o}}}1
{ũ}{{\~{u}}}1,
numbers=left,                    % where to put the line-numbers; possible values are (none, left, right)
numbersep=6pt,                   % how far the line-numbers are from the code
numberstyle=\tiny\color{gray}, % the style that is used for the line-numbers
rulecolor=\color{black},         % if not set, the frame-color may be changed on line-breaks within not-black text (e.g. comments (green here))
showspaces=false,                % show spaces everywhere adding particular underscores; it overrides 'showstringspaces'
showstringspaces=false,          % underline spaces within strings only
showtabs=false,                  % show tabs within strings adding particular underscores
stepnumber=1,                    % the step between two line-numbers. If it's 1, each line will be numbered
stringstyle=\color{purple},     % string literal style
tabsize=2,                       % sets default tabsize to 2 spaces
}


\title{\Huge{23MAT112}\\ Class Notes}
\author{\huge{Adithya Nair}}
\date{}

\begin{document}

\maketitle
\newpage% or \cleardoublepage
% \pdfbookmark[<level>]{<title>}{<dest>}
\tableofcontents
\chapter{Eigenvalues And Eigenvectors}
\section{Introduction} % (fold)
Reference - \textbf{Learning from Data, Gilbert Strang as well as Chapter 6 of Introduction To Linear Algebra}
\begin{align*}
	A \vec{x} = \lambda \vec{x}
\end{align*}

This equation is a mathematical way of expressing the idea that there's some vector $\vec{x}$ that does not change in direction, but only changes in size or magnitude by some factor $\lambda$.


\begin{definition}[Eigenvector]
   A vector that does not change in direction after a linear transformation.
\end{definition}
\begin{definition}[Eigenvalue]
The scalar factor by which an eigenvector changes after a linear transformation	
\end{definition}
\section{Example Matrices} % (fold)
\subsection{Projection Matrices}
A reminder, projection matrices are used to bring vectors outside the column space of a given matrix A to the column space of A.
\[
	Ax = b
\]
\section*{3 Cases}
When $\vec{x}$ is in the column space,
\[
	Px = \lambda x, where \\ \lambda = 1
\]
When $\vec{x}$ is out of the column space,
\[
   \text{It is not an eigen vector}
\]
When $\vec{x}$ is orthogonal to the column space,
\[
	Px = \lambda x, where \lambda = 0
\]
% section Example Matrices (end)
\subsection{Permutation Matrix}
\[
	\begin{bmatrix}
	   0 & 1 \\
	   1 & 0 \\
	\end{bmatrix}
	\begin{bmatrix}
	x_1 \\
	x_2 \\
	\end{bmatrix}
	= 
\begin{bmatrix}
   x_2 \\ 
   x_1 \\
\end{bmatrix}
\]
The eigenvectors for this matrix is:
\[
	\begin{bmatrix}
	1 \\
	1 \\
	\end{bmatrix}
	and 
	\begin{bmatrix}
      -1 \\
      1 \\
	\end{bmatrix}
\]
\subsection{Rotation matrix}
\[
	R = 
	\begin{bmatrix}
		cos \theta & - sin \theta \\
		sin \theta & cos \theta \\
	\end{bmatrix}
\]
For $\theta= \frac{\pi}{2}$
\[
	R = 
	\begin{bmatrix}
		 0 & -1 \\
		 1 & 0 \\
	\end{bmatrix}
\]
The calculated eigenvalues for this matrix is $\pm i$
\begin{note}
	A matrix is the representation of a linear transformation in a given basis
	\[
		L(\alpha a + \beta b) = \alpha L(a) + \beta L(b)
	\]
\end{note}
\begin{question}
	B is $3 \times3$ matrix with eigenvalues $\lambda = 0,1,2$.
	 \begin{enumerate}
		\item What is the rank of B?
		\item What is the determinant of B?  
	\end{enumerate}
	\textbf{Ans. 1. - } The rank of B is 2. There is one distinct eigenvalue which is 0. This means that the nullspace is one-dimensional. The nullspace is (n-r) dimensions, therefore the rank of the matrix is 2. \\
	\textbf{Ans. 2. - } The determinant is 0, since there exists an eigenvalue which is 0, which means that 
	\[
	   det(A - \lambda I) = 0, \text{where } \lambda \text{ is 0}
	\]
\end{question}
% section Introduction (end)
\section{Diagonalizable Matrices} % (fold)
\label{sec:Diagonalizable Matrices}

\begin{definition}[Diagonalizable Matrices]
	Let A and B be two square matrices of size $n \times n$. We say that A and B are similar if there is an invertible matrix of the same size P such that: 
	\[
	   A = PBP^{-1}
	\]
	Then we can say that A is \textbf{diagonalizable} if A is similar to a diagonal matrix D
	\label{dfn1}
\end{definition}
\begin{lemma}
   Suppose that A and B are two $n \times n$ matrices and P is an invertible matrix, such that $A = PBP^{-1}.$ Then, $A^{n} = PB^{n}P^{-1}$ 
\end{lemma}
\begin{proof}
	Using the principle of mathematical induction.\\
	We are given, $A = PBP^{-1}$
	to show, $A^{n} = PB^{n}P^{-1}$
	\textbf{Base step - } $n = 1, A^{1} = P^1 B^1 P^{-1}$, which is true 
	\textbf{Induction step - } Supppose $A^{n} = PB^{n}P^{-1}$. We need to show that $A^{n+1} = PB^{n+1}P^{-1}$
	\[
	   A^{n+1} = A \dot A^{n}
	\]
	\[
	   = (PBP^{-1})(PB^n P^{-1})
	\]
	\[
	   = PBB^nP^{-1} = PB^{n+1}P^{-1}
	\]
\end{proof}

\begin{theorem}
   Let A be an $n \times n$ matrix and let $v_1,v_2,\dots,v_k$ be eigenvectors of A with distinct eigenvalues $\lambda_1, \lambda_2,\dots\lambda_k$. Then $v_1,v_2\dots v_k$ are independent. In particular, if k = n, then $v_1,v_2\dots v_k$ are a basis of eigenvectors for $\mathbb{R}^n$
\end{theorem}
\begin{proof}
	Suppose $v_1,v_2\dots v_n$ are dependent such that $\exists r_i$  such that,
	\begin{equation}
	    \sum_{i=1}^{k} r_i v_i = 0
		\label{eq:1}
	\end{equation}
	Assume that k is minimal with this property and $r_k$ is all non-zero,
	\[
		A.0 = r_1 A v_1 + r_2 A v_2 \dots + r_k A v_k
	\]
	\begin{equation}
	0 = r_1\lambda_1 v_1 + r_2 \lambda_2 v_2 \dots r_k \lambda_k v_k = 0	
		\label{eq:2}
	\end{equation}
	\begin{equation}
	   \lambda_k \times \ref{eq:1}: r_1 \lambda_k v_1 + \dots + r_k \lambda_k v_k = 0	
		\label{eq:3}
	\end{equation}
	\ref{eq:3} - \ref{eq:2}
	\[
	   r_1(\lambda_k - \lambda_1)v_1 + \dots + r_{k-1}(\lambda_k - \lambda_{k-1})v_{k-1} = 0
	\]
	$\lambda_k - \lambda_i$ is non-zero(distinct eigenvalues)\\
	This forms a linear combination of vectors $v_1,v_2\dots v_{k-1}$ that equate to zero. \\
	This contradicts the assumption that $v_1,v_2\dots v_{k-1}$ is the minimal set which is dependent. \\
	Therefore the eigenvectors are independent.
     \end{proof}
\begin{theorem}
   Let A be a n $\times$ n matrix, Then A is diagonalizable if and only if we can find a basis $v_{1},\cdots, v_{n}$ of eigen vectors for $R^{n}$. In this case,
   \begin{equation}
      A = PDP^{1}	
   \end{equation}
   where P is the matrix whose eigenvectors $v_{1},\cdots, v_{n}$ and D is the diagonal matrix whose diagonal entries are the corresponding eigenvalues $\lambda_{1}, \cdots, \lambda_{n}$
\end{theorem}
\begin{proof}
   \[
   Av_{i} = (PDP^{-1}v_{i})
   \]
where P is the matrix with column vectors $v_{1},v_{2} \cdots v_{n}$ and D is a diagonal matrix with the entries $\lambda_{1},\cdots, \lambda_{n}$
\[
Av_i = (PDP^{-1})P \hat{e}_{i} \\
\]
\[
   Av_{i} = PD\hat{e}_i
\]
\[
Av_{i} = P\lambda_{i} \hat{e}_{i} = \lambda_{i} P \hat{e}_{i} = \lambda_i v_{i}
\]
This proves that $v_{i}$ is the eigenvector of A, and that $\lambda_{i}$ is the corresponding eigenvalue.
Because $P^{-1}$ exists, $v_{1},\cdots,v_{n}$ are independent, based on the theorem proven earlier, this is the basis for $R^{n}$ \\
\textbf{Part b}
Suppose $v_{1},\cdots, v_{n}$ are an eigenvector basis with corresponding eigenvalues $\lambda_{1},\cdots,\lambda_{n}$
Suppose that P is the matrix with column vectors $v_{1},\cdots,v_{n}$ \[
   \text{Let} D = P^{-1}AP 
\]
\[
   D \hat{e}_{i} = (P^{-1}AP)\hat{e}_i
\]
\[
   = P^{-1} A v_{i} = \lambda_{i} P^{-1}v_{i} = \lambda_{i} \hat{e}_{i}
\]
Thus D is the diagonal matrix with the diagonal entries $\lambda_{i}$
\end{proof}
% section Diagonalizable Matrices (end)
\section{Symmetric Matrices} % (fold)
\label{sec:Symmetric Matrices}
\[
	A = A^T
\]
A symmetric matrix has real eigenvalues and orthogonal eigenvectors.
\begin{example}
	\[
	   A =
	   \begin{bmatrix}
	      3 & 1\\
	      1 & 3\\
	   \end{bmatrix}
	\]
	The eigenvalues for the matrix is 2 and 4
	The eigenvectors are:
	\[
	   \vec{v} = 
	   \begin{bmatrix}
	   	1 \\
		-1
	   \end{bmatrix}
	   and
	   \begin{bmatrix}
	   	1 \\
		1 \\
	   \end{bmatrix}
	\]
	These eigenvectors are orthogonal.
\end{example}
\begin{definition}[Hermitian Matrices]
	A complex matrix in which all the entries in the given matrix are equal to their corresponding conjugate transposes.
\end{definition}
\begin{theorem}
   The eigenvalues for a symmetric matrix are real.
\end{theorem}
\begin{proof}
   The claim:
	\[
	For \\	Ax = \lambda x, A = A^T \\
	\]
	\[
		\lambda \in R
	\]
	Take:
	\[
		A x = \lambda x
	\]
	And the complex conjugate:
	\[
	   \overline{A}\overline{x} = \overline{\lambda} \overline{x} \indent  (\overline{A} = A)(real \\ matrix)
	\]
\begin{equation}
	   (\overline{A}\overline{x})^T = (\overline{\lambda} \overline{x})^T \iff \overline{x}^T A^T = \overline{\lambda} \overline{x}^T
	\label{eq:Conjugate}
\end{equation}	
\ref{eq:Conjugate} $\times$ x
\[
   \overline{x}^T A^T x = \lambda \overline{x^T} x \iff \overline{x}^T A x = \overline{\lambda} \overline{x}^T x \iff \overline{x}^T \lambda x = \overline{\lambda} \overline{x}^T x \iff \lambda = \overline{\lambda}
\]
Then $\lambda$ is real, when $\overline{x}^T x \neq 0$
\end{proof}
\begin{lemma}
   Let A be a symmetric matrix. If v and w are eigenvectors with distinct eigenvalues $\lambda \& \mu$ then v \& w are orthogonal.
\end{lemma}

\begin{proof}
	\[
		Av .  w = (Av)^T w = v^T A^Tw = v^T Aw = v . Aw
	\]
	\[
	\iff \lambda v . w = \mu v . w \iff (\lambda - \mu) v.w = 0
	\]
	\[
		\iff v . w  = 0
	\]
\end{proof}
What this means is that, when A is symmetric
\[
   A = PDP^{-1} = PDP^T \indent (P^{-1} = P^T \text{for orthogonal matrices})
\]
% section Symmetric Matrices (end)
\begin{theorem}
   Let A be a symmetric matrix. Then we can find a diagonal matrix D and an orthogonal matrix P such that,
\[
   A = QDQ^T
\]	
In particular, every symmetric matrix is diagonalisable.
\end{theorem}
\section{Positive Definite Matrices}
\begin{definition}[Positive Definite Matrices]
Any matrix A for which the following is true:	
	\begin{displaymath}
   \vec{x}^T A\vec{x} > 0, \forall \vec{x} \neq \vec{0}
	\end{displaymath}
is called a \textbf{positive definite matrix}
\end{definition}

For such a matrix: 
\begin{itemize}
	\item The matrix is always a full-rank matrix. 
	\item The eigenvalues are all positive
	\item $A+B$ is also a positive definite matrix, when A and B are positive definite. 
	\item For a rectangular $m \times n$ matrix, we check for $A^TA \text{ or } AA^T$, \[
	      x^T(A^TA)x = (x^TA^T)(Ax) - (Ax)^T(Ax) = \|{Ax}\|^2
	\]
	This means that $A^TA$ or $AA^T$ is always a positive definite matrix, unless x lies in the nullspace of A, in which case $Ax = 0$
\end{itemize}
\section{Similar Matrices}
Refer to the definition in section \ref{sec:Diagonalizable Matrices}.
A is any square matrix. 
When A is real \& symmetric
\[
	A = PDP^T = QDQ^T
\]
A and D are similar matrices
\begin{example}
	\[
		A = 
		\begin{bmatrix}
		   2 & 1 \\
		   1 & 2 \\
		\end{bmatrix},
		 D = 
		\begin{bmatrix}
		   1 & 0 \\
		   0 & 3 \\
		\end{bmatrix}
	\]
\end{example}
\begin{note}
	Similar matrices have the same eigenvalues
\end{note}
\begin{proof}[Proof Other Information]
   \[
   A\vec{x} - \lambda{x}	
   \]
   \[
      AI\vec{x} = APP^{-1}x = \lambda \vec{x}
   \]
   \[
      \implies P^{-1} A P P^{-1} \vec{x} = \lambda P^{-1} x
   \]
   \[
   	B(P^{-1}x) = \lambda(P^{-1}\vec{x})
   \]
   
\end{proof}
\section{Singular Value Decomposition}
\[
   a = u \sigma v^t
     \]
     where, u,v are orthogonal matrices, $\sum$ is a diagonal matrix.
we know that $r(a) \subseteq r^n$ and the $c(a) \subseteq r^m$, all matrices are linear transformations.
take two orthogonal vectors $u_1, u_2$ such that:
\[
   av_1 = u_1, av_2 = u_2 \text{ where r is the rank of the matrix.}
\]
\[
   u = 
   \begin{bmatrix}
      u_1 & u_2 & u_3 &  \dots & u_r & \dots u_m
   \end{bmatrix}
\]
\[
	v = 
	\begin{bmatrix}
	   v_1 & v_2 & v_3 \dots v_r & \dots & v_n
	\end{bmatrix}
\]
\[
   av = u \sigma \text{ (where $\sigma$ are the factors of $v_i$)}
\]
\[
   a = u \sigma v^{-1} = u\sigma v^t \text{ (since v is orthogonal) }
\]
\[
   a^{t}a = (u\sigma v^t)^t(u \sigma v^t)
\]
\[
	= v \sigma^2 v^t
\]
\section{Procedure For The Factorization}
\begin{enumerate}
	\item v is the eigenvectors of $a^ta$
	\item the eigenvalues are the same for u and v.
	use $av = u \sigma$ 
\end{enumerate}
\chapter{Fourier Series} % (fold)
\section{A Crash Course On Infinite Series} % (fold)
There was this paradox, formulated by Zeno. He proclaimed that, "Suppose that someone is running, towards  a particular place. He cannot reach there, unless he crosses half the distance. If I have to reach the desination, I have to cross the halfway mark of the halfway mark, and so on and so forth....". The proposition is that he will never be able to reach the destination, given this logic.

Let us assume uniform speed, of the runner. Suppose the distance to be covered is 1 unit. The midway mark is $\frac{1}{2}$ units, then $\frac{1}{4}$, then $\frac{1}{8}$ and so on so forth. If he has a uniform speed, it will take T units for the first half to be covered. Then $\frac{T}{2}$ then $\frac{T}{4}$ and so on and so forth. 

Let's add these times. 
\[
   T + \frac{T}{2} + \frac{T}{4} + \dots + \frac{T}{2^{n-1}} \approx 2T
\]

One of the ways to deal with this is a concept known as \textbf{partial sum.}. What should be done is to add the first n terms.

Taking this case,

\[
s_1 = T, s_2 = \frac{3T}{2}, s_3 = \frac{7T}{4}, s_n = \frac{(2^{n} -1)T}{2^{n-1}} = T(2- \frac{1}{2^{n-1}})
\]
When $n \rightarrow \infty$, $\frac{1}{2^{n-1}} \rightarrow 0$ then the total time taken is 2T.

Let's stop assuming that the runner is moving at uniform speed.

Let's say that the first half is covered in T time, then the next half is covered in $\frac{T}{2}$ time and $\frac{T}{3}$ time.

The total time is:

\[T + \frac{T}{2} + \frac{T}{3} + \dots + \frac{T}{n} \]
Taking the partial sum,

\[s_1 = T, s_2=\frac{3T}{2}, s_3 = \frac{11T}{6}\]

It is difficult to arrive at the general term of $s_n$ with the traditional methods.

Let's try a graphical analysis,

Take the graph of y = $\frac{1}{x}$
GRAPH REQUIRED HERE.
\newline
in the limit, $\sum_{k=1}^{n}A_k > \int \frac{1}{x}$
\[
\implies 1 + \frac{1}{2} + \frac{1}{3} + \dots + \frac{1}{n} > \int_{1}^{n+1} \frac{1}{x} dx = ln(n+1)\\ 
\]
\[
\implies s_n = T + \frac{T}{2} + \frac{T}{3} + \dots \frac{T}{n} > T ln(n+1)
\]
But this means, that Zeno was correct, the runner will never reach his destination.
The logarithmic curve is not finite. 

The case in which the runner was running at uniform speed is called a convergent series, which means it arrives at a finite sum.

The case in which the runner was running at a non-uniform speed, the series we obtained is called a divergent series, which means it tends towards a non-finite sum.

Earlier, this was solved using sums and approximations, until bright mathematicians such as Euler came along.
\section{Fourier Series}
The main reference is going to be - Olver, Introduction To Partial Differential Equations. Olver was a mathematician, just like his father before him. This textbook is structured in a way where the underlying frameworks behind the mathematics is covered well. This section of Fourier Series is covered in Chapter 3 and Chapter 7 of the textbook.

Calling back to the previous section 

\[A \vec{x} = \lambda \vec{x}\]
\[L \vec{f}] = \lambda \vec{f}\]
This is an analogue to the way differential equations work

When certain differential problems can be solved using the eigenvalue methods, these problems are known as \textbf{Sturm-Liouville}

It is important that we learn to solve these equations by hand rather than solve them computationally.

Let's begin,

\[\vec{v}.\vec{w} = 0, \text{that implies} v \perp \vec{w}\]

Recall, when sets have two operations with certain properties they are known as vector spaces or linear spaces.

There is a subcategory of them known as inner product spaces, e.g. the dot product. These spaces have a third operation known as the inner product.

These exist for vectors. We are going to take the vector space for all functions.

\[L^P: \int f g dx = <f,g>, \text{Where P = 2}\]

Here, $L^2$ is the defined norm we are going to take, and $\langle f,g\rangle$ is the notation for the inner product.

When we take the inner product for a function on itself,

\[\int f f dx = \langle f,f\rangle dx = \|f\|^2\]

\begin{lemma}
    Under the rescaled $L^2$ inner product, the trigonometric functions: 1, $\cos{x}$, $\sin{x}$, $\cos{2x}$, $\sin{2x}$, \dots satisfy the following orthogonality conditions:
    \begin{itemize}
       \item $\langle cos{kx},\cos{lx}\rangle = \langle \sin{kx}, \sin{lx}\rangle = 0 \forall k \neq l$
        \item $\langle \cos{kx} , \sin{lx}\rangle = 0 \ \forall \ k,l$
        \item $\|1\| = \sqrt{2}, \|\cos{kx}\| = \|\sin{kx}\| = 1 \forall k \neq 0$
    \end{itemize}
\end{lemma}
\begin{proof}
Take the equation,
    \[\frac{1}{\pi} \int_{-\pi}^{\pi} fg dx = \langle f,g \rangle\]
For the first orthogonality condition,
\[\langle\cos{kx},\cos{lx}\rangle = \frac{1}{\pi} \int_{-\pi}^{\pi} \cos{kx}\cos{lx} dx \]

\[\int \cos(Ax)\cos(Bx) dx\]
\[= \int \cos{(A+B)x} + \cos{(A-B)x}\]
\[ = \frac{\sin{(A+B)x}}{A+B} + \frac{\sin{(A-B)x}}{A-B}+C\]

Taking this to the first equation.
\[\langle\cos{kx}, \cos{lx}\rangle = \frac{1}{\pi} [\frac{\sin{(k+l)x}}{k+l} + \frac{\sin{(k-l)x}}{k-l}]^{\pi}_{-\pi}\]
Which is equal to 0

Proving,
\begin{equation}
   \langle \cos{kx} \sin{lx} \rangle = 0 \ \forall \ k, l
\end{equation}
\[
   \frac{1}{\pi} \int_{-\pi}^{\pi} \cos{kx}\cos{lx} dx
\]
\[
   \cos{kx}\sin{lx} =\frac{1}{2}(\sin{(k+l)x} - \sin{(k-l)x})
\]
\[
   = \frac{1}{2\pi} [-\frac{\cos{k+l}}{k+l} + \frac{\cos{k-l}}{k-l}]_{-\pi}^{\pi}
\]
Which equals to 0.

Proving, 
\[
	\|1\|^2 = \sqrt{2}, \|\cos{kx}\| = \|\sin{kx}\| = 1 \forall k \neq 0
\]
\[
   \|1\| = \frac{1}{\pi}\int_{\pi}^{\pi} 1 dx 
\]
\[
   = \frac{1}{\pi}[x]^{\pi}_{-\pi}	
\]
\[
   \frac{1}{\pi} (\pi + \pi)
\]
\[
	\|1\|^2 = 2
\]
\[
   \|1\| = \sqrt{2}
\]
For $\|cos x\|$
\[
   \frac{1}{\pi}\int_\pi^{-\pi} (cos^2 x)
\]
\[
   \frac{1}{\pi} \int_{\pi}^{-\pi} \frac{1+\cos{2x}}{2}
\]
\end{proof}
\begin{definition}[Fourier Series]
   The Fourier series of a function $f(x)$ is defined on $-\pi \leq x \leq \pi$ is f(x) where    \[
   	 f(x) \approxeq \frac{a_0}{2}+\sum_{k=1}^{\infty}[a_k \cos{kx} + b_k \sin{kx}]
   \]
   where  coefficients are given by the inner product formulae:
   \begin{align*}
      a_k = \langle f, \cos{kx} \rangle = \frac{1}{\pi}\int_{-\pi}^{\pi} f(x) \cos{(kx)} dx \ \forall k = 0,1,2\dots \\
      b_k = \langle f, \sin{kx} \rangle = \frac{1}{\pi}\int_{-\pi}^{\pi} f(x) \sin{(kx)} dx \ \forall k = 0,1,2\dots
   \end{align*}
\end{definition}
The Fourier series is an expression of a function using an infinite series of sin and cos functions, the previous lemma proved that the functions are orthogonal to each other. Thus we can use these functions as a basis to express all functions.
\begin{question}
   Find the Fourier Series for $f(x) = \{m, x > 0 \text{ and} -m, x<0\}$
\end{question}
\begin{solution}
   \[
   a_0 = \langle f,1 \rangle	
   \]
   \[
         \langle f,1 \rangle = \frac{1}{\pi} \int^{\pi}_{-\pi}f(x) 
   \]
   \[
   = \int_{-\pi}^{\pi} f(x)dx = \int_{-\pi}^{0}f(x) + \int_0^{\pi} f(x)dx 
   \]
   \[
   = [-mx]^0_\pi + [mx]^\pi_0
   \]
   \[
   = [-\pi k] + \pi k 
   \]
   \[
   	   = 0 
   \]
   \begin{align*}
      a_k = \langle f, \cos{x} \rangle
   \end{align*}
   \[
      \frac{1}{\pi} \int_{-\pi}^{\pi} f(x) \cos{(kx)}
   \]
\end{solution}
\[
	a_k = 0
\]
\[
   b_k = \frac{2}{\pi}[1 - \cos{(k\pi)}]
\]
\section{Change Of Scale} % (fold)
Take a function f(x) such that the period p is some $2L$, where L is some natural number. 
This would not naturally fit within the interval of $2\pi$. 
What we do is, set $x = \frac{p}{2\pi}v$ then $f(\frac{p}{2\pi}v + p) = f(\frac{p}{2\pi}(v+2\pi))$
\[
	x = \frac{p}{2\pi}v
\]
\[
	f(\frac{p}{2\pi}v + p) = f(\frac{p}{2\pi}(v+2\pi))
\]
\[
x = \frac{2L}{2\pi}
\]
\[
   x = \frac{L}{\pi}v
\]
To simplify,
\[
	g(v+2\pi)
\]
\[
   g(v) = f(\frac{p}{2\pi}v)
\]
\[
   g(v) = \frac{a_0}{2} + \sum_{k=1}^{\infty}[a_k \cos{kv}+b_k \sin{kv}]
\]
g(v) is periodic with p = $2\pi$

\[
   f(x) = \frac{a_0}{2} + \sum_{k=1}^{\infty}[a_k cos(\frac{k\pi x}{L})+b_k sin(\frac{k\pi x}{L})]
\]
Finding $a_0$

\[
   a_0 = \frac{1}{\pi} \int_{-\pi}^{\pi} g(v) dv
\]
\[
   = \frac{1}{\pi} \int_{-L}^{L} g(\frac{\pi}{L}x)(\frac{\pi }{L}x) dx
\]
\[
   \frac{1}{L} \int_{-L}^{L} f(x) dx = a_0
\]
Finding $a_k$
\[
   a_k = \frac{1}{\pi} \int_{-\pi}^\pi g(v) \cos{kv} dv
\]
\[
   \frac{1}{L} \int_{-L}^{L} g(\frac{\pi x}{L}) \cos{\frac{k\pi x}{L}}(\frac{\pi}{L})dx
\]
\[
   = \frac{1}{L} \int_{-L }^{L} f(x) \cos {\frac{k \pi x}{L}} dx = a_k
\]
\label{sec:Chnge Of Scale}

% section Change Of Scale (end)
\section{Odd And Even Functions} % (fold)
\label{sec:Odd And Even Functions}

% section Odd And Even Functions (end)
\begin{definition}[Even Function]
   A function such that $f(x) = f(-x)$
   \label{dfn2}
\end{definition} % (fold)
\begin{definition}[Odd Function]
   A function such that $f(-x) = -f(x)$
   \label{dfn3}
\end{definition}

\begin{theorem}
    The Fourier coefficients of the functions $f_1 + f_2$ are the sum of the corresponding coefficients of $f_1$ and $f_2$, The Fourier coefficients of cf is c times the Fourier coefficients of f.

\end{theorem}
\[f(x) = x + \pi , -\pi < x < \pi\] 


\begin{question}
    Find the Fourier series of $f(x) = |x|, -\pi < x < \pi$
\end{question}
\begin{solution}
    When $x < 0, |x| = -x$ and when $x>0, |x| = x$ 
    \[
    a_0 = \int_{-\pi}
    \]
\end{solution}

\section{Integrating Fourier Series}
There is an underlying assumption that all the functions we are dealing with are \textbf{convergent}.

When integrating a function's fourier series. The problem arises that $\frac{a_0}{2}$ is not being integrated. We are trading summation for integration and $\frac{a_0}{2}$ is not contained within the summation.


\section{Differentiation And Integration Of Fourier Series}
\subsection{Background Work}
\subsubsection{Half Range Expansions}
Sometimes, we're given a function f(x) on some interval [0,L].  We can only find the Fourier Series of periodic functions. To convert non-periodic functions, we limit them to the range $[-\pi,\pi]$ and then extend it periodically.

When we're given a range for a given function, what we do is assume both "even" and "odd" extensions(refer to \ref{dfn2} and \ref{dfn3}.

For e.g.
$$
f(x) = \sin{(x)}, 0\leq x\leq\pi,$$
Find the Fourier Series for:
a) Even periodic extension 
b) Odd periodic extension

Ans.
For even periodic extension...
\begin{displaymath}
    a_0 = 2\int_{0}^{\pi} \sin{(x)}dx    
\end{displaymath}
\begin{displaymath}
    a_0 = 2 [-\cos{(x)}]_0^{\pi} \\
    \end{displaymath}
    \begin{displaymath}
        a_0 = 2 [2]
    \end{displaymath}
    \begin{displaymath}
        a_0 = 4 \\
    \end{displaymath}
    \begin{displaymath}
        a_k = \frac{2}{\pi}\int_0^{\pi} \sin{x} \cos{kx} dx
    \end{displaymath}
    \begin{displaymath}
        a_k = \frac{2}{\pi} \int_0^{\pi} \frac{\sin{(kx + x)} + \sin{(kx-x)}}{2}
    \end{displaymath}
    \begin{displaymath}
        a_k = \frac{2}{\pi} [-\frac{1}{2(k+1)}\cos{(kx+x)} - \frac{1}{2(k-1)}\cos{(kx-x)}]_0^\pi
    \end{displaymath}
    \begin{displaymath}
        b_k = 0
    \end{displaymath}

\begin{displaymath}
   = \frac{4}{(1-k^2)\pi}, \text{for even k}
\end{displaymath}
\begin{displaymath}
   0, \text{for odd k}
\end{displaymath}

\begin{question}
   \[f(x) = \frac{2k}{L} x, 0 < x < \frac{L}{2}, \frac{2k}{L}(L-x), \frac{L}{2}<x<L\]
    Find the fourier series for even and odd periodic extensions.
\end{question}
\begin{solution}
   For even periodic extension
   \[
      a_k = \frac{2}{\pi} \int_0^{\frac{L}{2}} \frac{2k}{L} x + 2 \int_{\frac{L}{2}}^{L} \frac{2k}{L}(L-x)dx
   \]
   REVIEW
\end{solution}
\pagebreak

\section{Piecewise Continuous Functions}
The hallmark of the Fourier series is that you can have expansions for even \textbf{discontinuous} functions. 

\begin{definition}[Piecewise Continuous Functions]
   A function f(x) is said to be piecewise continuous on an interval [a,b] if it is defined and continuous ecxept possibly at a finite number of points $a\leq x_1 \leq x_2 \leq \dots \leq x_n \leq b$ Furthermore, at each point of discontinuity, we require that the left and right hand limits exists. 
   \[
      f(x_k^-) = \lim_{x \rightarrow x_k^-} f(x); f(x_k^+) = \lim_{x\rightarrow x_k^+} f(x)
   \]
   At the ends of the domain, the left hand is ignored at a and the right hand limit is ignored at b.
\end{definition}
We take the basis,
\[
   \sigma(x) = 1, x>0 \ and \ 0, x<0
\]
Then
\[
	h(x) = \beta \sigma(x - \xi) = \beta, x> \xi \ and \ 0, x<\xi
\]
\begin{definition}[Piecewise $C^1$]
   A function is called piecewise $C^1(U \subset R)$ (continuous and continuously differentiable) on the interval [a,b] except at a finite number of points $a\leq x_1 \leq x_2 \leq \dots \leq x_n \leq b$. At each exceptional point, the left and the right hand limits of both the function and its derivative exists.
\end{definition}
\section{Complex Fourier Series}
Refer to 3Blue1Brown's video on it
\[
   f(x) \approxeq \frac{a_0}{2}+\sum_{k=1}^{\infty}[a_k \cos{kx} + b_k \sin{kx}]
\]

The complex form of this is
\[
   f(x) = \sum_{k=-\infty}^{\infty} c_k e^{-ikx}
\]
Here the vector space is complex functions.
% section Complex Fourier Series (end)
The inner product is 
\[
   \langle f,g \rangle = \frac{1}{2\pi}\int_{-\pi}^{\pi} f(x) \overline{g(x)} dx
\]
\[
   C_k = \langle f, e^{i\pi x} \rangle = \frac{1}{2\pi} \int_{-\pi}^{\pi} f(x) e^{-i\pi x}
\]
\begin{note}
   Prove fundamental theorems pertaining to the project. Some projects can be on discrete mathematics. Numerical methods for solving Singular Vector Decomposition and implementing eigenvalues and eigenvectors.  
\end{note}
\[
e^{ikx} = \cos{(kx)} + i \sin{(kx)}
\]
\[
   e^{-ikx} = \cos{(kx)} - i \sin{(kx)}
\]
\[
\cos{(kx)} = \frac{e^{ikx} + e^{-ikx}}{2} ; \sin{(kx)} = \frac{e^{ikx} - e^{-ikx}}{2i}
\]
\[
   f(x) = \sum_{k=-\infty}^{\infty} c_k e^{ikx}
\]
\[
   c_k = \langle f, e^{ikx} \rangle = \frac{1}{2\pi} \int_{-\pi}^{\pi} f(x) e^{-ikx}dx
\]
\[
   \langle e^{ikx},e^{ilx} = \frac{1}{2\pi} \int_{-\pi}^\pi e^{ikx} e^{-lx}dx
\]
\[
   \frac{1}{2\pi} \int_{-\pi}^\pi e^{i(k-l)x}
\]
\[
   \frac{1}{2\pi i}[\frac{e^{i(k-l)x}}{(k-l)}]
\]
When k = l, the inner product yields 1, when k $\neq$ l then the inner product yields 0
\section{Differentiation And Integration} % (fold)
\[
   f(x) = \frac{a_0}{2} + \sum_{k=1}^{\infty} [a_k cos(kx) + b_k sin(kx)]
\]
The discussion was surrounding whether or not we can differentiate or integrate functions that have been expressed with their Fourier series... Only when these functions converge. 
Looking at integration first, the integral of sin x is -cos x, and the cos term gets converted to sin. 

The problem arises with the constant term $\frac{a_0}{2}$, whiich is not periodic.
\[
   \frac{a_0}{2} = \frac{1}{2\pi} \int_{-\pi}^{\pi} f(x) dx
\]
This term is known as the \"mean\" of the function. A Fourier Series is only integrable when the mean is zero.
\begin{lemma}
   If f(x)  is $2\pi$- periodic, then its integral $g(x) = \int_0^{x} f(y) dy$ is $2\pi$-periodic if and only if $\int_{-\pi}^\pi f(x) dx = 0$, so that f(x) has zero mean on the interval $[-\pi,\pi]$.
\end{lemma}
\begin{theorem}
   If f is piecewise continuous and has mean zero on the interval $[-\pi,\pi]$ then its Fourier Series  
	\begin{displaymath}
	   f(x) \approxeq \sum_{k=1}^{\infty} [a_k cos(kx) + b_k sin(kx)]
	\end{displaymath}
	can be integrated term by term to produce the Fourier Series. 
	\[
	   g(x) = \int_0^{x} f(y) dy \approxeq m + \sum_{k=1}^{\infty} - \frac{b_k}{k} cos kx + \frac{a_k}{k} sin kx
	\]
	where $m = \int_{-\pi}^{\pi}$g(x) dx
\end{theorem}
\begin{note}
   m is the inner product of g(x) with itself. The remarkable thing about this is it is equal to $\sum_{k=1}^{\infty} \frac{b_k}{k}$
\end{note}
\chapter{Taylor Series}
This concept marks the start of multi-variable calculus. 

Refer to 3Blue1Brown's videos on multi-varable calculus taken on Khan Academy, as well as MIT 18.02 SC. 

Earlier, we were trying to expand a function into orthogonal functions. 

Converting a function into a series, allows us to approximate them. The Taylor Series is important because it gives us the means to find the derivative of polynomials in higher dimensions.
\section{Taylor Polynomial} % (fold)
\label{sec:taylor_polynomial}

% section Taylor Polynomial (end)
\begin{definition}[Polynomials]
	A polynomial is an expression of x such that:
	\begin{displaymath}
	   p(x) = a_0 + a_1x + \dots + a_n x^n , n \geq 0 
	\end{displaymath}
\end{definition}

The goal of the Taylor Series is to approximate a function into a polynomial, known as the Taylor Polynomial.

At the functions at 0 should equal to the polynomial at 0 $x = 0, f(0) = p(0) = a_0$, so should the derivatives, $f'(0) = p'(0) = a_1$ and $f''(0) = p''(0) = 2a_2$ and so on so forth. 

Thus for,
\[
   f^{(n)}x = n!a_n
\]
Where \[
   f^{(1)} = f' = \frac{df}{dx} \ , \ \ f^{(2)} = f'' = \frac{d^2f}{dx^2}
\]
The number of terms is determined by the amount of derivatives that can be done on f(x)
\begin{theorem}
	Let f be a function with derivatives of order n, at the point x = 0. Then there exists one and only one polynomial P of degree $\leq$ n, which satisfies the (n+1) conditions. 
	\[
	   P(0) = f(0), P^{(1)}(0) = f^{(1)}(0), \dots P^{(n)}(0) = f^{(n)}(0)
	\]
	
	\[
	   P(x) = \sum_{k=0}^{n} \frac{f^{(n)}(0)}{n!} x^k = f(0) + f^{(1)}(0) + x + \frac{f^{(2)}(0)}{2}x^2 + \dots \frac{f^{(n)}(0)}{n!}x^n
	\]
\end{theorem}
For a Taylor polynomial at x = a, we take $(x-a)$ instead to get the coefficients again, since $x - a$ is equal to zero.
\[
   P(x) = a_0 + a_1(x-a) + a_2(x-a)^2 + \dots + a_n (x-a)^n 
\]
\[
   P(x) = \sum_{k=0}^n \frac{f^{(k)}(a)}{k!}(x-a)^k
\]
\subsection{Notation}
\[
P = T_n f(x)
\]
Where, P is the Taylor polynomial and $T_n$ is the Taylor operator and $x
$ is the point at which the polynomial is evaluated.
\begin{example}
   $T_n[e^x] \text{ at } x = 0$
   \[
      P = 1 + x + \frac{x^2}{2!} + \frac{x^3}{3!} + \dots + \frac{x^n}{n!}
   \]
\end{example}
\begin{example}
   $T_n[\sin{x}] \text{ at } x = 0$
   \[
      x - \frac{x^3}{3!} + \frac{x^5}{5!} + \dots + (-1)^{n}\frac{x^{2n+1}}{(2n+1)!}
   \]
\end{example}
\begin{example}
   $T_n[\cos{x}] \text{ at } x = 0$
   \[
      1 - \frac{x^2}{2!} + \frac{x^4}{4!} + \dots + (-1)^{n}\frac{x^{2n}}{2n!}
   \]
\end{example}
\section{Calculus Of Taylor Polynomials}
\begin{theorem}
	The Taylor operator $T_n$ has the following properties:
	\begin{enumerate}
		\item If $c_1$ and $c_2$ are constants, then $T_n(c_1f + c_2g) = c_1 T_n(f) + c_2 T_n(g)$
		\item Differentiation: The derivative of a Taylor polynomial of $f$ is a Taylor polynomial of $f'$, in fact, we have
		   $(T_n f)' = T_{n-1} (f')$
		\item Integration: An indefinite integral of a Taylor polynomial of $f$ is a Taylor polynomial of an indefinite integral of $f$. More precisely, if $g(x): \int_a^x f(t) dt$ then we have $T_{n+1}[g(x)] = \int T_n f(x)dx$
	\end{enumerate}
\end{theorem}
\begin{theorem}[The substitution property]
   Let $g(x) = f(cx)$, where c is a constant. Then we have,
   \[
      T_n [g(x;a)] = T_n[f(cx;a)]
   \]
   In particular, when a = 0,
   \[
      T_n[g(x)] = T_n[f(cx)]
   \]
\end{theorem}
\begin{proof}
	\begin{displaymath}
	   g(x) = f(cx) ; g^{k}(c^k) f^{k} (cx) \\ 
	\end{displaymath}
	\[
	   \sum_{k=0}^n \frac{f^{k}(ca)}{k!}(cx -ca)^k
	\]
	\[
	   = T_n [f(cx;ca)]
	\]
\end{proof}
\begin{theorem}
	Let $P_n$ be a polynomial of degree $n \geq 1$. Let $f$ and $g$ be two functions with derivatives of order n at 0 and assume that	
	\[
	   f(x) = P_n(x) + x^n g(x),
	\]
	where $g(x) \rightarrow 0$ as $x \rightarrow 0$. Then $P_n$ is the Taylor Polynomial generated by $f$ at $0$
\end{theorem}
\begin{example}
   \[
   	\frac{1}{1-x} = 1 + x + x^2 + \dots + x^n + \frac{x^{n+1}}{1-x}
   \]
   \[
      T_n[\frac{1}{1-x}] = 1 + x + x^2 + \dots + x^n
   \]
   The integral of this function,
   \[
      T_{n+1} [-ln(1-x)] = x + \frac{x^2}{2} + \frac{x^3}{3} + \dots + \frac{x^{n+1}}{N+1}
   \]
\end{example}
\section{Error} % (fold)
Since the Taylor polynomial is nothing but an approximation, we must understand just how much the approximation loses in information. That is denoted by $E(x)$ or the error.
\[
f(x) - \sum_{k=0}^{n} \frac{f^{(k)}a}{k!}(x-a)^{k} = E(x)
\]
\begin{theorem}
	Assume $f(x)$ has a continuous second derivative $f''(x)$($f \in C^2$) in some neighbourhood of a constant $a$. Then, for every x in this neighbourhood, we have $f(x) = f(a) + f'(a)(x-a) + E_1(x)$, 
	where, 
	\[
		E_1 = \int_a^x (x-t) f''(t) dt
	\]	
\end{theorem}
\begin{proof}
	\[
		f(x) - f(a) - f'(a)(x-a) = E_1(x)
	\]	
	\[
	   \int_{a}^{x} f'(t)dt - f'a(x-a)
	\]
	\[
	   \int_{a}^{x} f'(t)dt - f'a \int_a^x dt \text{[Conversion into definite integral form]}
	\]
	\[
	   \int_{a}^{x}[f'(t) - f'(a)] dt = E_1(x) \text{[Integration is a linear operator]}
	\]
	\[
	   \int_{a}^{x} udv = [f'(t) - f'(a)](t-x) - \int_{a}^{x} (t-x)f''(t)dt
	\]
	\[
	   = \int_{a}^{x} (x-t)f''(t) dt \text{[Applying $\int uv$ rule  ]}
	\]
\end{proof}
\pagebreak
\begin{theorem}
   Assume $f(x)$ has a continuous $(n+1)^th$ derivative $f^{(n+1)}(x)$($f \in C^{n+1}$) in some neighbourhood of a constant $a$. Then, for every x in this neighbourhood, we have $f(x) = \sum_{k=0}^{n}\frac{f^{(k)}}{k!}(x-a)^k+ E_n(x)$, 
	where, 
	\[
	   E_n = \frac{1}{n!}\int_a^x (x-t)^n f^{(n+1)}(t) dt
	\]	
\end{theorem}
\begin{proof}
	Using  mathematical induction.		\\
	i) Base case, n = 1 (Proven in previous theorem) \\
	ii) Inductive step, Assume $E_n$ is true, we have to show $E_{n+1}$ is true.  
	\[
	   E_{n+1}(x) - E_n(x) = -\frac{f^{(n+1)(a)}}{(n+1)!}(x-a)^{n+1}
	\]
	\[
	   E_{n+1}(x) = \frac{1}{n!} \int_a^x (x-t)^n f^{(n+1)}(x) dt - (- \frac{f^{(n+1)(a)}}{n+1!}(x-a)^{n+1})
	\]
	\[
	= \frac{1}{n!} \int_{a}^{x} (x-t)^n f^{(n+1)}(t) dt - - \frac{f^{(n+1)}(a)}{n!} \int_{a}^{x} (x-t)^n dt
     \]
     \[
  \frac{1}{n!} \int_{a}^{x} (x-t)^n [f^{(n+1)}(t) - f^{(n+1)}(a)dt]
     \]
     \[
	u = f^{(n+1)}(t) - f^{(n+1)}(a)
     \]
     \[
	Taking \frac{dv}{dt} = -(x-t)^n
     \]
     Applying uv rule,
     \[
	E_{n+1}(x) = \frac{1}{n!} \int_{a}^{x} [f^{n+1}(x) - f^{n+1}(a)] \frac{(x-t)^{n+1}}{n+1} + \frac{1}{n!} \int_{a}^{x} \frac{(x-t)^{n+1}}{n+1} f^{(n+2)}(t) dt
     \]
     We get,
     \[
	   E_n = \frac{1}{n!}\int_a^x (x-t)^n f^{(n+1)}(t) dt
     \]
\end{proof}
\begin{note}
	 The word neighbourhood here means it's just an open interval. 
	 \[
	    (a - \epsilon, a + \epsilon)
	 \]
\end{note}
% section Error (end)
\pagebreak
\section{Error Estimation} % (fold)
\label{sec:error_estimation}
If we know the bounds on the derivative of a function, we can estimate the boundaries on the error of the function. 
\begin{theorem}
   If the $(n+1)^{th}$ derivative of $f$ satisfies the inequalities, 
   \[
      m \leq f^{(n+1)}(t) \leq M
   \]
   for all $t$ in some interval containing $a$, then for every $x$ in this interval, we have the following inequalities/estimates
   \[
      \frac{m(x-a)^{n+1}}{(n+1)!} \leq E_n(x) \leq \frac{M(x-a)^{n+1}}{(n+1)!} \text{ if } x>a
   \]
   \[
      \frac{m(a-x)}{(n+1)!}  \leq (-1)^{n+1} E_n(x) \leq \frac{M(a-x)^{n+1}}{(n+1)!} \text{ if } x<a
   \]
\end{theorem}
\begin{proof}
   i) $x > a, t \in [a, x], then (x-a)^n > 0$
   We know,
   \[
      m \leq f^{(n+1)}(t) \leq M
   \]
   Performing an integral operation and multiplying by $\frac{(x-a)^{n+1}}{(n+1)!)}$
   \[
      \int_{a}^{x}m \frac{(x-t)^n}{n!}dt \leq \int_{a}^{x} \frac{(x-t)^{n}}{n!} f^{(n+1)} dt \leq \int_{a}^{x} M \frac{(x-t)^n}{n!} dt
   \]
   Upon integration, and substituting $E_n(x)$
   \[
      \frac{m(x-a)^{n+1}}{(n+1)!} \leq E_n(x) \leq \frac{M(x-a)^{n+1}}{(n+1)!} 
   \]
   ii) $x<a, t \in [x,a], (-1)^n (x-t)^n \geq 0 $
   \[
      m \int_{x}^{a} (-1)^n \frac{(x-t)^n}{n!} \leq (-1)^n \int_x^a f^{(n+1)}  \frac{(x-t)^n}{n!} \leq (-1)^n \int_x^a M \frac{(x-t)^{n+1}}{n!}dt
   \]
   Upon applying the limits and switching the limits for the middlemost term, we get 
   \[
      \frac{m(a-x)}{(n+1)!}  \leq (-1)^{n+1} E_n(x) \leq \frac{M(a-x)^{n+1}}{(n+1)!} 
   \]
\end{proof}
The Taylor Polynomial is used in the approximation of a polynomial. 
For a practical use, we will solve the integral, this integral has no formal way of solving for.
\[
   \frac{2}{\sqrt{\pi}} \int_0^{\frac{1}{2}} e^{-t^2} dt = erf(x) [Error Function]
\]
Taking the case of $e^x$
We know that,
\[
	e^x = 1 + x + \frac{x^2}{2} + \frac{x^3}{6} + \frac{x^4}{24}
\]
   Taking some interval [-c,0]	, $c > 0 \in R$
   When $x<0$
\[
	m \leq f^{(n+1)}(x) \leq M
\]
\[
   e^{-c} \leq f^{(n+1)}(x) \leq 1
\]
\[
   e^{-c} \leq e^x \leq 1
\]
Now we can use the theorem to find the bounds of the error function. 

\[
   \frac{e^{-c}{-x}^5}{5!} \leq (-1)^5 E_4(x) \leq \frac{(-x)^5}{5!}
\]
Since c is an arbitrary number and $e^{-c}$ forms an almost negligible value, we take that entire term to be zero, and we are more interested in the upper bound.
\[
   0 \leq (-1)^5 E_4(x) \leq \frac{(-x)^5}{5!}
\]
Substituing $-t^2$ instead of $x$ 
\[
E_4(-t^2) \geq -\frac{t^{10}}{5!}
\]
\[
   0 \leq \frac{t^10}{5!} \leq \frac{1}{2^{10}} \frac{1}{5!} = 8.14x10^{-6}
\]
\[
   0 \leq \frac{t^{10}}{5!} \leq 8.14\times10^{-6}
\]
Substituting $-t^2$ in the Taylor polynomial, we get the final answer to be 0.4612548616898148
\section{Little-Oh!} % (fold)
We learned that we can write a function $f(x)$ as 
\[
   f(x) = \sum_{k=0}^{n} \frac{f^{(k)}(a)}{k!}(x-a)^k + E_n(x)
\]
Taking an interval in the neighbourhood of a constant $a$, $a-c, a+c$, if the function $f(x)$ is bounded by this interval, then the Error function is also bounded by,
\[
   f^{n+1} (x) \leq M
\]
\[
   |E_n(x)| \leq \frac{M|(x-a)^{n+1}|}{(n+1)!}
\]
\[
 0 \leq  |\frac{E_n(x)}{x-a} | \leq \frac{M | x-a |^n}{(n+1)!}
\]
When $x \rightarrow a$ then usually the value becomes undefined but, on the right hand side, the function is tending to zero faster.

What this means is that,

\[
   \frac{E_n(x)}{(x-a)^n} \rightarrow 0 
\]
\[
	E_n(x) = o((x-a)^n
\]
\begin{definition}
	Assumg $g(x) \neq 0$ for all $x \neq a$ in some interval containing a (neighbourhood of a). The notation $f(x) = o(g(x))$ as $x \rightarrow a$ means that
	\[
	   \lim_{x \rightarrow a} \frac{f(x)}{g(x)} = 0
	\]
\end{definition}
$f(x)$ may be approximated near $a$ by a polynomial in $(x-a)$ of degree n, and the error in this approximation is of smaller order than $(x-a)^n$ as $x \rightarrow a$
\[
   f(x) = \sum_{k=0}^{n} \frac{f^{(k)}(a)}{k!}(x-a)^k + o(x-a)^n as x \rightarrow a
\]
The further away you get from $a$, the less accurate the Taylor polynomial becomes
\begin{example}
	 \[
	    f(x) = o(1) \text{ as } x \rightarrow a
	 \]
	 \[
	    \lim_{x \rightarrow a} f(x) = 0
	 \]
\end{example}
\begin{example}
	 \[
	    f(x) = o (x) \text{ as } x \rightarrow 0 
	 \]
	 \[
	    \lim_{x \rightarrow 0} \frac{f(x)}{x} = 0
	 \]
\end{example}
\begin{example}
	 \[
	 	f(x) = h(x) + o(g(x))
	 \]
	 \[
	    \lim_{x \rightarrow a} \frac{f(x) - h(x)}{g(x)} = 0
	 \]
\end{example}
\begin{example}
	 \[
	    \sin{x} = x + o(x) \text{ as }  x \rightarrow 0 
	 \]
	 \[
	    \lim_{x \rightarrow 0} \frac{\sin{x}- x}{x} = 0
	 \]
	 \[
	    \lim_{x \rightarrow 0} \frac{\sin{x}}{x} = 1
	 \]
\end{example}
\begin{theorem}[Algebra of Little-$o$]
   As $x \rightarrow a$, we have the following,
   \[
   	a) o(g(x)) \pm ox)(g(x)) = o(g(x))
     \]
   \[
      b) o(c(g(x)) = o(g(x) \text{ if } c \neq 0
   \]
   \[
   	c) f(x). o(g(x)) = o(f(x).g(x))
   \]
   \[
   	d) o(o(g(x)) = o(g(x)
   \]
   \[
      e) \frac{1}{1+g(x)} = 1 - g(x) + o(g(x)) \text{ if } g(x) \rightarrow 0 as x \rightarrow a
   \]
\end{theorem}
\begin{example}
\[
   tan x = x + \frac{1}{3} x^3 + o(x^3)
\]

\end{example}
% section Little-Oh! (end)
\begin{enumerate}
   \item Taylor Series will be done as a video

\item Fourier Transforms will be done as a video

\item Laplace Transforms will be done as a video
\end{enumerate}
% section  (end)
\chapter{Optimization}
The background needed here is single variable calculus(18.01SC from MIT). What we require for the course is multi-variable calculus.

\section{Resources To Refer To}
	\begin{enumerate}
	\item Robert Ghrist - series on single variable and multivariable calculus.
	\item 3Blue1Brown - a series of videos on multivariable calculus. 
      \end{enumerate}
\section{Differential calculus of scalar and vector fields} % (fold)
We're dealing with transformations. 
\[
	T: V \rightarrow W
\]
These transformations are either linear or nonlinear. Here V and W are subsets of $\mathbb{R}^n$ and $\mathbb{R}^m$ respectively.

Taking for example, 
Where n and m = 1. An example would be $y = f(x), \text{where } x \in \mathbb{R}, y \in \mathbb{R}$. This would be known as a real valued function of a real variable.

When n = 1, m $>$ 1: Vector valued function of a real variable.

When n $>$ 1, m = 1: Real valued function of a vector variable. Examples include temperature and norm. (Scalar field)
\label{sec:differential_calculus_of_scalar_and_vector_fields}

When n and m $>$ 1: Vector valued functions of a vector variable. (Vector field)
% section Differential calculus of scalar and vector fields (end)

On dealing with spaces, we use notations to represent a vector. 
\begin{enumerate}
	\item $R^2: (x,y)$
	\item $R^3: (x,y,z)$
	\item $R^n: (x_1, \dots, x_n)$
\end{enumerate}

\section{Open ball and open sets}
\begin{note}
In single variable calculus, we have a notation for representing intervals using (a,b) and [a,b]. We use these intervals when discussing differentiability over an open interval. While continuity is measured over a closed interval. \textbf{Why? Homework for today.}
My answer - Differentiability can only be checked on a continuous function. If continuity is true from $[a,b]$ then we cannot say anything about the LHL and RHL of $a$ and $b$ respectively, extending the argument further, we cannot say anything about the LHD and RHD of $a$ and $b$.
\end{note}

\begin{itemize}
   \item Let \underline{a} be a given point in $R^n$ and let r be a given positive number. The set of all points \underline{x} in $R^n$ such that
      \[
	 \|\underline{x} - \underline{a}\| < r
      \]
 is called an open n-ball of radius r and center a.
\item We denote this set by $B(\underline{a})$ or $B(\underline{a};r)$.
   \begin{itemize}
 \item in $R^1$: open interval 
\item in $R^2$: solid disc
\item in $R^3$: solid sphere
\item in $R^n$: n-ball
   \end{itemize}
\end{itemize}
\begin{definition}[Interior Point]
   Let S be a subset of $\mathbb{R}^n$, and assume that $\underline{a} \in S$. Then $\underline{a}$ is called an interior point of S if there is an open n-ball with center at $\underline{a}$, all of whose points belong to S

   Notation: int S(the set of all interior points of S)
\end{definition}
\begin{definition}[Open Set]
   A set S in $\mathbb{R}^n$ is open if all its points are interior points. In other words, S is open if and only if S = int S
\end{definition}
An open set containing point \underline{a} is also called as neighborhood of \underline{a}

If $A_1$ and $A_2$ are subsets of $R^1$, their Cartesian product $A_1 \times A_2$ is the set in $R^2$ defined by 
\[
   A_1 \times A_2 = \{(a_1,a_2)| a_1 \in A_1, a_2 \in A_2\}
\]
\begin{theorem}
   If $A_1$ and $A_2$ are open sets, then $A_1 \times A_2$ is also open in $R^2$
\end{theorem}
\begin{proof}
There is a 1-ball around $a_1$ in $A_1$ and for $a_2$ in $A_2$. These balls are represented by $B(a_1;r_1)$ and $B(a_2;r_2)$. Where r = min($r_1,r_2$). The 2-ball is $x_1, x_2 \in B(a;r)$ 
\begin{align*}
   \|x - a\| < r \rightarrow & |x_1 - a_1| < r_1 \\
			     & |x_2 - a_2| < r_2 \\ 
			     & x_1 \in B(a_1; r_1) \rightarrow x_1 \in A_1 \\ 
			     & x_2 \in B(a_2; r_2) \rightarrow x_2 \in A_2 \\
\end{align*}
Since $x_1$ and $x_2$ are interior points of $A_1$ and $A_2$, the cartesian product is $A_1 \times A_2$
\end{proof}

We were talking about transformations on  vector spaces, but on specific vector spaces with an additional structure added. These vector spaces are known as \textbf{normed linear spaces} or vector spaces with an inner product operation. 

Refresher:
\[
	T : V \rightarrow W, V \subseteq R^n , W \subseteq R^m
\]

\begin{definition}[Exterior And Boundary]
   A point x is said to be exterior to a set S in $\mathbb{R}^n$ if there is an n-ball B($\vec{x}$)	containing no points of S. The set of all points in $\mathbb{R}^n$ exterior to S is called the exterior of S and is denoted by ext S. 
   \[
   	\| x - a \| > r
   \]

   A point which is neither exterior nor interior is said to be the boundary point of S.
   
   The set of all boundary points of the set S is called the boundary of S. It is denoted by $\partial S$
   \[
   	\| x - a \| = r
   \]
\end{definition}
\section{Limits And Continuity} % (fold)
\[
   \lim_{x \rightarrow a} f(x) = b
\]
Suppose a small number $\epsilon$ ($\epsilon > 0, \epsilon \in \mathbb{R}$) and$|x - a| < \epsilon \rightarrow |f(x) - b| < \delta$) where $\delta$ and $\epsilon$ are two positive small numbers.

This is what a limit means.
% section Limts And Continuity (end)
Consider a function $f : S \rightarrow \mathbb{R}^m$ where $S \subseteq R^n$. If $\vec{a} \in \mathbb{R}^n$ and $\vec{b} \in \mathbb{R}^m$, we write 
\[
   \lim_{x \rightarrow a} f(x) = \vec{b}
\]
to mean that
\[
   \lim_{\|x - a\|} \|f(\vec{x} -\vec{b})=0
\]

In $\mathbb{R}^2$:
\[
   \lim_{(x,y) \rightarrow (a,b)} f(\vec{x}) = \vec{b}
\]

In $\mathbb{R}^3$:
\[
   \lim_{(x,y,z) \rightarrow (a,b,c)} f(\vec{x}) = \vec{b}
\]

A function is said to be continuous at $\vec{a}$ if f is defined at $\vec{a}$ and if 
\[
   \lim_{\vec{x}\rightarrow \vec{a}}f(\vec{x})  = \vec{b}
\]
\begin{theorem}
   If $\lim_{x \rightarrow a} f(x) = b$ and $\lim_{x \rightarrow a} g(x) = c$ 

   \begin{enumerate}
      \item $\lim_{x \rightarrow a}[f(x) + g(x)] = b + c$
      \item $\lim_{x \rightarrow a}[\lambda (x)] = \lambda b$ 
      \item $\lim_{x \rightarrow a}f(x)\cdot g(x) = b\cdot c$
   \item $\lim_{x \rightarrow a} \|f(x)\| = \|b\|$
   \end{enumerate}

\end{theorem}
A function $f$ is continuous at a point, if and only if each component $f_k$ is continuous at that point. 
\[
   \vec{f} = (f_1(x),f_2(x),\dots,f_m(x))
\]
	
\begin{theorem}
	Let $f$ and $g$ be the functions such that the composite function $fog$ is defined at $a$, where
	\begin{displaymath}
	   (f o g)(\vec{x}) = \vec{f}[\vec{g}(x)]
	\end{displaymath}
	If $g$ is continuous at $a$ and if $f$ is continuous at $g(a)$, then the conmposition $fog$ is continous at $a$.
	
\end{theorem}
Example - 
$sin(x^2y), \log(x^2 + y^2), \frac{e^{x+y}}{x+y}$

Take this example
\begin{align*}
   f(x,y) &= \frac{xy}{x^2 + y^2} \text{when } (x,y) \neq 0\\
	  &= 0, \text{when } (x,y) = 0 
\end{align*}

When taking the limit on this function, $\lim_{x \rightarrow a} f(x)$ Take the x-axis. Then the function would give 0. Take the y-axis, then the function would also give 0. But, when $x=y$, the function  gives $\frac{1}{2}$ until $x = y = 0$, the function gives 0. 

Therefore this function is not continuous.

\section{Differentiation(Scalar Fields)}
A scalar field is a real valued function taking a vector variable.

\[
   T : V \rightarrow W, \text{where } V \subseteq \mathbb{R}^{n}, W \subseteq \mathbb{R}
\]

The function, 
\[
   f(\vec{x}) \ \ f:S \rightarrow \mathbb{R}, S \subseteq \mathbb{R}^n
\]

$\vec{a}$ is an interior point of S.

In the figure, 

\[
   \| \vec{a} - hy - \vec{a} \| = \| h\vec{y} \| = |h|\|\vec{y}\|
\]
Where, 
\[
   |h|\|\vec{y}\| < r
\]
Or $|h|\|y\|$ is an interior point

We know that for single-variable calculus,
\[
   f'(x) = \lim_{h \rightarrow 0}[\frac{f(x+h) - f(x)}{(x+h-x)}]
\]

This expression is known as the difference quotient. 

Using these terms, we can arrive at a definition for the difference quotient and derivative for a scalar field. 

\[
   \frac{f(\vec{a}+h\vec{y}) - f(\vec{a})}{h}
\]
\begin{definition}[Derivative]
   Given a scalar field, $f:S \rightarrow \mathbb{R}$, where S $\subseteq \mathbb{R}^n$. Let $a$ be an interior point of S and let $\vec{y}$ be an arbitrary point in $\mathbb{R}^n$. The derivative of $f$ at $a$ with respect to $\vec{y}$ is denoted by the symbol $f'(a;y)$ and is defined by the equation,
   \[
      f'(a;y) = \lim_{h \rightarrow 0} \frac{f(\vec{a}+h\vec{y} - f(\vec{a}))}{h}
   \]
when the the limit exists.		
\end{definition}
e.g. Let's say the direction $\vec{y} = 0$

$$f'(a;0) = \vec{0}$$

e.g. f is linear
$$f(\alpha a + \beta b) = \alpha f(a) + \beta f(b)$$

$$f'(\vec{a};y) = f(\vec{y})$$

e.g. $f(\vec{x}) = \|x\|^2$
\begin{align*}
   f(\vec{x} + h\vec{y}) &= \|\vec{x}+h\vec{y}\|^2\\ 
			 &= (\vec{x} + h\vec{y}) \cdot (x+hy) \\ 
			 &= \| \vec{x}\|^2 + 2h\vec{x} \cdot \vec{y} + h^2 \| \vec{y} \|^2 \\ 
   f(x) &= \|x^2\| \\
   f'(x;y) &= \lim_{h \rightarrow 0} \frac{2h\vec{x}\cdot \vec{y} + h \|y\|^2}{h} = 2x\cdot \vec{y} + h\|y\| = 2x\cdot y \\
\end{align*}

\section{Directional Derivative} % (fold)
\label{sec:directional_derivative}
\begin{definition}
   If $\vec{y}$ is a unit vector, the derivative $f'(\vec{a};\vec{y})$ is called the directional derivative of $f$ at $\vec{a}$. In particular, if $\vec{y} = \hat{e}^k$(the kth coordinate unit vector) The directional derivative $f'(\vec{a}; \hat{e}_k)$ is called the partial derivative with respect to $\hat{e}_k$ and is also denoted by $D_k f(\vec{a})$. Thus,
   \[
   	D_k f(\vec{a}) = f'(\vec{a};\hat{e}_k)
   \]
   \[
      D_1 f(\vec{a}) = \frac{\partial f(\vec{a})}{\partial x_1}
   \]
\end{definition}

For derivatives of multiple variables,
\begin{align*}
   \frac{\partial }{\partial y} (\frac{\partial f}{\partial x}) &= \frac{\partial f}{\partial y \partial x} = D_{21} f(\vec{a})
\end{align*}
For the function $y=f(x)$, let's say we remove the limit from the difference quotient, we'd still need to account for the error.
\begin{align*}
   f'(a) = [\frac{f(a+h) - f(a)}{h}] + E(a,h) \\
   \implies f'(a) - \frac{f(a+h) - f(a)}{h}] = E(a,h)
\end{align*}
As $h \rightarrow 0, E(a,h) \rightarrow 0$
\begin{align*}
   f(a+h) - f(a) = \frac{f(a+h) - f(a)}{h}.h &=(f'(a) 0 = 0) \\
   \implies f(a+h) = f(a) + hf'(a) + hE(a,h)
\end{align*}
Referring back to chapter 3, this is a way to write $o(x)$ notation. 
\[
	f(a+h) - f(a) = hf'(a) + o(h)
\]
Taking this to multivariable calculus.
The argument can be made,
\begin{align*}
   f(\vec{a} + h\vec{y}) - f(\vec{a}) &= (\frac{f(a+hy)-f(a)}{h}.h) \\
\end{align*}
The argument can be made that when $h$ tends to $0$, $E(\vec{a};h)$ also tends to zero.
% section Directional Derivative (end)
e.g. 
\begin{align*}
   f(x,y) &= \{\frac{xy^2}{x^2 + y^4}; (x,y \neq 0)\} \\
	  &= \{0; \ (x,y) = (0,y)\} \\
	 \vec{a} + h\vec{y} &= (0,0) + h(a,b) = h(a,b) \\
	 f(\vec{a} + h\vec{y}) &= f(h(a,b)) = f(ha,hb) = \frac{hab^2}{a^2 + h^2 b^4} \\
	 \frac{f(a+h) - f(a)}{h} &= \frac{ab^2}{a^2+h^2b^4}; \\
	 \lim_{h\rightarrow 0} f'(a;y) &= \frac{b^2}{a} \\
\end{align*}

The problem arises when we look at the function. We have made the argument that when the derivative of a function exists, it implies continuity. But that does not seem to be true in multivariable calculus. 

Take this function, and put in $(k,\sqrt{k})$. The function gives $\frac{1}{2}$ everywhere except at 0. This means that the function is discontinuous, by our understanding of continuity.


Let's try extending the definition within single variable calculus, in such a way that the existence of a derivative implies continuity.
\[
   y = f(x) , x,y \in \mathbb{R}
\]
For single-variable calculus, this works as a definition for the derivative.
\[
	f(x+h) - f(x) = hf'(x) + o(h)
\]
What term would fit that same description for multi-variable calculus?
\[
   f(\vec{a} + h\vec{v}) - f(\vec{a}) = (?) + o \|v\|
\]
\section{Total derivative}
\begin{definition}[Total Derivative]
   We say that $f$ is differentiable at $\vec{a}$, if there exists a linear transformation. 
   \[
      T_{\vec{a}} : \mathbb{R}^n \rightarrow \mathbb{R}
   \]
   from $\mathbb{R}^n$ to $\mathbb{R}$, and a scalar function $E(\vec{a};\vec{v})$ such that 
   \[
      f(\vec{a} + \vec{v}) = f(\vec{a}) + T_{\vec{a}}(\vec{v}) + \|\vec{v}\| E(\vec{a};\vec{v})
   \]
   for $\|v\| < r$, where $E(\vec{a};\vec{v}) \rightarrow 0$ as $\|\vec{v}\| \rightarrow 0$. The linear transformation $T_{\vec{a}}$ is called the total derivative of $f$ at $\vec{a}$ 
\end{definition}
\begin{align*}
   f(\vec{a} + \vec{v}) &= f(\vec{a}) + T_{\vec{a}}(\vec{v}) + \|\vec{v}\| E(\vec{a};\vec{v}) &\text{(By definition)}\\
   \frac{f(\vec{a} + h\vec{y} - f)\vec{a}}{h} &= T_a(y) + \frac{|h|\|y\|E(\vec{a};h\vec{y}}{h} &\text{(Dividing both sides by $h$)} \\
f'(\vec{a};\vec{y}) &= T_{\vec{a}}(\vec{y})  &(\lim_{h \rightarrow 0}) \\
\end{align*}

\begin{theorem}
   Assume $f$ is differentiable at $\vec{a}$, with total derivative $f'(a;y)$ exists for every $\vec{y}$ in $\mathbb{R}^n$ and we have,
   \[
      T_{\vec{a}}(\vec{y}) = f'(\vec{a};\vec{y})
   \]
   Moreover, $f'(a;\vec{y})$ is a \textbf{linear combination} of $\vec{y}$. In fact, if $\vec{y} = (y_1,\dots, y_n)$, we have 
   \[
      f'(\vec{a};\vec{y}) = \sum_{k=1}^{n} D_k f(\vec{a}) y_k
   \]
\end{theorem}
\begin{proof}
\begin{align*}
   T_{\vec{a}}(\vec{y}) &= y_1 T_{\vec{a}}(\hat{e_1}) + y_2 T_{\vec{a}}(\hat{e_2}) +  \dots + y_n T_{\vec{a}}(\hat{e_n}) \\
   T_{\vec{a}}(\sum_{k=1}^{n}y_k\hat{e_k}) &= \sum_{k=1}^{n} y_k T_{\vec{a}}(\hat{e_k}) \\
   f'(\vec{a};\hat{e_k}) &= T_{\vec{a}}(\hat{e_k}) \\
   f'(\vec{a};\vec{y}) &= \sum_{k=1}^{n} D_k f(\vec{a}) y_k  \\
\end{align*}
\end{proof}

\begin{definition}[Gradient Operator]
   The gradient operator is a linear operator that helps calculate the total derivative.
\begin{align*}
   T_{\vec{a}}(\vec{y}) &= \nabla f(\vec{a}) \cdot (\vec{y}) \\
   \text{where}, \nabla &= \frac{\partial ()}{\partial x_1} \hat{e_1} + \frac{\partial ()}{\partial x_2} \hat{e_2} \dots \frac{\partial ()}{\partial x_n} \hat{e_n} \\
   \nabla f &= \frac{\partial f}{\partial x_1}\hat{e_1} + \frac{\partial f}{\partial x_2} \hat{e_2} + \dots + \frac{\partial f}{\partial x_n} \hat{e_n} \\
\end{align*}
\end{definition}
This gradient, is a vector.
The total derivative, 
\[
   f'(\vec{a};\vec{y}) = \nabla f \cdot \vec{\hat{y}}
\]
\begin{note}
   \begin{align*}
      &\nabla \times \vec{v} \text{ is known as a curl.}\\
      &\nabla \cdot \vec{v} \text{ is known as a divergence.}
   \end{align*}
	 
\end{note}
If the partial derivatives for a function exist and they are continuous, then the total derivative exist. If the total derivative exists, then the partial derivatives exist.

\section{Chain Rule}
For one variable, 
\[
	g(t) = f(p(t))
\]
When there are multiple functions, we apply the chain rule.
\[
   g'(t) = f' p' ; f' = \frac{df}{dp}, p' = \frac{dp}{dt}
\]
\[
   \frac{dg}{dt} = \frac{df}{dp} \frac{dp}{dt}
\]
Taking this to vectors,
\[
   g(t) = f(\vec{r}(t))
\]

The derivative of a vector function,
\[
   \vec{r} (t) - (x(t),y(t))
\]
\[
   \vec{r'}(t) = (x'(t),y'(t))
\]
\begin{theorem} 
 Let f be a scalar field defined on an open set $S 
   \subseteq R^n$, and let $\vec{r}$ be a vector-valued function which maps an interval $J$ from $\mathbb{R}^1$ into S. Define the composite function $g = f o \vec{r}$ on J, by the equation.
\[
   g(t) = f(\vec{r}(t)) \text{ if } t \in J
\]	
Let t be a point in J at which $\vec{r'}(t)$ exists and assume that $f$ is differentiable at $\vec{r}(t)$. Then $g'(t)$ exists and is equal to the dot product,
\[
   g'(t) = \nabla f(\vec{a}) \cdot \vec{r'}(t), \text{ where } \vec{a} = \vec{r}(t)
\]
\end{theorem}
\begin{proof}
\begin{align*}
g(t) &= f(r(t)) \\
g(t+h) - g(t) &= f(r(t+h)) - f(r(t)) \\
	      &= f(\vec{y} + \vec{a}) - f(\vec{a}) &\text{$y = \vec{r}(t+h)- \vec{r}(t)$ and $\vec{a} = \vec{r}(t)$} \\
\frac{g(t+h) - g(t)}{h} &= \nabla f \cdot \frac{\vec{r}(t+h) - \vec{r}(t)}{h} + \frac{\|y\|}{h} E(\vec{a};\vec{y}) &\text{Applying limit $h \rightarrow 0$}\\
g'(t) &= \nabla f \cdot r'(t)  + 0 \\
\end{align*}
\end{proof}
\section{Vector Fields(Quick Recap)}
\[
	y = f(x)
\]
For a scalar function over a real variable, with a derivative.
\[
	f'(a) = \lim_{h \rightarrow 0}\frac{f(a+h) - f(a)}{h}
\] 
\[ 
	f(a+h) - f(a) = hf'(a) + o(h)
\]
For a scalar field, where $f : S \subseteq R^n \rightarrow R$
\[ 
f'(\vec{a};\vec{y}) = \lim_{h \rightarrow 0} \frac{f(\vec{a} + h \vec{y}) - f(\vec{a})}{h}
\]
With $\|y\| = 1$, to create directional derivatives.
But the existence of a directional derivative did not guarantee continuity at the point. There were still functions that seemed to have a derivative that weren't continuous at that point.
From this arose the total derivative, $\nabla f$ which was a linear combination of all the directional derivatives along the standard basis (where $y = \hat{e_k}$) of a function.

For functions with vector fields: $\vec{f} : S \subseteq R^n \rightarrow R^m$

Where, 

\[ \vec{f} = (f_1,f_2,\dots, f_m)\]
and 
\[\vec{f'} = (f_1',f_2',\dots, f_m')\]

Now we'll have, 
\[ \vec{f}(\vec{a} + h \vec{y}) - \vec{f}(\vec{a}) = \vec{T}_a (h \vec{y}) + \|hy\| \vec{E}(\vec{a};h\vec{y})) \]
\[
	\vec{T}_{\vec{a}}(\vec{y}) = \vec{f'}(a;\vec{y})
\]
\[ 
	\vec{f'} = \sum_{k=1}^m f'_k \hat{e_k} = \sum \nabla f_k \cdot \vec{y} \hat{e_k}
\]
This might look familar , it's the directional derivative of that function along the direction of that function. This can be written as,

\[
\nabla f_k = 
\begin{bmatrix} \frac{\partial f_k}{\partial x_1} \\ \dots \\ \frac{\partial f_k}{\partial x_n} \\ \end{bmatrix}^T \] \[
\begin{bmatrix}
	\frac{\partial f_1}{\partial x_1} & \dots & \frac{\partial f_1}{\partial x_n} \\
	\vdots & \ddots & \vdots \\
	\frac{\partial f_m}{\partial x_1} & \dots & \frac{\partial f_m}{\partial x_n}
\end{bmatrix}
\begin{bmatrix}
	y_1 \\ 
	\vdots \\
	y_n \\
\end{bmatrix}
\]
Let's take an example,

\[
	f(x,y) = x^2 - 3y 
\]
\[ 
	\text{Parabola}: y = x^2 - x + 2
\]
\[
	P = (1,2)
\]

What we have to do is find the derivative of the scalar field along the parabola, evaluated at the point $P$
To do this we first find, 
\begin{align*}
	f'(\vec{a}, \vec{y}) &= \nabla f \cdot \vec{y} \\
	\|y\| &= 1 \\
	\nabla f &= (2x-3y,-3x) \\
		 &= (\frac{\partial f}{\partial x}, \frac{\partial f}{\partial y})
\end{align*}
Now we need to parametrize the parabola, where $x = t$
\begin{align*}
	x &= t \\
	y &= t^2 - t + 2\\
	\vec{r}(t) &= (t, t^2 - t + 2) \\
\end{align*}
Evaluating at point P.
\begin{align*}
	\vec{r'}(t) = (1,2t-1)\\
	\vec{r'}(t) = (1,1); \hat{r}(t) = \frac{(1,1)}{\sqrt{2}} \\
\end{align*}
What we're finding is the directional derivative along the tangent of the given curve. 
\[
	\nabla f \cdot \vec{y}
\]
So the final answer is,
\[ 
	\begin{bmatrix}
		-4 & -3
	\end{bmatrix}
	\begin{bmatrix}
		\frac{1}{\sqrt{2}}	\\
		\frac{1}{\sqrt{2}} \\
	\end{bmatrix}
	= \frac{-7}{\sqrt{2}}
\]
\section{Extremums}
For a scalar function for a real variable, $f(x)$
The Taylor polynomial for a first degree derivative is,
\[ 
	f(a+h) - f(a) = hf'(a) + o(h)
\]
But what do we know about maximas and minimas? The function evaluated within that neighbourhood always yields a difference that is positive.

Yet, h can be positive or negative. What this means is that, the derivative at those points must \textbf{necessarily} be 0.

Similarly for scalar fields, 
\[ 
	f(\vec{a} + h\vec{y}) - f(\vec{a}) = \nabla f \cdot \vec{y} + \|y\|E(\vec{a};h\vec{y})
\]
The partial derivatives are 0, therefore the gradient at an extremum is also 0.
\begin{theorem}
	Let $f$ be a scalar field with continuous second-order partial derivatives $D_{ij}f$ in an n-ball $B(\vec{a})$. Then for all $\vec{y} \in \mathbb{R}^n$ such that $\vec{a} + \vec{y} \in B(\vec{a})$, we have
\[ 
	f(\vec{a} + \vec{y}) - f(\vec{a}) = \nabla f(\vec{a}) \cdot \vec{y} + \frac{1}{2!} \vec{y}^T H(\vec{a} + c \vec{y})\vec{y}, \text{where } 0 < c< 1
\] 
or 
\[
f(\vec{a} + \vec{y}) - f(\vec{a} = \nabla f(\vec{a})\cdot \vec{y} + \frac{1}{2!} \vec{y}^T H(\vec{a}) \vec{y} + \|y \|^2 E_2(\vec{a},\vec{y})
\]
\end{theorem}
\begin{proof}
	Keep $y$ fixed. $g(u) = f(\vec{a} + u\vec{y}$) , $u \in [0,1] \subseteq \mathbb{R} $
\[ f(\vec{a} + \vec{y}) - f(a) = g(1) - g(0) = g^{(1)}(0)(1-0) + \frac{g^{(2)}(c)}{2!} \]
\end{proof}
The Hessian here is a gradient of a gradient.
	\[ 
		\begin{bmatrix}
			f^2_{x_1^2} & f^2_{x_2 x_1} & \dots & f^2_{x_n x_1} \\
			f^2_{x_1 x_2} & \dots & & \vdots \\
			\vdots & \ddots & & \vdots \\
			f^2_{x_1 x_n} & \dots & \dots &  f^2_{x_n^2} \\
		\end{bmatrix}
	\]
\begin{note}
	There is a change in the evaluations, Exam 1, Exam 2 are of 10\% and 5$\%$ each and Exams 3 and 4 are 10\% each. The homework(on Probability and Statistics) is 15\%
\end{note}

\section{Tangent Plane}
\begin{align*}
	f: S \subseteq \mathbb{R}^n \rightarrow \mathbb{R} \\
	L(c) = {x \in S | f(\vec{x}) = c}
\end{align*}

The set L is known as the level set for a given constant c.
For a given level set L(c)
\begin{align*}
&c = f(\vec{r}(t)) = g(t), t \in J \subseteq \mathbb{R}\\
&g'(t) = \nabla f \cdot \vec{r'}(t) = 0 \\
\end{align*}

From this, the gradient is orthogonal to the tangent of that curve.

\begin{definition}[Tangent Plane]
The plane formed by all the tangents orthogonal to the gradient at that point.The gradient has to be non-zero.
\end{definition}

Recall, the equation of a plane is

\[
	\hat{N} \cdot (\vec{x} - \vec{x}_0) = 0
\]

So the tangent plane is defined by,
\[
	\nabla f \cdot r'(t) = 0
\]

\section{Mean-value theorem for derivatives}
Assume that f is continuous everywhere on a closed interval $[a,b]$ and has a derivative at each point of the derivative of the open interval $(a,b)$. Then there exists at least one interior point  $c$ of $[a,b]$ for which,
\[ f'(c) = \frac{f(b) - f(a)}{b-a} \]
\section{Mean-value theorem for integrals}
Assume that f is continuous everywhere on a closed interval $[a,b]$ Then there exists at least one interior point  $c$ of $[a,b]$ for which,
\[ \int_a^b f(x) dx = f(c)(b-a) \]
\section{Weighted Mean-value theorem for derivatives}
Assume that f, g is continuous everywhere on a closed interval $[a,b]$. If g never changes sign in the interval $[a,b]$ Then there exists for some interior point  $c$ of $[a,b]$ for which,
\[ \int_a^b g(x) f(x) dx = f(c) \int_a^b g(x)dx \]

\begin{note}
	Recall Taylor's Approximation Theorem - Assume $f$ has a continuous derivative of order $(n+1)$ in some interval containing $a$. Then, for every $x$ in this interval, we have the Taylor formula, 
	\[ f(x) = \sum_{k=0}^n \frac{f^{(k)}(a)}{k!}(x-a)^k + E_n(x), E_n(x) = \frac{1}{n!} \int_{a}^x (x-t)^n f^{(n+1)} (t) dt \]
\end{note}

The Taylor's approximation has an error function which does not change sign. Let's write it in the form of the weighted mean-value.


\begin{align*}
	E_n(x) &= \frac{1}{n!} \int_a^x (x-t)^n f^{(n+1)}(t) dt \\
	       &= \frac{1}{n!} f^{(n+1)}(c) \int_a^x (x-t)^n dt \\
	       &= \frac{1}{(n+1)!} f^{(n+1)} (c) (x-a)^{n+1}\\
\end{align*}

\[ f(x) = f^{(0)}(a) + f^{(1)}(c)(x-a) \]
\pagebreak

\begin{exercise} 
	\begin{enumerate}
		\begin{align*}
			z = 2 - x^2 - y^2 \\
			z = x^2 + y^2 \\
			z = xy \\
			z = x^3 - 3xy^2 \\
			z = x^2y^2 \\
			z = 1 - x^2 \\
		\end{align*}
	\end{enumerate}
\end{exercise}
	
\begin{theorem}
	Let $A= [a_{ij}]$ be an $n \times n$  real symetric matrix and let the quadratic form, 
	\[ Q(\vec{y}) = \vec{y^T} A \vec{y}\]
	Then we have,
	\begin{enumerate}
		\item $Q(\vec{y}) > 0 \ \forall \ \vec{y} \neq \vec{0}$ iff all eigenvalues of A are positive(A is positive definite)
		\item $Q(\vec{y}) < 0 \ \forall \ \vec{y} \neq \vec{0}$ iff all eigenvalues of A are negative.(A is negative definite)
	\end{enumerate}
\end{theorem}
The motivation behind proving this theorem, look at the Taylor expansion of the derivative,

\[f(\vec{a} + \vec{y}) = f(\vec{a}) + \nabla f(\vec{a}) \cdot \vec{y} + \frac{1}{2} \vec{y^T}H(\vec{a}) \vec{y} + \| \vec{y}^2 \| E_2(a,y)\]

At maximas or minimas,$\nabla f = 0$

The only way to find $y^T H(\vec{a}) y$ is to know at the very least what the sign of the term is.
\begin{proof}
	\begin{align*}
		\text{Let } y^T P &= \vec{x}^T \\
				  &= \vec{x}^T D \vec{x} \\
				  &= \sum_{i} \lambda_i x_i^2 \\
	\end{align*}
\end{proof}
\begin{note}
	A function having continuous second-order partial derivatives means that the mixed partial derivatives are equal. (See Clairaut's Theorem.)
\end{note}
\begin{theorem}
	Let $f$ be a scalar field with continuous second-order partial derivatives, and let $H(\vec{a})$ denote the Hessian at a stationary point $\vec{a}$.Then we have:
	\begin{itemize}
		\item If all the eigenvalues of the $H(\vec{a})$ are positive, $f$ has a local minimum at $\vec{a}$.
		\item If all the eigenvalues of $H(\vec{a})$ are negative, $f$ has a local axima at $\vec{a}$
		\item If $H(\vec{a})$ has both positive and negative eigenvalues, then $f$ has a saddle point at $\vec{a}$.
	\end{itemize}
\end{theorem}
\pagebreak
\begin{theorem}
	Let $\vec{a}$ be a stationary point of a scalar field $f(x_1, x_2)$ with continous second-order partial derivatives in a 2-ball $B(\vec{a})$. Let $A = D_{11}f(\vec{a}), B = D_{12}f(\vec{a}, C = D_{22} f(\vec{a})$ and let $\triangle = \det H(\vec{a}) = \det \begin{bmatrix}A & B \\ B & C\end{bmatrix} = AC - B^2$
	\begin{itemize}
		\item If $\triangle < 0$, $f$ has a saddle.
		\item If $\triangle > 0, A > 0$, then $f$ has a local minimum.
		\item If $\triangle > 0, A < 0$, then $f$ has a local maximum.
		\item If $\triangle = 0$, then the test is inconclusive.
	\end{itemize}
\end{theorem}
\begin{proof}
The characteristic equation for a determinant
	\begin{align*}
		\lambda^2 - (A + C) \lambda + AC-B^2
	\end{align*}
	The trace is (A+C), and $\lambda_1$ and $\lambda_2$, the determinant is $AC-B^2$ or $\lambda_1 \lambda_2$
	If $\lambda_1 \lambda_2< 0$, that means $\lambda_1, \lambda_2$ have opposite signs, look at the previous theorem (c)

For the 2nd and 3rd parts of the proof.
	\[\triangle > 0, A > 0, C > 0, A + C > 0, \lambda_1 + \lambda_2 > 0 \implies \lambda_1 > 0, \lambda_2 > 0\]
	\[\triangle > 0, A < 0, C < 0, A + C < 0, \lambda_1 + \lambda_2 < 0 \implies \lambda_1 < 0, \lambda_2 < 0\]
\end{proof}
\begin{note}
	Probability will be covered using resources from John Tsitsiklis
\end{note}
\end{document} 

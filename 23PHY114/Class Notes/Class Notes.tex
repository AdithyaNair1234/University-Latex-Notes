\documentclass{report}
\input{preamble}
\usepackage{cancel}
\title{\Huge{23PHY114}\\ Class Notes}
\author{\huge{Adithya Nair}}
\date{}
\begin{document}

\maketitle
\newpage% or \cleardoublepage
% \pdfbookmark[<level>]{<title>}{<dest>}
\tableofcontents
\chapter{Solids}
\createintro
\section{Moment Of Area}
The moment of inertia is used to help find the "resistance" to the force, given a specific axis or direction.

\section{Resisting Force And Moments From Supports} % (fold)
\label{sec:resisting_force_and_moments_from_supports}
There are three main kinds of supports - 
\begin{enumerate}
  \item Pin/Hinge - Fixes linear motion but leaves rotation free. 
  \item Roller - Fixes rotation but leaves motion free. 
  \item Clamp - Fixes both linear and rotational motion.
\end{enumerate}
% section Resisting Force And Moments From Supports (end)
Take this case, with a bunch of forces being applied to the given load. 

There are three main things we need to find for this figure. 
\begin{enumerate}
  \item The resultant force acting on this bar fixed to a hinge. 
  \item The support reaction force and moment.
  \item The moment on the object (maximum)
\end{enumerate}
The way to approach the problem is as always,
\begin{enumerate}
  \item Free Body Diagram first. 
  \item Assuming $\Sigma f = 0 $, because the object has no acceleration currently, because it is fixed.
  \item Assuming $\Sigma M = 0$
  \item The point at which the resultant force is applied is found by, 
    \[
      \frac{\int |r|dm}{\int dm}
    \]
\end{enumerate}
\section{Derivation Of The Uniaxial Formula}

\section{Code For Uniaxial Deformation} % (fold)


\label{sec:for_uniaxial_deformation}
Main Subroutines For Uniaxial Deformation
% section For Uniaxial Deformation (end)
\subsection{Finding The Local Stiffness Matrix} % (fold)
\begin{lstlisting}{octave}
function stiffnessLocal = localStiffnessGenerator(E,A,l,theta);
    stiffnessConstant = E*A/l;
    R = [cos(theta) -sin(theta); sin(theta) cos(theta)];
    stiffnessMatrix = [stiffnessConstant 0 -stiffnessConstant 0; zeros(1,4);-stiffnessConstant 0 stiffnessConstant 0; zeros(1,4)];
    R4 = [R zeros(2,2); zeros(2,2) R];
    stiffnessLocal = R4*stiffnessMatrix*R4';
end
\end{lstlisting}
\label{sec:finding_the_local_stiffness_matrix}
% subsubsection Finding The Local Stiffness Matrix (end)
\subsection{Converting The Local Stiffness To A Global Stiffness Matrix}
\begin{lstlisting}{octave}
function stiffnessLocalGlobal = local2Global(stiffnessLocal,node1,node2,nodeCount)
    stiffnessLocalGlobal = zeros(2*nodeCount,2*nodeCount);
    i = 2*node1 - 1;
    j = 2*node2 - 1;
    stiffnessLocalGlobal(i:(i+1),i:(i+1)) = stiffnessLocal(1:2,1:2);
    stiffnessLocalGlobal(i:(i+1),j:(j+1)) = stiffnessLocal(1:2,3:4);
    stiffnessLocalGlobal(j:(j+1),i:(i+1)) = stiffnessLocal(3:4,1:2);
    stiffnessLocalGlobal(j:(j+1),j:(j+1)) = stiffnessLocal(3:4,3:4);
end
\end{lstlisting}
\subsection{Main Loop Through Evaluating The Global Stiffness Matrix}
\begin{lstlisting}{octave}
 A = 0; theta = 0; l = 0; stiffnessLocal = zeros(4,4); nodeAxialForces = zeros(nodeCount,1);
for element = 1:elementCount % For first three bars
    A = areaVector(element);
    theta = angleVector(element);
    l = lengthVector(element);
    stiffnessLocal = localStiffnessGenerator(E,A,l,theta);
    nodeCounter = element*2 - 1;
    stiffnessLocalGlobal = local2Global(stiffnessLocal,nodeVector(nodeCounter),nodeVector(nodeCounter+1),nodeCount);
    stiffnessLocal
    stiffnessLocalGlobal
    stiffnessGlobal += stiffnessLocalGlobal;
end
\end{lstlisting}
\subsection{Applying Boundary Conditions}
\begin{lstlisting}{octave}
selectedVector = [3 5:end]
forceEval = forceVector(selectedVector);
displacementEval = displacementVector(selectedVector);
stiffnessEval = stiffnessGlobal(selectedVector);
displacementEval = stiffnessEval\forceEval;
\end{lstlisting}
\section{Deriving For Bending Deformation}
Taking this example of a bar fixed to the wall, and examining small components of the bar and examining the forces at work on the component, we get a Tension, Moment and Stress.

Taking things along the $\hat{i}$ direction, we get
$$
-T + T + \Delta T = 0
$$
 What does this tell us? Tension is constant. 

While $-S + S + \Delta S - w\Delta x = 0$

$$
\frac{dS}{dx} = \lim_{\Delta x \rightarrow 0} \frac{\Delta S}{\Delta x} = w(x)
$$
We get an expression for the effects of the distributed force on the shear force of an individual object.

Looking at the angular momentum balance,

$$
\sum M_{/P} = 0
$$
$$
-M + (M + \Delta M) \hat{k} + (-\Delta x \hat{i} \times(-T\hat{i} - S \hat{j}) + (\frac{-\Delta x}{2} \hat{i} \times (-w(x) \Delta x \hat{j})) $$ $$
\implies \frac{\Delta M }{\Delta x} + S - w\frac{\Delta x}{2} = 0
$$
Taking the limit on the equations, we get,
$$
\frac{dM}{dx} + S = 0
$$

Thus we get,
$$
\frac{dT}{dx} = 0
$$
$$
\frac{dS}{dx} = w(x)
$$
$$
\frac{dM}{dx} = -S
$$
Now we get to the Finite Element Method for bending.
$$
\theta(x) = v'(x)
$$
We find by simplification,

$$
c_1 = v_l 
$$
$$
c_2 = \theta_l
$$
$$
c_3 = \frac{3}{l^2}(v_r-v_l) - \frac{1}{l} (2\theta_l + \theta_r)
$$
$$
c_4 = \frac{2}{l^3} (v_l-v_r) + \frac{\theta_l+\theta_r}{l^2}
$$

$$
v(x) = H_1(x) v_l + H_2(x)\theta_l (H_3(x)) v_r + (H_4(x)) \theta_r
$$
$$
H_1 = 2(\frac{x}{l}^3) - 3\frac{x}{l}^2 + 1
$$
$$
H_2 = x-2\frac{x^2}{l} + \frac{x^3}{l^2}
$$
$$
H_3 = 3(\frac{x}{l})^2 -2\frac{x}{l}^3
$$
$$
H_4 = \frac{x^3}{l^2} - \frac{x^2}{l}
$$



Since we know that,
$$
\int_0^l EIq''v''dx
$$
$$
v = \underline{H}(x)^T v
$$
$$
q = H(x)
$$

\section{Quick Reference Notes}
\subsection{Derivations}
The governing differential equation is 
\[
  EI v^{(4)} = w(x)
\]

Then we can form a linear relationship
\begin{align*}
  v(x) & - \text{deflection} \\
  v^{(2)}(x) = \theta(x) & - \text{slope} \\
  EIv^{(2)}(x) = M(x) & - \text{bending moment related to curvature} \\
  EIv^{(3)}(x) = \frac{dM}{dx} = V(x) & - \text{transverse shear} \\ 
  EIv^{(4)}(x) = \frac{dV}{dx} = w(x) & - \text{load}
\end{align*}

Where, v is the deflection or displacement of the beam. E is the Young's Modulus and I is 

\subsection{Deriving Stiffness Matrix}
Use Hermite's interpolation formula to derive cubic shape functions for the deflection of beams.

\section{Final Equation}
For a bar, with axial, bending and shear deformation... The final equation is, where R, S and M are Axial, Shear and Moment.

\[
  F = k\Delta x
\]

\[
  \begin{bmatrix}
    R_1 \\ 
    S_1 \\
    M_1\\
    R_2 \\ 
    S_2 \\
    M_2\\
  \end{bmatrix}
  = 
  \begin{bmatrix}
    C_1 & 0 & 0 & -C_1 & 0 & 0 \\ 
    0 & 12C_2 & 6C_2L & 0 & -12C_2 & 6C_2L \\
    0 & 6C_2L & 4C_2L^2 & 0 & -6C_2L & 2C_2L^2 \\
    -C_1 & 0 & 0 & C_1 & 0 & 0 \\ 
    0 & -12C_2 & -6C_2L & 0 & 12C_2 & -6C_2L \\
    0 & 6C_2L & 2C_2L^2 & 0 & -6C_2L & 2C_2L^2 \\
  \end{bmatrix}
  \begin{bmatrix}
    u_1 \\ 
    v_1 \\ 
    \theta_1 \\
    u_2 \\ 
    v_2 \\ 
    \theta_2 \\
  \end{bmatrix}
\]
Where, $C_1 = \frac{EA}{L}$ and $C_2 = \frac{EI}{L^3}$

Rotating the matrix by angle $\theta$

\[
  R = \begin{bmatrix}
    c & s & 0 & 0 & 0 & 0 \\ 
    -s & c & 0 & 0 & 0 & 0 \\ 
    0 & 0 & 1 & 0 & 0 & 0 \\
    0 & 0 & 0 & c & s & 0 \\ 
    0 & 0 & 0 & -s & c & 0 \\ 
    0 & 0 & 0 & 0 & 0 & 1 \\
  \end{bmatrix}
\]

\chapter{Fluids}
\section{Main Contents}
\textbf{Computational Fluids Dynamics}
\begin{enumerate}
  \item Flow in channels and pipes
  \item Flow over objects 
\end{enumerate}

\begin{definition}[Laminar Region]
The region in a fluid where the flow is in a straight line is known as the laminar region 
\end{definition}

\begin{definition}[Turbulent Region]
The region in a fluid where the flow is not smooth is known as the turbulent region
\end{definition}

\textbf{The motivation behind learning CFD}

Computational Fluid Dynamics has been around as long as computers have been around. Using computation to calculate and visualize the flow of fluids is the basis behind wind tunnels. Trials can be done to help optimize solutions for the best aerodynamics.

Examples of data used is the velocity profile, the force and the acceleration. These fluid simulations can take months and be used to simulate the effects of fluids on a given shape.

\section{Terminology}
In engineering terms, fluids are deformable bodies. They take up the shape of its container. Viscosity, is defined as the friction to flow - $\mu$. It dictates the rate of flow of a fluid.

The counterintuitive picture. Imagine a stack of papers on your palm.  On tilting your palm, the first stack of paper goes out then the next then the next. It is dependent on the friction of the piece of paper below it. The last piece of paper has zero friction, and flows out the easiest. But a fluid does not work that way, when some water is placed on your palm, even after flowing out, some imprint of the water remains. 

Think of fluids as layered.

Taking a co-ordinate system, plotting the horizontal and vertical velocity of a fluid, u and v.

\begin{definition}[No Slip Condition]
  A condition where the part of the fluid at the base or in contact with the surface has no horizontal velocity.
  \[
    U(x,0) = 0
  \]
\end{definition}

Unlike solids, fluids have a profile that is counterintuitive to the way we imagine physics.

Counterintuitively, a fluid's profile is parabolic, in a pipe. Closed on all sides, the middlemost component of the fluid moves with the most velocity.

\begin{note}
\textbf{Lagrangian Method} -
The method of following a single point and discovering interesting properties about it is known as the Lagrangian Method

\textbf{Eulerian Method} -
The method of following a single location and performing analyses on that location.
\end{note}
The study of fluids is performed using the Eulerian Method. 

\section{Derivation Of Equations}
Taking the fluid flowing through the figure given. In the Eulerian Method, we focus on a given location. This location here is the outlined box. 

The mass flow rate for a given location is 
\begin{align*}
  \dot{m}_{in} - \dot{m}_{out} &= \frac{dm}{dt} &(\dot{m} = (\frac{d \rho V}{dt})) \\
  \dot{m} &= \frac{d\rho V}{dt} \\
  \dot{m} &= \rho \frac{dV}{dt} \\
          &= \rho Q & \text{(Q - volume flow rate)} \\
\end{align*}
\[
  \rho \frac{dV}{dt} = \Delta y \Delta z . u
\]
\begin{align*}
  &(\dot{m}_{intx} + \dot{m}_{inty}) - (\dot{m}_{outx} + \dot{m}_{outy}) = \frac{dm_{inside}}{dt} \\
  &(\rho \Delta Y \Delta Z u + \rho \Delta X \Delta Z v) - (\rho \Delta Y \Delta Z (U + \Delta U) + \rho \Delta X \Delta Z ( v + \Delta v) = \frac{d(\rho \Delta X \Delta Y \Delta Z)}{dt} \\
  &(\rho \Delta Y \Delta Z u + \rho \Delta X \Delta Z v) - (\rho \Delta Y \Delta Z (U + \Delta U) + \rho \Delta X \Delta Z ( v + \Delta v) = \frac{d(\rho \Delta X \Delta Y \Delta Z)}{dt} \\
  &- \rho \Delta Y \Delta Z \frac{\Delta U}{\Delta X} - \rho \Delta x \Delta z \frac{\Delta V}{\Delta Y} = \Delta X \Delta Y \Delta Z \frac{d \rho}{dt} \\
  &- \rho \frac{\partial U}{\partial x}  - \rho \frac{\partial v}{dy} = \frac{d\rho}{dt} \\
\end{align*}
\section{Recall}
\begin{align*}
  \dot{m}_{in} - \dot{m}_{out} = \dot{m}_{within} & &\text{(Continuity Equation)} \\
  \frac{\partial \rho}{\partial t} = -\rho \frac{\partial u}{\partial x} - \rho \frac{\partial v}{\partial y} & &\text{(Mass Continuity Equation)} or \\
  \frac{\partial \rho}{\partial t} = -\rho \frac{\partial u}{\partial x} - \rho \frac{\partial v}{\partial y} -\rho \frac{\partial w}{\partial z} & &\text{(in 3 dimensions)} \\
\end{align*}

Density is a function of position and time, then it can be expressed as, $\rho(x,y,z,t)$. Instantaneously, density can be observed to change. Similarly with viscosity,

But the question might come about that the density in the mass continuity equation, is taken as constant, since they move outside the derivative. We assume that the density does not change with position rather than changing with time.

\section{Density As A Function Of Time And Position}
\begin{align*}
  1) &\frac{\partial \rho}{\partial t} + \frac{\partial (\rho u)}{\partial y} + \frac{\partial (\rho w)}{\partial z} = 0 \\
  2) &\frac{\partial \rho}{\partial t} + \rho (\frac{\partial u}{\partial x} + \frac{\partial v}{\partial y} + \frac{\partial w}{\partial z}) + \frac{\partial \rho}{\partial x} u + \frac{\partial \rho}{\partial y}v + \frac{\partial \rho}{\partial z} w &\text{(Applying chain rule)} \\
  3) &\frac{d \rho}{dt} = \frac{\partial \rho}{\partial t} + \frac{\partial \rho}{\partial x}\frac{\partial x}{\partial t} + \frac{\partial \rho}{\partial y}\frac{\partial y}{\partial t} + \frac{\partial \rho}{\partial z}\frac{\partial z}{\partial t} &\text{(General form of partial derivatives to total derivatives)} \\
  4) & d\frac{\rho}{dt} + \rho(\frac{\partial u}{\partial x} + \frac{\partial v}{\partial y} + \frac{\partial w}{\partial z}) &\text{(Final expression)} \\
  5) & \vec{\nabla}  = \frac{\partial }{\partial x} \hat{i} + \frac{\partial }{\partial y} \hat{j} + \frac{\partial }{\partial z} \hat{k} &\text{(Differential Operator)} \\
  6) &\vec{v} = u \hat{i} + v \hat{j} + w \hat{k} &\text{(Velocity Vector)} \\
  7)&\frac{d \rho}{dt} + \rho \vec{\nabla}\cdot{\vec{v}} = 0 &{\nabla \cdot v \text{(Divergence of $\vec{v}$)}}
\end{align*}

\section{External Forces On Fluid}
\[
  \vec{\sigma} =
  \begin{bmatrix}
    \sigma_{xx} & \sigma_{xy} & \sigma_{xz} \\
    \sigma_{yx} & \sigma_{yy} & \sigma_{yz} \\
    \sigma_{zx} & \sigma_{zy} & \sigma_{zz} \\
  \end{bmatrix}
\]

Given the fact that a fluid's smallest element does not rotate. Then,
\[
  \Sigma M_{/O} = 0
\]
Then looking at any given face for the element, the shear forces are equal and opposite to each other, from this we get. 
\begin{align*}
  \sigma_{xy} = \sigma_{yx} \\ 
  \sigma_{xz} = \sigma_{zx} \\ 
  \sigma_{yz} = \sigma_{zy} \\
\end{align*}

\section{Variation of stress across a small distance}
insert diagram here(of cube with all forces measured)

\begin{align*}
  &\Sigma \vec{f} = m \vec{a} \\
  &\hat{i}(-\sigma_{xx} \Delta y \Delta z + (\sigma_{xx} + \Delta\sigma_{xx})\Delta y \Delta z + (\sigma_{xy} + \Delta \sigma_{xy})\Delta x \Delta z - \sigma_{xy}\Delta x \Delta z))  + \\
  &\hat{j}(-\sigma_{yy} \Delta x \Delta z + (\sigma_{yy} + \Delta \sigma_{yy}) \Delta x \Delta z + (\sigma_{xy}+\Delta \sigma_{xy})\Delta y \Delta z - \sigma_{xy} \Delta y \Delta z) - (\rho \Delta x \Delta y \Delta z)g \hat{j} =  (\rho \Delta x \Delta y \Delta z)\frac{d \vec{v}}{dt}\\ 
  &\vec{v} = u \hat{i} + v \hat{j} + w \hat{k}  \\
\end{align*}

Dotting along $\hat{i}$ and $\hat{j}$, when $\Delta x \rightarrow 0, \Delta y \rightarrow 0, \Delta z \rightarrow 0$
\begin{align*}
  \frac{\partial \sigma_{xx}}{\partial x} + \frac{\partial \sigma_{xy}}{\partial y} = \rho \frac{du}{dt} \\
  \frac{\partial \sigma_{xy}}{\partial x} + \frac{\partial \sigma_{yy}}{\partial y} - \rho g = \rho \frac{dv}{dt}
\end{align*}
There are certain equations that cannot be derived, they are simply observed and a relation is formed. Such equations are known as the constituent equation. Such as Hooke's Law. 

If the fluid is Newtonian(which means that stress is directly proportional to strain) and isotropic(properties are same in all directions)
\[
  \sigma_{xx} = -p + 2 \mu \frac{\partial u}{\partial x} + \lambda \vec{\nabla}\cdot\vec{v}
\]
$\mu$-viscosity, $\lambda = -\frac{2\mu}{3}$ This constant is related to $\mu$ for many fluids(including air). 
\section{Navier-Stokes Equation For Newtonian Isotropic Fluids} % (fold)
\begin{align*}
  &-\frac{\partial P}{\partial x} + \mu(\frac{\partial^2 u}{\partial x^2} + \frac{\partial^2 u}{\partial y^2} + \frac{\partial^2 u}{\partial z^2}) = \rho \frac{du}{dt} \\
  &-\frac{\partial P}{\partial y} + \mu(\frac{\partial^2 v}{\partial x^2} + \frac{\partial^2 v}{\partial y^2} + \frac{\partial^2 v}{\partial z^2}) - \rho g = \rho \frac{dv}{dt} \\
  &-\frac{\partial P}{\partial z} + \mu(\frac{\partial^2 w}{\partial x^2} + \frac{\partial^2 w}{\partial y^2} + \frac{\partial^2 w}{\partial z^2}) = \rho \frac{dw}{dt} \\
  &\frac{du}{dt} = \frac{\partial u}{\partial t} + u \frac{\partial u}{\partial x} + v\frac{\partial u}{\partial x} + v \frac{\partial u}{\partial y} + w \frac{\partial u}{\partial z}
\end{align*}
Analyzing the given equations, here's a breakdown of what the individual terms mean, 

\begin{align*}
  &\frac{\partial P}{\partial x} &\text{(The force due to the pressure on the fluid)} \\
  &\mu(\frac{\partial^2 u}{\partial x^2} + \frac{\partial^2 u}{\partial y^2} + \frac{\partial^2 u}{\partial z^2}) &\text{(Viscous Forces)} \\
  &-\rho g & \text{(Gravitational forces)}
\end{align*}

Simplifying using the gradient or \textit{nabla} operator, 
\begin{align*}
	-\frac{\partial P}{\partial x} \hat{i} - \frac{\partial P}{\partial y} \hat{j} - \frac{\partial P}{\partial z}\hat{k} &= -\vec{\nabla} p \\ 
    \mu(\frac{\partial^2}{\partial x^2} + \frac{\partial^2}{\partial y^2} + \frac{\partial^2}{\partial z^2})u &= \mu \vec{\nabla} \cdot \vec{\nabla} u  \\
    \rho \frac{d \vec{v}}{dt} = \rho(\frac{\partial \vec{v}}{\partial t} + u \frac{\partial \vec{v}}{\partial x} + v \frac{\partial \vec{v}}{\partial y} + w \frac{\partial \vec{v}}{\partial z}) &= \rho(\frac{\partial \vec{v}}{\partial t} + \vec{v} \cdot \vec{\nabla} \vec{v})
\end{align*}

Resulting with, 
\[
  - \vec{\nabla}p + \mu \nabla^2\vec{v} - \rho g \hat{j} = \rho \frac{d \vec{v}}{dt} = \rho (\frac{\partial \vec{v}}{\partial t} + \vec{v} \cdot \vec{\nabla} \vec{v})
\]
The continuity equation is,
\[
  \frac{\partial \rho}{\partial t} + \vec{\nabla} \cdot (\rho \vec{v}) = \frac{d \rho}{dt} + \rho \vec{\nabla} \cdot \vec{\nabla} = 0
\]
Refer to \textit{Schaum's Fluid Dynamics Chapter 5} for derivations
\section{Special Flows}
\begin{enumerate}
  \item Bernoulli's Equation \[
   \frac{d}{dx}(p + \rho g + \frac{v^2}{2})  = 0
  \]
\item \textbf{Flow between parallel plates} - in this case, $\vec{v} = u \hat{i}$ only, since the water only flows horizontally, which means $\frac{\partial u}{\partial x} = 0$. 
  \begin{align*}
    \cancel{\frac{d \rho}{d t}} + \rho (\vec{\nabla} \cdot \vec{v}) = 0 \\ 
    \rho(\frac{\partial u}{\partial x} + \cancel{\frac{\partial v}{\partial y}} + \cancel{\frac{\partial w}{\partial z}}) = 0 \\
    \frac{\partial u}{\partial x} = 0
  \end{align*}
  The second thing to look at is, 
  \begin{align*}
    -\frac{dp}{dx} + \mu(\frac{\partial^2 u}{\partial y^2}) &= \rho \cancel{\frac{du}{dt}}+ u \cancel{\frac{\partial u}{\partial x}} + v \frac{\partial v}{\partial y} + w \cancel{\frac{\partial u}{\partial z}}) \\
  \end{align*}
  {These are fully developed flows}
  \begin{align*}
    =-\frac{\partial p}{\partial x} + \mu \frac{\partial^2 u}{\partial y^2} &= 0\\
    u \text{ is a function of y}
  \end{align*} 
  \[
    -\frac{d p}{dy} + \mu(\cancel{\frac{d^2v}{dx^2}}+ \cancel{\frac{d^2v}{dy^2}} + \cancel{\frac{d^2v}{dz^2}} - \rho g = \rho \cancel{\rho\frac{dv}{dt}} - \rho g) = \rho \frac{dv}{dt}
  \]
  \begin{align*}
    \therefore -\frac{dP}{dy} - \rho g = 0 \\
    - \frac{-d P}{dy} = \rho g \\
   \int d p = - \int \rho g dy  \\
    p = - \rho g y + c
  \end{align*}
  p is a function of y, but we don't know whether $p$ is a function of x or not. 
  Now, 
  \[
    \mu \frac{d^2 u}{dy^2} = \frac{dp}{dx}
  \]
  Since u is  a function of y, and pressure is a partial derivative of x, the only possible value for both of them to have is a constant C.
  \[
    \frac{d^2 u}{dy^2} = \frac{C}{\mu} 
  \]
  Upon first integration, 
  \[
    \frac{du}{dy} = \frac{C}{\mu} y + C_1 
  \]
  Upon 2nd integration,
  \[
    U(y) = \frac{c}{2\mu} y^2 + C_1 y + C_2
  \]
Using boundary conditions,
\[
  U(0) = 0 \text{(No slip condition)}
\]

If the top part of the plates are fixed, then $U(L) = 0$ but $U(L) = u_0$
Since,
\[
  U(y) = \frac{c}{2\mu}y^2 + c_1 y + c_2
\]
\[
  U(0) = C_2 \implies C_2 = 0
\]
\[
  U(L) = \frac{C}{2\mu}L^2 + C_1 L + C_2
\]
In no slip condition,
\[
  0 = \frac{C}{2\mu}L^2 + C_1 L \implies \frac{C}{2\mu}L = -C_1
\]
If plate is moving, 
\[
  u_0 = \frac{c}{2\mu} L^2 + C_1 L
\]
\[
  \frac{u_0}{L} = \frac{c}{2\mu}L + c_1
\]
\[
  c_1 = \frac{u_0}{L} - \frac{cL}{2\mu}
\]
\[
  U(y) = \frac{c}{2\mu}(y^2) +  c_1 y + c_2 \text{ (Poiseuille flow)}
\]
\end{enumerate}

\section{Computational Fluid Dynamics}
This is a method to solve the Navier-Stokes equation, and all other fluids equations. This equation is a partial differential equation. We take finite difference approximations(using the Taylor series refer to Maths notes.) The beauty of this method is that every partial differential equation can be approximated to a Taylor series.

And we take this further,
\[
  A \vec{x} = \vec{b}
\]
Where, $x$ is a vector composed of $u,v,w,p$. These components are functions of $x,y,z$ and $t$
\subsection{Taylor Series}
Let's look at a case where $U$ is just a function of $y$, $U(y)$. What if I wanted to find what $U(y+\Delta y)$ or some point later. 
The Taylor series of $U(y+\Delta y)$ is,
\[
  U(y+\Delta y) = U(y) + U'(y)\frac{\Delta y}{1!} + U''(y)\frac{\Delta y^2}{2!} + U'''(y)\frac{\Delta y^3}{3!} + \dots
\]
For $U(y - \Delta y)$
\[
  U(y-\Delta y) = U(y) - U'(y)\frac{\Delta y}{1!} + U''(y)\frac{\Delta y^2}{2!} - U'''(y)\frac{\Delta y^3}{3!} + \dots
\]
Let's say I want to find $U'(y)$ 
\begin{align*}
  U'(y) &= \frac{\partial u}{\partial y} = [\frac{U(y+\Delta y) - U(y)}{\Delta y}] - \Delta y (\frac{U''(y)}{2!} - U'''(y)\frac{\Delta y}{3!} + \dots)
\end{align*}
I can assume the remaining terms after $[\frac{U(y+\Delta y) - U(y)}{\Delta y}] $ to be an error of the order of $\Delta y$. What this means is that, the further away I get from y, the higher the error is going to be. It is denoted by $O(\Delta y)$

We can find another way to express U'(y) using the second equation.
\[
  U'(y) = \frac{U(y) + U(y+\Delta y)}{\Delta y} + O(\Delta y)
\]
This is termed as the forward difference approximation
\[
  U'(y) = \frac{U(y) - U(y-\Delta y)}{\Delta y} - O(\Delta y)
\]
This is termed as the backward difference approximation

On subtracting $U(y-\Delta y)$ from $U(y+\Delta y)$
\[
  U(y + \Delta y) - U(y - \Delta y) = 2 U'(y) \frac{\Delta y}{1} + 2U''(y)\frac{\Delta y^3}{3!}
\]

An expression for $U'(y)$ with this equation
\[
  U'(y) = \frac{U(y + \Delta y) - U(y - \Delta y)}{2 \Delta y} - O(\Delta y^2) 
\]
This is termed as the central difference approximation
With these three approximations, we will solve the problem from earlier.
For $U''(y)$,
\[
U''(y) = \frac{U(y+\Delta y) - 2U(y) + U(y-\Delta y)}{\Delta y^2}
\]

\section{For Unidirectional Flow}
Let's look at two cases and solve,
  $\frac{\partial p}{\partial x}$, the pressure gradient is given
    We assume, to be general, that the plates are moving with a speed, $U_h$ and $U_0$ respectively for the top and bottom.

    The equation is,
    \[
      -\frac{\partial p}{\partial x} + \mu \frac{\partial^2 u}{\partial x^2} + \rho g_x = \rho \frac{\partial u}{\partial t}
    \]
    Substituting the equations we found using the Taylor series.
    \[
      -\frac{\partial p}{\partial x} + \mu(\frac{U(y+\Delta y) - 2U(y) + U(y-\Delta y)}{\Delta y^2}) + \rho g_y = \rho(\frac{U(t + \Delta t) - U(t-\Delta t)}{2\Delta t})
    \]
    Finding the partial derivative due to time,
    \[
      U'' = \frac{U(y+\Delta y, t)-2U(y,t) + U(y-\Delta y, t)}{\Delta y^2}
    \]
    \[
      \dot{U} = \frac{U(y, t+ \Delta t) - U(y,t - \Delta t)}{2\Delta t}
    \]

    We finally arrive at the final method for solving this. We break each point of the fluid vertically into discrete locations. Termed as ($U_1, U_2,\dots, U_n$)

    \[
      U_i(t) = \frac{-\partial p}{\partial x} + \mu(\frac{U_{i+1} - 2U_{i} + U_{i-1}}{\Delta y}) + \rho g_y
    \]
    As well as,
    \[
      U_i(t) = \rho(\frac{U_i(t + \Delta t)-U(t-\Delta t)}{2\Delta t})
    \]

    So,
    \[
    U_{i}(t+\Delta t) = 2\Delta t (-\frac{-1}{\rho} \frac{\partial p}{\partial x} + g_y) + \frac{2 \Delta t \mu}{\rho}(\frac{U_{i+1} - 2U_i + U_{i-1}}{\Delta y^2}) + U_{t - \Delta t}
    \]
  These two equations form the basis of what we are trying to code.

  The final equation to be is,
  \[
    U_{new}[i] = \frac{2 \Delta t}{\rho}(-\frac{\partial p}{\partial x} + \rho g_y) + \frac{2 \Delta t \mu}{\rho \Delta y^2}(U_{now}[i+1] - 2U_{now}[i] + U_{now}[i-1] + U_{old}[i])
  \]
  What these terms mean,
  \begin{align*}
    \frac{\mu}{\rho} & & \text{(Kinematic viscosity)}-\nu \\
    \frac{\Delta t\mu}{\rho \Delta y^2} & & \text{(Difffusion number)}-\alpha
  \end{align*}
  How do we put boundary conditions, we use the velocities of the plates.
  \begin{align*}
   &U(0,t) = U_0 = U_1(t)  \\
   &U(h,t) = U_h = U_{N+1}(t)
  \end{align*}
\begin{lstlisting}{octave}
% Defining constants
mu = 0.6;
rho = 100;
g = 0.4;
h = 1;
N = 32;
deltaY = h/N;

% Terms in the equation
nu = mu/rho; % Kinematic Viscosity
alpha = 0.51; % Diffusion Number
deltaT = alpha*rho*(deltaY^2)/mu; % Evaluating time step from alpha
dpdx = -2; % Pressure Gradient

% Velocities of the plates
U0 = 0;
Uh = 0;

% Time for code to run
T = 5;
U = zeros(floor(T/deltaT),N+1);
% Initial Conditions
y = 0:deltaY:h;

C = (dpdx + rho*g)/2*mu;
Uexact = C.*(-y.^2 + h.*y);
% Equation
for k = 2:floor(T/deltaT)
  U(k,1) = U0;
  U(k,N+1) = Uh;
  for i = 2:N
    U(k,i) = deltaT/rho*(-dpdx    + rho*g) + alpha*(U(k-1,i+1)-2*U(k-1,i) + U(k-1,i-1)) + U(k-1,i);
  end
end

% Plotting
% Y axis - y, x axis - U(y), For a given time t
plot(U(end,:),y,'blue');
\end{lstlisting}
\section{Implicit Fluid Dynamics}
\begin{align*}
&\rho \frac{du}{dt} = - \frac{\partial p}{\partial t} + \mu \frac{\partial^2 u}{\partial y^2} + \rho g_x \\
&\rho \frac{U(y, t + \Delta t) - U(y,t)}{\Delta t} =\frac{1}{\rho} \frac{\partial P}{\partial x}(t + \Delta t) + \frac{\mu}{\rho}U_{i+1}(t+\Delta t) - \frac{-2U_i(t+\Delta t) + U_{i-1}(t+\Delta t) + g_x(t+\Delta t)}{\Delta y^2} \\
&\frac{\rho U_i(t+\Delta t) - U_i(t))}{\Delta t} = \frac{-\partial p}{\partial x} + \rho g_x + \mu \frac{U_{i+1}(t + \Delta t) - 2 u_i (t+\Delta t) + U_{i-1}(t+\Delta t)}{\Delta y^2}\\
&\text{We finally get the expression,} \\
&U_{i}(t+\Delta t)(\frac{\rho}{\Delta t} + \frac{2\mu}{(\Delta y)^2}) + (\frac{-\mu}{\Delta y^2})U_{i+1}(t+\Delta t) + (\frac{-\mu}{\Delta y^2}U_{i-1}(t+\Delta t) = \frac{S}{\Delta t}U_i(t) - \frac{\partial P}{\partial x} + \rho g_x\\
\end{align*}

The final equations we get are,
\begin{align*}
	&a U _{i-1}(t + \Delta t) + b U_i(t+ \Delta t) + a U_{i+1}(t + \Delta t) = U_{i}(t) - \frac{\Delta t}{\rho}(\frac{\partial p}{\partial x}(t + \Delta t))_i + \Delta t g_x \\
\end{align*}
Where, 
\[ 
	a = \frac{-\mu \Delta t}{\rho \Delta y^2}
\]
\[
	b = 1 + \frac{2\mu \Delta t}{\rho \Delta y^2}
\]

\begin{align*}
	\begin{bmatrix}
		b  & a & 0 & 0 & \cdots \\
		a & b & a & 0 & \cdots \\
		0 & a & b & a & \cdots \\  
		\cdots & 0 & a & b & 0\\
		\cdots & 0 & 0 & a & b \\
	\end{bmatrix}
	\begin{bmatrix}
		u_1 \\
		u_2 \\
		\vdots \\
		u_n \\
		u_{n+1}
	\end{bmatrix}
	= 
	\begin{bmatrix}
		u2 - \frac{\Delta t}{\rho}\frac{\partial p}{\partial x} + \Delta t g_x \\
		\vdots \\
		\vdots \\
		un - \cdots \\
	\end{bmatrix}
\end{align*}
The boundary conditions tell us that the top and bottom most nodes are 0, we can then cancel out the the first and last row and column.
Onto coding it all in Octave.
\begin{lstlisting}
%  Constants
mu = 0.6;
N = 32;
h = 1;
rho = 0.5;
dpdx = -2;
gx = 0.4;
alpha1 = 0.1;
alpha2 = 0.5;


% Time and space steps
deltaY = h/N;
deltaT = -a*rho*deltaY^2/mu;
T = 20000*deltaT;

a = -0.4;
b = 1 + 2*mu*deltaT/rho*(deltaY^2);

% Initial Velocities
uInitial = zeros(N-1,1);
uInitial(1) = alpha1;
uInitial(end) = alpha2;

U = uInitial;
y = deltaY:deltaY:h-deltaY;

% Calculating the velocities
function A = constantCoefficientMatrix(a,b,N)
	A = b*eye(N-1,N-1);
	aMatrix1 = zeros(N-1,N-1);
	aMatrix1(1:end-1,2:end) = a*eye(N-2,N-2);
	aMatrix2 = zeros(N-1,N-1);
	aMatrix2(2:end,1:end-1) = a*eye(N-2,N-2);
	A = A + aMatrix1 + aMatrix2;
end

c = deltaT/rho*dpdx + deltaT*gx;
function f = forceVector(c,a,alpha1,alpha2,Ut,N)
	f = c*ones(N-1,1);
	f(1) -= a*alpha1;
	f(end) -= a*alpha2;
	f += Ut;
end
for t = 0:deltaT:T
	f = forceVector(c,a,alpha1,alpha2,U,N);
	A = constantCoefficientMatrix(a,b,N);
	U = A\f;
	plot(U,y);
	pause(0.2);
end
\end{lstlisting}
\chapter{Introduction To Robotics}
This is a very brief introduction to robotics, which will connect back to Mechanics 1. 
There are two main components to this section of the course.
\begin{enumerate}
	\item Control
	\item \textit{Two-arm} robotic hand
\end{enumerate}

The first thing to know in such a robotics course, is that the word robot is an acronym. Random Optical Binary Oscillating Technology. Robots are fundamentally about trying to replicate human motion. They are a replacement for human effort. 
 
Now we try to understand with a problem.

Refer to Problem \textit{19.3.26},

We usually have a equation for force, that governs the motion and force of the body.

\[ 
	m\ddot{x} + c\dot{x} + kx = F
\]

Recall, the pendulum equation

\[ 
	\frac{d^2 \theta}{dt^2} - \frac{g}{l} \cos{\theta} = M
\]

Where M is an external moment supplied at the point from which the pendulumm is suspended.

We can calculate the force applied on this pendulum by tracking its position.

In robotics, it's the other way around. What we can control is the force acting on the pendulum, to get the desired motion.

All of robotics can be summarized as this - controlling f and m.

We control position, velocity and acceleration, given that they are proportional to force.
\begin{align*}
	f &\propto x \\
	  &\propto \dot{x} \\
	  &\propto \ddot{x} \\
\end{align*}

We have some output $x(t)$, and some input $\overline{x}(t) = K_p x(t) + K_d \dot{x}(t) + K_I \int x(t) dt$ 

The flow of it all is $\text{transfer  function} \rightarrow \text{PID Control}\rightarrow$  and doing 3 simple robots:

\begin{enumerate}
	\item Balancing a stick
	\item Two-arm elbow manipulator (double bar)
	\item Two-arm cartesian robot
\end{enumerate}
The majority of this course is focused on deriving the equations of motions for the above 3 robots. The next step is PID Control (Proportional Integral Differential) Control. We control these robots by controlling the force and moments.

Looking at PID
The force should be such that, it is a linear combination of $\theta$, $\int \theta dt$ and $\dot{\theta}$. The last and final step is to derive the transfer function, since this is an introduction.

The transfer function is done by plugging the independent variable as a function of s like so:
\begin{align*}
	x &= Ae^{st} \\
	\dot{x} &= Ase^{st} \\
	\ddot{x} &= As^2e^{st} \\
\end{align*}

So for the PID relation of F to $\theta$,

\begin{align*}
	\theta(t) \rightarrow \theta(s) \\
	\dot{\theta(t)} \rightarrow s\theta(s) \\
	\int \theta (t) \rightarrow \frac{\theta(s)}{s} \\
\end{align*}

Now the original equation, 
\[m \ddot{x} + c\dot{x} + k x = F(t) = K_p x + K_I \int x dt + K_d \dot{x} \]
Becomes,
\[ms^2  + cs + k = K_p + \frac{K_I}{s} + K_Ds\]

But there's something missing. Robots do not necessarily behave the way we want them to, the assumption that they do has lead to the transfer function losing the inputs and outputs. We assume that the robot has some desired displacement $x_i$ or action it wants to perform, and $x$ is the real displacement that occurs.
We get,
\[ x_i = B(s)e^{st}\]
\[ x = A(s)e^{st}\]

\[m \ddot{x} + c\dot{x} + k x = F(t) = K_p x_i + K_I \int x_i dt + K_d \dot{x_i} \]
\[Ams^2  + Acs + Ak = B K_p + \frac{BK_I}{s} + BK_Ds\]
\[A(ms^2  + cs + k) = B(K_p + \frac{K_I}{s} + K_Ds)\]

If $x_i$ and $x$ are the input and output, then the transfer function has to be $\frac{A}{B}$

Therefore,
\[\frac{K_p + \frac{K_I}{s} + K_d s}{ms^2 + cs + k}\]
\begin{note}
	FIND WHAT THIS KIND OF FRACTION IS CALLED
\end{note}
This was one way to derive the transfer function. The engineering way is,

\[TF = \frac{A(s)}{F(s)} = \frac{K_p}{ms^2 + cs + k + K_p}\]

Engineers put in a "P-control" instead of a PID control, which is easier to solve for.

The way this is obtained is by putting an error variable, $e$, the error is $x - x_i$. This derivation is out of the scope of the current course, but the basic idea is to use the error.

If we used PI-Control, then, finding through the error,

\[TF = \frac{K_p + \frac{K_I}{s}}{ms^2 + cs + (K + K_P + \frac{K_I}{s})}\]
Simplifying further,
\[TF = \frac{K_p s + K_I}{ms^3 + cs^2 + (K + Kp)s + K_I}\]

For a PID-control, then,
\[TF = \frac{K_p s + K_i + K_d s^2}{ms^3 + (c + K_d)s^2 + (K + K_p)s + K_I}\]
\section{MATLAB/Octave Notes}
\begin{lstlisting}
Kp = 1;
Ki = 1;
Kd = 1;

s = tf('s');
C = Kp + Ki/s + Kd*s
% Alternatively, we may use MATLAB's pid object to generate an equivalent continuous-time controller as follows: 
C = pid(Kp,Ki,Kd)
% Let's convert the pid object to a transfer function to verify that it yields the same result as above:
tf(C)
\end{lstlisting}
Suppose we have a simple mass-spring-damper system.

The governing equation of this system is
\[m \ddot{x} + c\dot{x} + k x = F(t) = K_p x + K_I \int x dt + K_d \dot{x} \]
Taking the Laplace transform of the governing equation, we get
\[ms^2  + cs + k = F(s)\]

The goal of this problem is to show how each of the terms, $K_p$, $K_i$, and $K_d$, contributes to obtaining the common goals of:
\begin{enumerate}
	\item Fast rise time
	\item Minimal overshoot
	\item Zero steady-state error
 
\end{enumerate}

Substituting these values into the above transfer function

\[\frac{X(s)}{F(s)} = \frac{1}{s^2 + 10s + 20} \]
\subsection{Open-Loop Step Response}
\begin{lstlisting}
s = tf('s');
P = 1/(s^2 + 10*s + 20);
step(P)
\end{lstlisting}
The DC gain of the plant transfer function is 1/20, so 0.05 is the final value of the output to a unit step input. This corresponds to a steady-state error of 0.95, which is quite large. Furthermore, the rise time is about one second, and the settling time is about 1.5 seconds. Let's design a controller that will reduce the rise time, reduce the settling time, and eliminate the steady-state error.
\subsection{Proportional Control}
From the table shown above, we see that the proportional controller ($K_p$) reduces the rise time, increases the overshoot, and reduces the steady-state error.

The closed-loop transfer function of our unity-feedback system with a proportional controller is the following, where $X(s)$ is our output (equals $Y(s)$) and our reference $R(s)$ is the input:

\[T(s) = \frac{X(s)}{R(s)} = \frac{K_p}{s^2 + 10s + (20 + K_p)} \]


Let the proportional gain ($K_p$) equal 300 and change the m-file to the following:
\begin{lstlisting}
Kp = 300;
C = pid(Kp)
T = feedback(C*P,1)

t = 0:0.01:2;
step(T,t)

\end{lstlisting}
Now, let's take a look at PD control. From the table shown above, we see that the addition of derivative control ($K_d$) tends to reduce both the overshoot and the settling time. The closed-loop transfer function of the given system with a PD controller is:

\[T(s) = \frac{X(s)}{R(s)} = \frac{K_d s + K_p}{s^2 + (10 + K_d) s + (20 + K_p)} \]

Let $K_p$ equal 300 as before and let $K_d$ equal 10. Enter the following commands into an m-file and run it in the MATLAB command window.
\begin{lstlisting}
Kp = 300;
Kd = 10;
C = pid(Kp,0,Kd)
T = feedback(C*P,1)

t = 0:0.01:2;
step(T,t)
\end{lstlisting}
\subsection{PI Control}
Before proceeding to PID control, let's investigate PI control. From the table, we see that the addition of integral control ($K_i$) tends to decrease the rise time, increase both the overshoot and the settling time, and reduces the steady-state error. For the given system, the closed-loop transfer function with a PI controller is:

\[ T(s) = \frac{X(s)}{R(s)} = \frac{K_p s + K_i}{s^3 + 10 s^2 + (20 + K_p )s + K_i} \]

Let's reduce $K_p$ to 30, and let $K_i$ equal 70. Create a new m-file and enter the following commands.
\begin{lstlisting}
Kp = 30;
Ki = 70;
C = pid(Kp,Ki)
T = feedback(C*P,1)

t = 0:0.01:2;
step(T,t)
\end{lstlisting}
\subsection{PID Control}
ow, let's examine PID control. The closed-loop transfer function of the given system with a PID controller is:

\[ T(s) = \frac{X(s)}{R(s)} = \frac{K_d s^2 + K_p s + K_i}{s^3 + (10 + K_d)s^2 + (20 + K_p)s + K_i } \]

After several iterations of tuning, the gains $K_p$ = 350, $K_i$ = 300, and $K_d$ = 50 provided the desired response. To confirm, enter the following commands to an m-file and run it in the command window. You should obtain the following step response.
\begin{lstlisting}
Kp = 350;
Ki = 300;
Kd = 50;
C = pid(Kp,Ki,Kd)
T = feedback(C*P,1);

t = 0:0.01:2;
step(T,t)
\end{lstlisting}
\section{Inverted Pendulum System Modelling}
Below are the free-body diagrams of the two elements of the inverted pendulum system.

Summing the forces in the free-body diagram of the cart in the horizontal direction, you get the following equation of motion.

$$ M\ddot{x}+b\dot{x}+N = F $$

Note that you can also sum the forces in the vertical direction for the cart, but no useful information would be gained.

Summing the forces in the free-body diagram of the pendulum in the horizontal direction, you get the following expression for the reaction force $N$.

$$ N= m\ddot{x}+ml\ddot{\theta}\cos\theta-ml\dot{\theta}^2\sin\theta $$

If you substitute this equation into the first equation, you get one of the two governing equations for this system.

$$(M+m)\ddot{x}+b\dot{x}+ml\ddot{\theta}\cos\theta-ml\dot{\theta}^2\sin\theta=F $$

To get the second equation of motion for this system, sum the forces perpendicular to the pendulum. Solving the system along this axis greatly simplifies the mathematics. You should get the following equation.

$$P\sin\theta+N\cos\theta-mg\sin\theta=ml\ddot{\theta}+m\ddot{x}\cos\theta$$

To get rid of the $P$ and $N$ terms in the equation above, sum the moments about the centroid of the pendulum to get the following equation.

$$-Pl\sin\theta-Nl\cos\theta=I\ddot{\theta}$$

Combining these last two expressions, you get the second governing equation.

$$(I+ml^2)\ddot{\theta}+mgl\sin\theta=-ml\ddot{x}\cos\theta $$

Since the analysis and control design techniques we will be employing in this example apply only to linear systems, this set of equations needs to be linearized. Specifically, we will linearize the equations about the vertically upward equillibrium position, $\theta$ = $\pi$, and will assume that the system stays within a small neighborhood of this equillbrium. This assumption should be reasonably valid since under control we desire that the pendulum not deviate more than 20 degrees from the vertically upward position. Let $\phi$ represent the deviation of the pedulum's position from equilibrium, that is, $\theta$ = $\pi$ + $\phi$. Again presuming a small deviation ($\phi$) from equilibrium, we can use the following small angle approximations of the nonlinear functions in our system equations:

$$ \cos \theta = \cos(\pi + \phi) \approx -1 $$

$$ \sin \theta = \sin(\pi + \phi) \approx -\phi $$

$$ \dot{\theta}^2 = \dot{\phi}^2 \approx 0 $$

After substiting the above approximations into our nonlinear governing equations, we arrive at the two linearized equations of motion. Note $u$ has been substituted for the input $F$.

$$ (I+ml^2)\ddot{\phi}-mgl\phi=ml\ddot{x} $$

$$ (M+m)\ddot{x}+b\dot{x}-ml\ddot{\phi}=u $$

\subsection{Transfer Function}
To obtain the transfer functions of the linearized system equations, we must first take the Laplace transform of the system equations assuming zero initial conditions. The resulting Laplace transforms are shown below.

$$(I+ml^2)\Phi(s)s^2-mgl\Phi(s)=mlX(s)s^2$$

$$(M+m)X(s)s^2+bX(s)s-ml\Phi(s)s^2=U(s)$$

Recall that a transfer function represents the relationship between a single input and a single output at a time. To find our first transfer function for the output $\Phi(s)$ and an input of $U(s)$ we need to eliminate $X(s)$ from the above equations. Solve the first equation for $X(s)$.

$$ X(s)=\left[{\frac{I+ml^2}{ml}-\frac{g}{s^2}}\right]\Phi(s) $$

Then substitute the above into the second equation.

$$(M+m)\left[\frac{I+ml^2}{ml}-\frac{g}{s^2}\right]\Phi(s)s^2+b\left[\frac{I+ml^2}{ml}-\frac{g}{s^2}\right]\Phi(s)s-ml\Phi(s)s^2=U(s)$$

Rearranging, the transfer function is then the following

$$\frac{\Phi(s)}{U(s)}=\frac{\frac{ml}{q}s^2}{s^4+\frac{b(I+ml^2)}{q}s^3-\frac{(M+m)mgl}{q}s^2-\frac{bmgl}{q}s}$$

where,

$$q=[(M+m)(I+ml^2)-(ml)^2]$$

From the transfer function above it can be seen that there is both a pole and a zero at the origin. These can be canceled and the transfer function becomes the following.

$$P_{pend}(s) = \frac{\Phi(s)}{U(s)}=\frac{\frac{ml}{q}s}{s^3+\frac{b(I+ml^2)}{q}s^2-\frac{(M+m)mgl}{q}s-\frac{bmgl}{q}} \qquad [ \frac{rad}{N}]$$

Second, the transfer function with the cart position $X(s)$ as the output can be derived in a similar manner to arrive at the following.

$$P_{cart}(s) = \frac{X(s)}{U(s)} = \frac{ \frac{ (I+ml^2)s^2 - gml } {q} }{s^4+\frac{b(I+ml^2)}{q}s^3-\frac{(M+m)mgl}{q}s^2-\frac{bmgl}{q}s} \qquad [ \frac{m}{N}] $$
\subsection{State Space}
The linearized equations of motion from above can also be represented in state-space form if they are rearranged into a series of first order differential equations. Since the equations are linear, they can then be put into the standard matrix form shown below.

$$ \left[{\begin{array}{c} \dot{x}\\ \ddot{x}\\ \dot{\phi}\\ \ddot{\phi} \end{array}}\right] = \left[{\begin{array}{cccc} 0&1&0&0\\ 0&\frac{-(I+ml^2)b}{I(M+m)+Mml^2}&\frac{m^2gl^2}{I(M+m)+Mml^2}&0\\ 0&0&0&1\\ 0&\frac{-mlb}{I(M+m)+Mml^2}&\frac{mgl(M+m)}{I(M+m)+Mml^2}&0 \end{array}}\right] \left[{\begin{array}{c} x\\ \dot{x}\\ \phi\\ \dot{\phi} \end{array}}\right]+ \left[{\begin{array}{c}0\\ \frac{I+ml^2}{I(M+m)+Mml^2}\\ 0 \\ \frac{ml}{I(M+m)+Mml^2} \end{array}}\right]u$$

$${\bf y} = \left[{\begin{array}{cccc} 1&0&0&0\\0&0&1&0 \end{array}}\right] \left[{\begin{array}{c} x\\ \dot{x}\\ \phi\\ \dot{\phi} \end{array}}\right]+ \left[{\begin{array}{c} 0\\0 \end{array}}\right]u$$

The $C$ matrix has 2 rows because both the cart's position and the pendulum's position are part of the output. Specifically, the cart's position is the first element of the output $\mathbf{y}$ and the pendulum's deviation from its equilibrium position is the second element of $\mathbf{y}$.


\subsection{In MATLAB}
\subsubsection{Transfer Function}
\begin{lstlisting}
M = 0.5;
m = 0.2;
b = 0.1;
I = 0.006;
g = 9.8;
l = 0.3;
q = (M+m)*(I+m*l^2)-(m*l)^2;
s = tf('s');

P_cart = (((I+m*l^2)/q)*s^2 - (m*g*l/q))/(s^4 + (b*(I + m*l^2))*s^3/q - ((M + m)*m*g*l)*s^2/q - b*m*g*l*s/q);

P_pend = (m*l*s/q)/(s^3 + (b*(I + m*l^2))*s^2/q - ((M + m)*m*g*l)*s/q - b*m*g*l/q);

sys_tf = [P_cart ; P_pend];

inputs = {'u'};
outputs = {'x'; 'phi'};

set(sys_tf,'InputName',inputs)
set(sys_tf,'OutputName',outputs)

sys_tf
\end{lstlisting}
\subsubsection{State Space}
\begin{lstlisting}
M = .5;
m = 0.2;
b = 0.1;
I = 0.006;
g = 9.8;
l = 0.3;

p = I*(M+m)+M*m*l^2; %denominator for the A and B matrices

A = [0      1              0           0;
     0 -(I+m*l^2)*b/p  (m^2*g*l^2)/p   0;
     0      0              0           1;
     0 -(m*l*b)/p       m*g*l*(M+m)/p  0];
B = [     0;
     (I+m*l^2)/p;
          0;
        m*l/p];
C = [1 0 0 0;
     0 0 1 0];
D = [0;
     0];

states = {'x' 'x_dot' 'phi' 'phi_dot'};
inputs = {'u'};
outputs = {'x'; 'phi'};

sys_ss = ss(A,B,C,D,'statename',states,'inputname',inputs,'outputname',outputs)
\end{lstlisting}
The above state-space model can also be converted into transfer function form employing the tf command as shown below. Conversely, the transfer function model can be converted into state-space form using the ss command. 
\begin{lstlisting}
sys_tf = tf(sys_ss)
\end{lstlisting}
\section{The Flow Of Robotics}
The problems have this flow:
\begin{itemize}
	\item Equations Of Motion
	\item Laplace Transform
	\item Transfer Function
	\item Roots, poles, plots, state space analysis
\end{itemize}
We now try to find 
\subsection{Equations Of Motion}
\[\Sigma \vec{f} = m \vec{a}_{cm}\] 

\[ \Sigma \vec{M}_{/O} = I_{ZZ}^O \vec{\alpha}\]
For a given problem involving a roller support fixing a pendulum.

\begin{note}
The Laplace Transform is only possible if the equation is linear.
\end{note}
\begin{align*}
	f_{in} \hat{i} + (R_y - mg)\hat{j} &= m \frac{d^2}{dt} (\frac{l}{2} \cos(\theta) \hat{i}  + \frac{l}{2} \sin{\theta} \hat{j}) \\
					   &= \frac{ml}{2} \frac{d}{dt}(-\dot{\theta} \sin{\theta} \hat{i} + \dot{\theta} \cos{\theta})\hat{j} \\
					   &= \frac{ml}{2}(-\dot{\theta} \sin{\theta}-\dot{\theta} \cos{theta})\hat{i} + (\ddot{\theta} \cos{theta} - \theta^2 \sin{theta}) \\
\end{align*}
This is a non-linear equation, we cannot apply the Taylor series onto it.

What about if $\theta \rightarrow 0$,

\begin{align*}
	f_{in} = \frac{ml}{2}(\ddot{\theta} \theta + \dot{\theta}^2) \\
	R_y - mg = \frac{ml}{2}(\ddot{\theta} - \dot{(\theta)}\theta)
\end{align*} 
Here's the next problem, $(\dot{\theta)}^2$ is still part of each equation, here comes an engineering guess. We're making the assumption that $\dot{\theta}^2 = 0$

We get,
\begin{align*}
	f_{in} = \frac{ml}{2}(\ddot{\theta} \theta) \\
	R_y - mg = \frac{ml}{2}(\ddot{\theta})  \\ 
\end{align*}
The problem still remains with the term $\ddot{\theta} \theta$,

We look at the 2nd equation,

\begin{align*}
	\Sigma M_{/O} &= I_{ZZ}^O \ddot{\theta} \\
	\implies \vec{r}_{CM} \times (-mg)\hat{j} &= I_{ZZ}^{O} \ddot{\theta} \\
	\implies \frac{l}{2} (\cos{\theta}\hat{i} + \sin{\theta} \hat{j} \times (-mg)\hat{j} &= I_{ZZ}^O \ddot{\theta}) \\
	\implies -mg \frac{l}{2} \cos{\theta} &= \frac{ml^2}{3} \ddot{\theta}
\end{align*}
Now we take the limit $\theta \rightarrow 0$
\[-mg \frac{l}{2} = \frac{ml^2}{3} \ddot{\theta}\]
This is easier to deal with now.
But, we've lost something. We do not have any equations that talk about the force input or the path of the object.

The remedy is the origin, assume that the origin is moving, and we keep another system that tracks the movement of this origin.

\begin{align*}
	\vec{r}_{CM} &= \frac{l}{2}(\cos{\theta}\hat{i} + \sin{\theta}\hat{j} + x_o \hat{i}) \\
	\vec{v}_{CM} &= \frac{l}{2}(\dot{\theta}\sin{\theta}\hat{i} + \dot{\theta} \cos{\theta} \hat{j} + \dot{x}_{o} \hat{i}) \\
	\vec{a}_{CM} &= \frac{l}{2} (-\ddot{\theta} \cos{\theta} - \dot{\theta}^2 \sin{\theta} \hat{i}  + (\ddot{\theta} \cos{\theta} - \theta^2 \sin{\theta} \hat{j})) + \ddot{x_o} \hat{i}
\end{align*}

We can convert this into a set of matrices,
\[
	M \begin{bmatrix}
		\ddot{x_o} \\
		\ddot{\theta} \\
	\end{bmatrix}
	+ 
	K \begin{bmatrix}
		x_o \\
		\theta \\
	\end{bmatrix}
	= 
	\begin{bmatrix}
	&	\\
	&	\\
	\end{bmatrix}
\]

\end{document}


\documentclass{report}
%%%%%%%%%%%%%%%%%%%%%%%%%%%%%%%%%%%%%%%%%%%%%%%%%%%%%%%%%%%%%%%%%%%%%%%%%%%%%%%
%                                Basic Packages                               %
%%%%%%%%%%%%%%%%%%%%%%%%%%%%%%%%%%%%%%%%%%%%%%%%%%%%%%%%%%%%%%%%%%%%%%%%%%%%%%%

% Gives us multiple colors.
\usepackage[dvipsnames,pdftex]{xcolor}
% Lets us style link colors.
\usepackage{hyperref}
% Lets us import images and graphics.
\usepackage{graphicx}
% Lets us use figures in floating environments.
\usepackage{float}
% Lets us create multiple columns.
\usepackage{multicol}
% Gives us better math syntax.
\usepackage{amsmath,amsfonts,mathtools,amsthm,amssymb}
% Lets us strikethrough text.
\usepackage{cancel}
% Lets us edit the caption of a figure.
\usepackage{caption}
% Lets us import pdf directly in our tex code.
\usepackage{pdfpages}
% Lets us do algorithm stuff.
\usepackage[ruled,vlined,linesnumbered]{algorithm2e}
% Use a smiley face for our qed symbol.
\usepackage{tikzsymbols}
\renewcommand\qedsymbol{$\Laughey$}

\def\class{article}


%%%%%%%%%%%%%%%%%%%%%%%%%%%%%%%%%%%%%%%%%%%%%%%%%%%%%%%%%%%%%%%%%%%%%%%%%%%%%%%
%                                Basic Settings                               %
%%%%%%%%%%%%%%%%%%%%%%%%%%%%%%%%%%%%%%%%%%%%%%%%%%%%%%%%%%%%%%%%%%%%%%%%%%%%%%%

%%%%%%%%%%%%%
%  Symbols  %
%%%%%%%%%%%%%

\let\implies\Rightarrow
\let\impliedby\Leftarrow
\let\iff\Leftrightarrow
\let\epsilon\varepsilon

%%%%%%%%%%%%
%  Tables  %
%%%%%%%%%%%%

\setlength{\tabcolsep}{5pt}
\renewcommand\arraystretch{1.5}

%%%%%%%%%%%%%%
%  SI Unitx  %
%%%%%%%%%%%%%%

\usepackage{siunitx}
\sisetup{locale = FR}

%%%%%%%%%%
%  TikZ  %
%%%%%%%%%%

\usepackage[framemethod=TikZ]{mdframed}
\usepackage{tikz}
\usepackage{tikz-cd}
\usepackage{tikzsymbols}

\usetikzlibrary{intersections, angles, quotes, calc, positioning}
\usetikzlibrary{arrows.meta}

\tikzset{
  force/.style={thick, {Circle[length=2pt]}-stealth, shorten <=-1pt}
}

%%%%%%%%%%%%%%%
%  PGF Plots  %
%%%%%%%%%%%%%%%

\usepackage{pgfplots}
\pgfplotsset{compat=1.13}

%%%%%%%%%%%%%%%%%%%%%%%
%  Center Title Page  %
%%%%%%%%%%%%%%%%%%%%%%%

\usepackage{titling}
\renewcommand\maketitlehooka{\null\mbox{}\vfill}
\renewcommand\maketitlehookd{\vfill\null}

%%%%%%%%%%%%%%%%%%%%%%%%%%%%%%%%%%%%%%%%%%%%%%%%%%%%%%%
%  Create a grey background in the middle of the PDF  %
%%%%%%%%%%%%%%%%%%%%%%%%%%%%%%%%%%%%%%%%%%%%%%%%%%%%%%%

\usepackage{eso-pic}
\newcommand\definegraybackground{
  \definecolor{reallylightgray}{HTML}{FAFAFA}
  \AddToShipoutPicture{
    \ifthenelse{\isodd{\thepage}}{
      \AtPageLowerLeft{
        \put(\LenToUnit{\dimexpr\paperwidth-222pt},0){
          \color{reallylightgray}\rule{222pt}{297mm}
        }
      }
    }
    {
      \AtPageLowerLeft{
        \color{reallylightgray}\rule{222pt}{297mm}
      }
    }
  }
}

%%%%%%%%%%%%%%%%%%%%%%%%
%  Modify Links Color  %
%%%%%%%%%%%%%%%%%%%%%%%%

\hypersetup{
  % Enable highlighting links.
  colorlinks,
  % Change the color of links to blue.
  linkcolor=blue,
  % Change the color of citations to black.
  citecolor={black},
  % Change the color of url's to blue with some black.
  urlcolor={blue!80!black}
}

%%%%%%%%%%%%%%%%%%
% Fix WrapFigure %
%%%%%%%%%%%%%%%%%%

\newcommand{\wrapfill}{\par\ifnum\value{WF@wrappedlines}>0
    \parskip=0pt
    \addtocounter{WF@wrappedlines}{-1}%
    \null\vspace{\arabic{WF@wrappedlines}\baselineskip}%
    \WFclear
\fi}

%%%%%%%%%%%%%%%%%
% Multi Columns %
%%%%%%%%%%%%%%%%%

\let\multicolmulticols\multicols
\let\endmulticolmulticols\endmulticols

\RenewDocumentEnvironment{multicols}{mO{}}
{%
  \ifnum#1=1
    #2%
  \else % More than 1 column
    \multicolmulticols{#1}[#2]
  \fi
}
{%
  \ifnum#1=1
\else % More than 1 column
  \endmulticolmulticols
\fi
}

\newlength{\thickarrayrulewidth}
\setlength{\thickarrayrulewidth}{5\arrayrulewidth}


%%%%%%%%%%%%%%%%%%%%%%%%%%%%%%%%%%%%%%%%%%%%%%%%%%%%%%%%%%%%%%%%%%%%%%%%%%%%%%%
%                           School Specific Commands                          %
%%%%%%%%%%%%%%%%%%%%%%%%%%%%%%%%%%%%%%%%%%%%%%%%%%%%%%%%%%%%%%%%%%%%%%%%%%%%%%%

%%%%%%%%%%%%%%%%%%%%%%%%%%%
%  Initiate New Counters  %
%%%%%%%%%%%%%%%%%%%%%%%%%%%

\newcounter{lecturecounter}

%%%%%%%%%%%%%%%%%%%%%%%%%%
%  Helpful New Commands  %
%%%%%%%%%%%%%%%%%%%%%%%%%%

\makeatletter

\newcommand\resetcounters{
  % Reset the counters for subsection, subsubsection and the definition
  % all the custom environments.
  \setcounter{subsection}{0}
  \setcounter{subsubsection}{0}
  \setcounter{paragraph}{0}
  \setcounter{subparagraph}{0}
  \setcounter{theorem}{0}
  \setcounter{claim}{0}
  \setcounter{corollary}{0}
  \setcounter{lemma}{0}
  \setcounter{exercise}{0}

  \@ifclasswith\class{nocolor}{
    \setcounter{definition}{0}
  }{}
}

%%%%%%%%%%%%%%%%%%%%%
%  Lecture Command  %
%%%%%%%%%%%%%%%%%%%%%

\usepackage{xifthen}

% EXAMPLE:
% 1. \lesson{Oct 17 2022 Mon (08:46:48)}{Lecture Title}
% 2. \lesson[4]{Oct 17 2022 Mon (08:46:48)}{Lecture Title}
% 3. \lesson{Oct 17 2022 Mon (08:46:48)}{}
% 4. \lesson[4]{Oct 17 2022 Mon (08:46:48)}{}
% Parameters:
% 1. (Optional) Lesson number.
% 2. Time and date of lecture.
% 3. Lecture Title.
\def\@lesson{}
\newcommand\lesson[3][\arabic{lecturecounter}]{
  % Add 1 to the lecture counter.
  \addtocounter{lecturecounter}{1}

  % Set the section number to the lecture counter.
  \setcounter{section}{#1}
  \renewcommand\thesubsection{#1.\arabic{subsection}}

  % Reset the counters.
  \resetcounters

  % Check if user passed the lecture title or not.
  \ifthenelse{\isempty{#3}}{
    \def\@lesson{Lecture \arabic{lecturecounter}}
  }{
    \def\@lesson{Lecture \arabic{lecturecounter}: #3}
  }

  % Display the information like the following:
  %                                                  Oct 17 2022 Mon (08:49:10)
  % ---------------------------------------------------------------------------
  % Lecture 1: Lecture Title
  \hfill\small{#2}
  \hrule
  \vspace*{-0.3cm}
  \section*{\@lesson}
  \addcontentsline{toc}{section}{\@lesson}
}

%%%%%%%%%%%%%%%%%%%%
%  Import Figures  %
%%%%%%%%%%%%%%%%%%%%

\usepackage{import}
\pdfminorversion=7

% EXAMPLE:
% 1. \incfig{limit-graph}
% 2. \incfig[0.4]{limit-graph}
% Parameters:
% 1. The figure name. It should be located in figures/NAME.tex_pdf.
% 2. (Optional) The width of the figure. Example: 0.5, 0.35.
\newcommand\incfig[2][1]{%
  \def\svgwidth{#1\columnwidth}
  \import{./figures/}{#2.pdf_tex}
}

\begingroup\expandafter\expandafter\expandafter\endgroup
\expandafter\ifx\csname pdfsuppresswarningpagegroup\endcsname\relax
\else
  \pdfsuppresswarningpagegroup=1\relax
\fi

%%%%%%%%%%%%%%%%%
% Fancy Headers %
%%%%%%%%%%%%%%%%%

\usepackage{fancyhdr}

% Force a new page.

\newcommand\forcenewpage{\clearpage\mbox{~}\clearpage\newpage}
\newcommand\createintro{
  \pagestyle{fancy}
  \fancyhead{}
  \fancyhead[C]{23PHY114}
  \fancyfoot[L]{Adithya Nair}
  \fancyfoot[R]{AID23002}
  % Create a new page.
}
  \newpage
\makeatother

%%%%%%%%%%%%%%%%%%%%%%%%%%%%%%%%%%%%%%%%%%%%%%%%%%%%%%%%%%%%%%%%%%%%%%%%%%%%%%%
%                               Custom Commands                               %
%%%%%%%%%%%%%%%%%%%%%%%%%%%%%%%%%%%%%%%%%%%%%%%%%%%%%%%%%%%%%%%%%%%%%%%%%%%%%%%

%%%%%%%%%%%%
%  Circle  %
%%%%%%%%%%%%

\newcommand*\circled[1]{\tikz[baseline=(char.base)]{
  \node[shape=circle,draw,inner sep=1pt] (char) {#1};}
}

%%%%%%%%%%%%%%%%%%%
%  Todo Commands  %
%%%%%%%%%%%%%%%%%%%

\usepackage{xargs}
\usepackage[colorinlistoftodos]{todonotes}

\makeatletter

\@ifclasswith\class{working}{
  \newcommandx\unsure[2][1=]{\todo[linecolor=red,backgroundcolor=red!25,bordercolor=red,#1]{#2}}
  \newcommandx\change[2][1=]{\todo[linecolor=blue,backgroundcolor=blue!25,bordercolor=blue,#1]{#2}}
  \newcommandx\info[2][1=]{\todo[linecolor=OliveGreen,backgroundcolor=OliveGreen!25,bordercolor=OliveGreen,#1]{#2}}
  \newcommandx\improvement[2][1=]{\todo[linecolor=Plum,backgroundcolor=Plum!25,bordercolor=Plum,#1]{#2}}

  \newcommand\listnotes{
    \newpage
    \listoftodos[Notes]
  }
}{
  \newcommandx\unsure[2][1=]{}
  \newcommandx\change[2][1=]{}
  \newcommandx\info[2][1=]{}
  \newcommandx\improvement[2][1=]{}

  \newcommand\listnotes{}
}

\makeatother

%%%%%%%%%%%%%
%  Correct  %
%%%%%%%%%%%%%

% EXAMPLE:
% 1. \correct{INCORRECT}{CORRECT}
% Parameters:
% 1. The incorrect statement.
% 2. The correct statement.
\definecolor{correct}{HTML}{009900}
\newcommand\correct[2]{{\color{red}{#1 }}\ensuremath{\to}{\color{correct}{ #2}}}


%%%%%%%%%%%%%%%%%%%%%%%%%%%%%%%%%%%%%%%%%%%%%%%%%%%%%%%%%%%%%%%%%%%%%%%%%%%%%%%
%                                 Environments                                %
%%%%%%%%%%%%%%%%%%%%%%%%%%%%%%%%%%%%%%%%%%%%%%%%%%%%%%%%%%%%%%%%%%%%%%%%%%%%%%%

\usepackage{varwidth}
\usepackage{thmtools}
\usepackage[most,many,breakable]{tcolorbox}

\tcbuselibrary{theorems,skins,hooks}
\usetikzlibrary{arrows,calc,shadows.blur}

%%%%%%%%%%%%%%%%%%%
%  Define Colors  %
%%%%%%%%%%%%%%%%%%%

\definecolor{myblue}{RGB}{45, 111, 177}
\definecolor{mygreen}{RGB}{56, 140, 70}
\definecolor{myred}{RGB}{199, 68, 64}
\definecolor{mypurple}{RGB}{197, 92, 212}

\definecolor{definition}{HTML}{228b22}
\definecolor{theorem}{HTML}{00007B}
\definecolor{example}{HTML}{2A7F7F}
\definecolor{definition}{HTML}{228b22}
\definecolor{prop}{HTML}{191971}
\definecolor{lemma}{HTML}{983b0f}
\definecolor{exercise}{HTML}{88D6D1}

\colorlet{definition}{mygreen!85!black}
\colorlet{claim}{mygreen!85!black}
\colorlet{corollary}{mypurple!85!black}
\colorlet{proof}{theorem}

%%%%%%%%%%%%%%%%%%%%%%%%%%%%%%%%%%%%%%%%%%%%%%%%%%%%%%%%%
%  Create Environments Styles Based on Given Parameter  %
%%%%%%%%%%%%%%%%%%%%%%%%%%%%%%%%%%%%%%%%%%%%%%%%%%%%%%%%%

\mdfsetup{skipabove=1em,skipbelow=0em}

%%%%%%%%%%%%%%%%%%%%%%
%  Helpful Commands  %
%%%%%%%%%%%%%%%%%%%%%%

% EXAMPLE:
% 1. \createnewtheoremstyle{thmdefinitionbox}{}{}
% 2. \createnewtheoremstyle{thmtheorembox}{}{}
% 3. \createnewtheoremstyle{thmproofbox}{qed=\qedsymbol}{
%       rightline=false, topline=false, bottomline=false
%    }
% Parameters:
% 1. Theorem name.
% 2. Any extra parameters to pass directly to declaretheoremstyle.
% 3. Any extra parameters to pass directly to mdframed.
\newcommand\createnewtheoremstyle[3]{
  \declaretheoremstyle[
  headfont=\bfseries\sffamily, bodyfont=\normalfont, #2,
  mdframed={
    #3,
  },
  ]{#1}
}

% EXAMPLE:
% 1. \createnewcoloredtheoremstyle{thmdefinitionbox}{definition}{}{}
% 2. \createnewcoloredtheoremstyle{thmexamplebox}{example}{}{
%       rightline=true, leftline=true, topline=true, bottomline=true
%     }
% 3. \createnewcoloredtheoremstyle{thmproofbox}{proof}{qed=\qedsymbol}{backgroundcolor=white}
% Parameters:
% 1. Theorem name.
% 2. Color of theorem.
% 3. Any extra parameters to pass directly to declaretheoremstyle.
% 4. Any extra parameters to pass directly to mdframed.
\newcommand\createnewcoloredtheoremstyle[4]{
  \declaretheoremstyle[
  headfont=\bfseries\sffamily\color{#2}, bodyfont=\normalfont, #3,
  mdframed={
    linewidth=2pt,
    rightline=false, leftline=true, topline=false, bottomline=false,
    linecolor=#2, backgroundcolor=#2!5, #4,
  },
  ]{#1}
}

%%%%%%%%%%%%%%%%%%%%%%%%%%%%%%%%%%%
%  Create the Environment Styles  %
%%%%%%%%%%%%%%%%%%%%%%%%%%%%%%%%%%%

\makeatletter
\@ifclasswith\class{nocolor}{
  % Environments without color.

  \createnewtheoremstyle{thmdefinitionbox}{}{}
  \createnewtheoremstyle{thmtheorembox}{}{}
  \createnewtheoremstyle{thmexamplebox}{}{}
  \createnewtheoremstyle{thmclaimbox}{}{}
  \createnewtheoremstyle{thmcorollarybox}{}{}
  \createnewtheoremstyle{thmpropbox}{}{}
  \createnewtheoremstyle{thmlemmabox}{}{}
  \createnewtheoremstyle{thmexercisebox}{}{}
  \createnewtheoremstyle{thmdefinitionbox}{}{}
  \createnewtheoremstyle{thmquestionbox}{}{}
  \createnewtheoremstyle{thmsolutionbox}{}{}

  \createnewtheoremstyle{thmproofbox}{qed=\qedsymbol}{}
  \createnewtheoremstyle{thmexplanationbox}{}{}
}{
  % Environments with color.

  \createnewcoloredtheoremstyle{thmdefinitionbox}{definition}{}{}
  \createnewcoloredtheoremstyle{thmtheorembox}{theorem}{}{}
  \createnewcoloredtheoremstyle{thmexamplebox}{example}{}{
    rightline=true, leftline=true, topline=true, bottomline=true
  }
  \createnewcoloredtheoremstyle{thmclaimbox}{claim}{}{}
  \createnewcoloredtheoremstyle{thmcorollarybox}{corollary}{}{}
  \createnewcoloredtheoremstyle{thmpropbox}{prop}{}{}
  \createnewcoloredtheoremstyle{thmlemmabox}{lemma}{}{}
  \createnewcoloredtheoremstyle{thmexercisebox}{exercise}{}{}

  \createnewcoloredtheoremstyle{thmproofbox}{proof}{qed=\qedsymbol}{backgroundcolor=white}
  \createnewcoloredtheoremstyle{thmexplanationbox}{example}{qed=\qedsymbol}{backgroundcolor=white}
}
\makeatother

%%%%%%%%%%%%%%%%%%%%%%%%%%%%%
%  Create the Environments  %
%%%%%%%%%%%%%%%%%%%%%%%%%%%%%

\declaretheorem[numberwithin=section, style=thmtheorembox,     name=Theorem]{theorem}
\declaretheorem[numbered=no,          style=thmexamplebox,     name=Example]{example}
\declaretheorem[numberwithin=section, style=thmclaimbox,       name=Claim]{claim}
\declaretheorem[numberwithin=section, style=thmcorollarybox,   name=Corollary]{corollary}
\declaretheorem[numberwithin=section, style=thmpropbox,        name=Proposition]{prop}
\declaretheorem[numberwithin=section, style=thmlemmabox,       name=Lemma]{lemma}
\declaretheorem[numberwithin=section, style=thmexercisebox,    name=Exercise]{exercise}
\declaretheorem[numbered=no,          style=thmproofbox,       name=Proof]{replacementproof}
\declaretheorem[numbered=no,          style=thmexplanationbox, name=Proof]{expl}

\makeatletter
\@ifclasswith\class{nocolor}{
  % Environments without color.

  \newtheorem*{note}{Note}

  \declaretheorem[numberwithin=section, style=thmdefinitionbox, name=Definition]{definition}
  \declaretheorem[numberwithin=section, style=thmquestionbox,   name=Question]{question}
  \declaretheorem[numberwithin=section, style=thmsolutionbox,   name=Solution]{solution}
}{
  % Environments with color.

  \newtcbtheorem[number within=section]{Definition}{Definition}{
    enhanced,
    before skip=2mm,
    after skip=2mm,
    colback=red!5,
    colframe=red!80!black,
    colbacktitle=red!75!black,
    boxrule=0.5mm,
    attach boxed title to top left={
      xshift=1cm,
      yshift*=1mm-\tcboxedtitleheight
    },
    varwidth boxed title*=-3cm,
    boxed title style={
      interior engine=empty,
      frame code={
        \path[fill=tcbcolback]
        ([yshift=-1mm,xshift=-1mm]frame.north west)
        arc[start angle=0,end angle=180,radius=1mm]
        ([yshift=-1mm,xshift=1mm]frame.north east)
        arc[start angle=180,end angle=0,radius=1mm];
        \path[left color=tcbcolback!60!black,right color=tcbcolback!60!black,
        middle color=tcbcolback!80!black]
        ([xshift=-2mm]frame.north west) -- ([xshift=2mm]frame.north east)
        [rounded corners=1mm]-- ([xshift=1mm,yshift=-1mm]frame.north east)
        -- (frame.south east) -- (frame.south west)
        -- ([xshift=-1mm,yshift=-1mm]frame.north west)
        [sharp corners]-- cycle;
      },
    },
    fonttitle=\bfseries,
    title={#2},
    #1
  }{def}

  \NewDocumentEnvironment{definition}{O{}O{}}
    {\begin{Definition}{#1}{#2}}{\end{Definition}}

  \newtcolorbox{note}[1][]{%
    enhanced jigsaw,
    colback=gray!20!white,%
    colframe=gray!80!black,
    size=small,
    boxrule=1pt,
    title=\textbf{Note:-},
    halign title=flush center,
    coltitle=black,
    breakable,
    drop shadow=black!50!white,
    attach boxed title to top left={xshift=1cm,yshift=-\tcboxedtitleheight/2,yshifttext=-\tcboxedtitleheight/2},
    minipage boxed title=1.5cm,
    boxed title style={%
      colback=white,
      size=fbox,
      boxrule=1pt,
      boxsep=2pt,
      underlay={%
        \coordinate (dotA) at ($(interior.west) + (-0.5pt,0)$);
        \coordinate (dotB) at ($(interior.east) + (0.5pt,0)$);
        \begin{scope}
          \clip (interior.north west) rectangle ([xshift=3ex]interior.east);
          \filldraw [white, blur shadow={shadow opacity=60, shadow yshift=-.75ex}, rounded corners=2pt] (interior.north west) rectangle (interior.south east);
        \end{scope}
        \begin{scope}[gray!80!black]
          \fill (dotA) circle (2pt);
          \fill (dotB) circle (2pt);
        \end{scope}
      },
    },
    #1,
  }

  \newtcbtheorem{Question}{Question}{enhanced,
    breakable,
    colback=white,
    colframe=myblue!80!black,
    attach boxed title to top left={yshift*=-\tcboxedtitleheight},
    fonttitle=\bfseries,
    title=\textbf{Question:-},
    boxed title size=title,
    boxed title style={%
      sharp corners,
      rounded corners=northwest,
      colback=tcbcolframe,
      boxrule=0pt,
    },
    underlay boxed title={%
      \path[fill=tcbcolframe] (title.south west)--(title.south east)
      to[out=0, in=180] ([xshift=5mm]title.east)--
      (title.center-|frame.east)
      [rounded corners=\kvtcb@arc] |-
      (frame.north) -| cycle;
    },
    #1
  }{def}

  \NewDocumentEnvironment{question}{O{}O{}}
  {\begin{Question}{#1}{#2}}{\end{Question}}

  \newtcolorbox{Solution}{enhanced,
    breakable,
    colback=white,
    colframe=mygreen!80!black,
    attach boxed title to top left={yshift*=-\tcboxedtitleheight},
    title=\textbf{Solution:-},
    boxed title size=title,
    boxed title style={%
      sharp corners,
      rounded corners=northwest,
      colback=tcbcolframe,
      boxrule=0pt,
    },
    underlay boxed title={%
      \path[fill=tcbcolframe] (title.south west)--(title.south east)
      to[out=0, in=180] ([xshift=5mm]title.east)--
      (title.center-|frame.east)
      [rounded corners=\kvtcb@arc] |-
      (frame.north) -| cycle;
    },
  }

  \NewDocumentEnvironment{solution}{O{}O{}}
  {\vspace{-10pt}\begin{Solution}{#1}{#2}}{\end{Solution}}
}
\makeatother

%%%%%%%%%%%%%%%%%%%%%%%%%%%%
%  Edit Proof Environment  %
%%%%%%%%%%%%%%%%%%%%%%%%%%%%

\renewenvironment{proof}[1][\proofname]{\vspace{-10pt}\begin{replacementproof}}{\end{replacementproof}}
\newenvironment{explanation}[1][\proofname]{\vspace{-10pt}\begin{expl}}{\end{expl}}

\theoremstyle{definition}

\newtheorem*{notation}{Notation}
\newtheorem*{previouslyseen}{As previously seen}
\newtheorem*{problem}{Problem}
\newtheorem*{observe}{Observe}
\newtheorem*{property}{Property}
\newtheorem*{intuition}{Intuition}

%%%%%%%%%%%%%%%%%%%%%%%%%%%%%%%%%%%%%%%%%%%%%%%%%%%%%%%%%%%%%%%
%                 Code Highlighting                           %
%%%%%%%%%%%%%%%%%%%%%%%%%%%%%%%%%%%%%%%%%%%%%%%%%%%%%%%%%%%%%%%
\usepackage{listings}
\lstset{
language=Octave,
backgroundcolor=\color{white},   % choose the background color; you must add \usepackage{color} or \usepackage{xcolor}
basicstyle=\footnotesize\ttfamily,        % the size of the fonts that are used for the code
breakatwhitespace=false,         % sets if automatic breaks should only happen at whitespace
breaklines=true,                 % sets automatic line breaking
captionpos=b,                    % sets the caption-position to bottom
commentstyle=\color{gray},    % comment style
%escapeinside={\%*}{*)},          % if you want to add LaTeX within your code
extendedchars=true,            % lets you use non-ASCII characters; for 8-bits encodings only, does not work with UTF-8
frame=single,                    % adds a frame around the code
% frameround=fttt,
keepspaces=true,                 % keeps spaces in text, useful for keeping indentation of code (possibly needs columns=flexible)
columns=flexible,
classoffset=0,
keywordstyle=\color{RoyalBlue},       % keyword style
deletekeywords={function,endfunction, if,endif},
classoffset=1,
morekeywords={function,endfunction, if,endif},
keywordstyle=\bf\color{Red},       % keyword style
classoffset=2,
morekeywords={persistent},            % if you want to add more keywords to the set
keywordstyle=\bf\color{ForestGreen},       % keyword style
classoffset=0,
literate=
{/}{{{\color{Mahogany}/}}}1
{*}{{{\color{Mahogany}*}}}1
{.*}{{{\color{Mahogany}.*}}}2
{+}{{{\color{Mahogany}+{}}}}1
{=}{{{\bf\color{Mahogany}=}}}1
{-}{{{\color{Mahogany}-}}}1
{[}{{{\bf\color{RedOrange}[}}}1
{]}{{{\bf\color{RedOrange}]}}}1
{ç}{{\c{c}}}1 % Cedilha
{á}{{\'{a}}}1 % Acentos agudos
{é}{{\'{e}}}1
{í}{{\'{i}}}1
{ó}{{\'{o}}}1
{ú}{{\'{u}}}1
{â}{{\^{a}}}1 % Acentos circunflexos
{ê}{{\^{e}}}1
{î}{{\^{i}}}1
{ô}{{\^{o}}}1
{û}{{\^{u}}}1
{à}{{\`{a}}}1 % Acentos graves
{è}{{\`{e}}}1
{ì}{{\`{i}}}1
{ò}{{\`{o}}}1
{ù}{{\`{u}}}1
{ã}{{\~{a}}}1 % Tils
{ẽ}{{\~{e}}}1
{ĩ}{{\~{i}}}1
{õ}{{\~{o}}}1
{ũ}{{\~{u}}}1,
numbers=left,                    % where to put the line-numbers; possible values are (none, left, right)
numbersep=6pt,                   % how far the line-numbers are from the code
numberstyle=\tiny\color{gray}, % the style that is used for the line-numbers
rulecolor=\color{black},         % if not set, the frame-color may be changed on line-breaks within not-black text (e.g. comments (green here))
showspaces=false,                % show spaces everywhere adding particular underscores; it overrides 'showstringspaces'
showstringspaces=false,          % underline spaces within strings only
showtabs=false,                  % show tabs within strings adding particular underscores
stepnumber=1,                    % the step between two line-numbers. If it's 1, each line will be numbered
stringstyle=\color{purple},     % string literal style
tabsize=2,                       % sets default tabsize to 2 spaces
}


\title{\Huge{23PHY114}\\ Class Notes}
\author{\huge{Adithya Nair}}
\date{}
\begin{document}

\maketitle
\newpage% or \cleardoublepage
% \pdfbookmark[<level>]{<title>}{<dest>}
\tableofcontents
\chapter{Solids}
\createintro
\section{Moment Of Area}
The moment of inertia is used to help find the "resistance" to the force, given a specific axis or direction.

\section{Resisting Force And Moments From Supports} % (fold)
\label{sec:resisting_force_and_moments_from_supports}
There are three main kinds of supports - 
\begin{enumerate}
  \item Pin/Hinge - Fixes linear motion but leaves rotation free. 
  \item Roller - Fixes rotation but leaves motion free. 
  \item Clamp - Fixes both linear and rotational motion.
\end{enumerate}
% section Resisting Force And Moments From Supports (end)
Take this case, with a bunch of forces being applied to the given load. 

There are three main things we need to find for this figure. 
\begin{enumerate}
  \item The resultant force acting on this bar fixed to a hinge. 
  \item The support reaction force and moment.
  \item The moment on the object (maximum)
\end{enumerate}
The way to approach the problem is as always,
\begin{enumerate}
  \item Free Body Diagram first. 
  \item Assuming $\Sigma f = 0 $, because the object has no acceleration currently, because it is fixed.
  \item Assuming $\Sigma M = 0$
  \item The point at which the resultant force is applied is found by, 
    \[
      \frac{\int |r|dm}{\int dm}
    \]
\end{enumerate}
\section{Derivation Of The Uniaxial Formula}

\section{Code For Uniaxial Deformation} % (fold)


\label{sec:for_uniaxial_deformation}
Main Subroutines For Uniaxial Deformation
% section For Uniaxial Deformation (end)
\subsection{Finding The Local Stiffness Matrix} % (fold)
\begin{lstlisting}{octave}
function stiffnessLocal = localStiffnessGenerator(E,A,l,theta);
    stiffnessConstant = E*A/l;
    R = [cos(theta) -sin(theta); sin(theta) cos(theta)];
    stiffnessMatrix = [stiffnessConstant 0 -stiffnessConstant 0; zeros(1,4);-stiffnessConstant 0 stiffnessConstant 0; zeros(1,4)];
    R4 = [R zeros(2,2); zeros(2,2) R];
    stiffnessLocal = R4*stiffnessMatrix*R4';
end
\end{lstlisting}
\label{sec:finding_the_local_stiffness_matrix}
% subsubsection Finding The Local Stiffness Matrix (end)
\subsection{Converting The Local Stiffness To A Global Stiffness Matrix}
\begin{lstlisting}{octave}
function stiffnessLocalGlobal = local2Global(stiffnessLocal,node1,node2,nodeCount)
    stiffnessLocalGlobal = zeros(2*nodeCount,2*nodeCount);
    i = 2*node1 - 1;
    j = 2*node2 - 1;
    stiffnessLocalGlobal(i:(i+1),i:(i+1)) = stiffnessLocal(1:2,1:2);
    stiffnessLocalGlobal(i:(i+1),j:(j+1)) = stiffnessLocal(1:2,3:4);
    stiffnessLocalGlobal(j:(j+1),i:(i+1)) = stiffnessLocal(3:4,1:2);
    stiffnessLocalGlobal(j:(j+1),j:(j+1)) = stiffnessLocal(3:4,3:4);
end
\end{lstlisting}
\subsection{Main Loop Through Evaluating The Global Stiffness Matrix}
\begin{lstlisting}{octave}
 A = 0; theta = 0; l = 0; stiffnessLocal = zeros(4,4); nodeAxialForces = zeros(nodeCount,1);
for element = 1:elementCount % For first three bars
    A = areaVector(element);
    theta = angleVector(element);
    l = lengthVector(element);
    stiffnessLocal = localStiffnessGenerator(E,A,l,theta);
    nodeCounter = element*2 - 1;
    stiffnessLocalGlobal = local2Global(stiffnessLocal,nodeVector(nodeCounter),nodeVector(nodeCounter+1),nodeCount);
    stiffnessLocal
    stiffnessLocalGlobal
    stiffnessGlobal += stiffnessLocalGlobal;
end
\end{lstlisting}
\subsection{Applying Boundary Conditions}
\begin{lstlisting}{octave}
selectedVector = [3 5:end]
forceEval = forceVector(selectedVector);
displacementEval = displacementVector(selectedVector);
stiffnessEval = stiffnessGlobal(selectedVector);
displacementEval = stiffnessEval\forceEval;
\end{lstlisting}
\section{Deriving For Bending Deformation}
Taking this example of a bar fixed to the wall, and examining small components of the bar and examining the forces at work on the component, we get a Tension, Moment and Stress.

Taking things along the $\hat{i}$ direction, we get
$$
-T + T + \Delta T = 0
$$
 What does this tell us? Tension is constant. 

While $-S + S + \Delta S - w\Delta x = 0$

$$
\frac{dS}{dx} = \lim_{\Delta x \rightarrow 0} \frac{\Delta S}{\Delta x} = w(x)
$$
We get an expression for the effects of the distributed force on the shear force of an individual object.

Looking at the angular momentum balance,

$$
\sum M_{/P} = 0
$$
$$
-M + (M + \Delta M) \hat{k} + (-\Delta x \hat{i} \times(-T\hat{i} - S \hat{j}) + (\frac{-\Delta x}{2} \hat{i} \times (-w(x) \Delta x \hat{j}))
$$
$$
\implies \frac{\Delta M }{\Delta x} + S - w\frac{\Delta x}{2} = 0
$$
Taking the limit on the equations, we get,
$$
\frac{dM}{dx} + S = 0
$$

Thus we get,
$$
\frac{dT}{dx} = 0
$$
$$
\frac{dS}{dx} = w(x)
$$
$$
\frac{dM}{dx} = -S
$$
Now we get to the Finite Element Method for bending.
$$
\theta(x) = v'(x)
$$
We find by simplification,

$$
c_1 = v_l 
$$
$$
c_2 = \theta_l
$$
$$
c_3 = \frac{3}{l^2}(v_r-v_l) - \frac{1}{l} (2\theta_l + \theta_r)
$$
$$
c_4 = \frac{2}{l^3} (v_l-v_r) + \frac{\theta_l+\theta_r}{l^2}
$$

$$
v(x) = H_1(x) v_l + H_2(x)\theta_l (H_3(x)) v_r + (H_4(x)) \theta_r
$$
$$
H_1 = 2(\frac{x}{l}^3) - 3\frac{x}{l}^2 + 1
$$
$$
H_2 = x-2\frac{x^2}{l} + \frac{x^3}{l^2}
$$
$$
H_3 = 3(\frac{x}{l})^2 -2\frac{x}{l}^3
$$
$$
H_4 = \frac{x^3}{l^2} - \frac{x^2}{l}
$$



Since we know that,
$$
\int_0^l EIq''v''dx
$$
$$
v = \underline{H}(x)^T v
$$
$$
q = H(x)
$$

\section{Quick Reference Notes}
\subsection{Derivations}
The governing differential equation is 
\[
  EI v^{(4)} = w(x)
\]

Then we can form a linear relationship
\begin{align*}
  v(x) & - \text{deflection} \\
  v^{(2)}(x) = \theta(x) & - \text{slope} \\
  EIv^{(2)}(x) = M(x) & - \text{bending moment related to curvature} \\
  EIv^{(3)}(x) = \frac{dM}{dx} = V(x) & - \text{transverse shear} \\ 
  EIv^{(4)}(x) = \frac{dV}{dx} = w(x) & - \text{load}
\end{align*}

Where, v is the deflection or displacement of the beam. E is the Young's Modulus and I is 

\subsection{Deriving Stiffness Matrix}
Use Hermite's interpolation formula to derive cubic shape functions for the deflection of beams.

\section{Final Equation}
For a bar, with axial, bending and shear deformation... The final equation is, where R, S and M are Axial, Shear and Moment.

\[
  F = k\Delta x
\]

\[
  \begin{bmatrix}
    R_1 \\ 
    S_1 \\
    M_1\\
    R_2 \\ 
    S_2 \\
    M_2\\
  \end{bmatrix}
  = 
  \begin{bmatrix}
    C_1 & 0 & 0 & -C_1 & 0 & 0 \\ 
    0 & 12C_2 & 6C_2L & 0 & -12C_2 & 6C_2L \\
    0 & 6C_2L & 4C_2L^2 & 0 & -6C_2L & 2C_2L^2 \\
    -C_1 & 0 & 0 & C_1 & 0 & 0 \\ 
    0 & -12C_2 & -6C_2L & 0 & 12C_2 & -6C_2L \\
    0 & 6C_2L & 2C_2L^2 & 0 & -6C_2L & 2C_2L^2 \\
  \end{bmatrix}
  \begin{bmatrix}
    u_1 \\ 
    v_1 \\ 
    \theta_1 \\
    u_2 \\ 
    v_2 \\ 
    \theta_2 \\
  \end{bmatrix}
\]
Where, $C_1 = \frac{EA}{L}$ and $C_2 = \frac{EI}{L^3}$

Rotating the matrix by angle $\theta$

\[
  R = \begin{bmatrix}
    c & s & 0 & 0 & 0 & 0 \\ 
    -s & c & 0 & 0 & 0 & 0 \\ 
    0 & 0 & 1 & 0 & 0 & 0 \\
    0 & 0 & 0 & c & s & 0 \\ 
    0 & 0 & 0 & -s & c & 0 \\ 
    0 & 0 & 0 & 0 & 0 & 1 \\
  \end{bmatrix}
\]

\chapter{Fluids}
\section{Main Contents}
\textbf{Computational Fluids Dynamics}
\begin{enumerate}
  \item Flow in channels and pipes
  \item Flow over objects 
\end{enumerate}

\begin{definition}[Laminar Region]
The region in a fluid where the flow is in a straight line is known as the laminar region 
\end{definition}

\begin{definition}[Turbulent Region]
The region in a fluid where the flow is not smooth is known as the turbulent region
\end{definition}

\textbf{The motivation behind learning CFD}

Computational Fluid Dynamics has been around as long as computers have been long. Using computation to calculate and visualize the flow of fluids is the basis behind wind tunnels. Trials can be done to help optimize solutions for the best aerodynamics.

Examples of data used is the velocity profile, the force and the acceleration. These fluid simulations can take months and be used to simulate the effects of fluids on a given shape.

\section{Terminology}
In engineering terms, fluids are deformable bodies. They take up the shape of its container. Viscosity, is defined as the friction to flow - $\mu$. It dictates the rate of flow of a fluid.

The counterintuitive picture. Imagine a stack of papers on your palm.  On tilting your palm, the first stack of paper goes out then the next then the next. It is dependent on the friction of the piece of paper below it. The last piece of paper has zero friction, and flows out the easiest. But a fluid does not work that way, when some water is placed on your palm, even after flowing out, some imprint of the water remains. 

Think of fluids as layered.

Taking a co-ordinate system, plotting the horizontal and vertical velocity of a fluid, u and v.

\begin{definition}[No slip Condition]
  A condition where the part of the fluid at the base or in contact with the surface has no horizontal velocity.
  \[
    U(x,0) = 0
  \]
\end{definition}

Unlike solids, fluids have a profile that is counterintuitive to the way we imagine physics.

Counterintuitively, a fluid's profile is parabolic, in a pipe. Closed on all sides, the middlemost component of the fluid moves with 

\begin{note}
\textbf{Lagrangian Method} -
The method of following a single point and discovering interesting properties about it is known as the Lagrangian Method

\textbf{Eulerian Method} -
The method of following a single location and performing analyses on that location.
\end{note}
The study of fluids is performed using the Eulerian Method. 

\section{Derivation Of Equations}
Taking the fluid flowing through the figure given. In the Eulerian Method, we focus on a given location. This location here is the outlined box. 

\[
\]

The mass flow rate for a given location is 
\begin{align*}
  \dot{m}_{in} - \dot{m}_{out} &= \frac{dm}{dt} \\
  \dot{m} &= \frac{d\rho V}{dt} \\
  \dot{m} &= \rho \frac{dV}{dt} \\
          &= \rho Q & \text{(Q - volume flow rate)} \\
\end{align*}
\[
  \rho \frac{dV}{dt} = \Delta y \Delta z . u
\]
\begin{align*}
  (\dot{m}_{intx} + \dot{m}_{inty}) - (\dot{m}_{outx} + \dot{m}_{outy}) &= \frac{dm_{inside}}{dt} \\
  (\rho \Delta Y \Delta Z u + \rho \Delta X \Delta Z v) - (\rho \Delta Y \Delta Z (U + \Delta U) + \rho \Delta X \Delta Z ( v + \Delta v) &= \frac{d(\rho \Delta X \Delta Y \Delta Z)}{dt} \\
  (\rho \Delta Y \Delta Z u + \rho \Delta X \Delta Z v) - (\rho \Delta Y \Delta Z (U + \Delta U) + \rho \Delta X \Delta Z ( v + \Delta v) &= \frac{d(\rho \Delta X \Delta Y \Delta Z)}{dt} \\
  - \rho \Delta Y \Delta Z \frac{\Delta U}{\Delta X} - \rho \Delta x \Delta z \frac{\Delta V}{\Delta Y} &= \Delta X \Delta Y \Delta Z \frac{d \rho}{dt} \\
  - \rho \frac{\partial U}{\partial x}  - \rho \frac{\partial v}{dy} &= \frac{d\rho}{dt} \\
\end{align*}
\end{document}


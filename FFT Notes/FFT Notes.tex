\documentclass[twoside]{article}

\input{preamble}
\title{\Huge{Fast Fourier Transform}\\ Notes}
\author{\huge{Adithya Nair}}
\date{}

\begin{document}

\maketitle
\newpage% or \cleardoublepage
% \pdfbookmark[<level>]{<title>}{<dest>}
\section{Overview}
\begin{enumerate}
	\item Modern digital media is based on discrete media, rather than continuous functions, analog signals are sampled at equal time intervals and operations are performed on the resulting discrete data. The Fourier series can be used to re-express this sampled data as continuous functions.
	\item An important application of this is signal and image processing.
	\item The Fast Fourier Transform is a numerical algorithm that can convert a signal to its discrete Fourier coefficients or vice versa.
	\item A limitation of the classical Fourier methods, is that it does not do well for localized data, this is solved using wavelets.
\end{enumerate}
\section{Discrete Fourier Analysis}
Fourier analysis decomposes the sampled signal into its fundamental periodic constituents or complex exponential constituents. Each exponential constituent together, forms an infinite orthogonal basis. Functions are expressed in infinite dimensional spaces, a Fourier series of a given function gives us an orthogonal basis to express that same function which is convenient for differentiation and integration.

\begin{align*}
	&x_j = a + jh, \ j = 0, \dots, n, & \text{where} \ h = \frac{b-a}{n}
\end{align*}

In signal processing, $x$ represents time instead of space, $x_j$ represents the times we sample $f(x)$.

We use $0 \leq x \leq 2 \pi$ with n equally spaced points,(any function outside that can simply be rescaled.)
\[ 
	x_0 = 0, x_1 = \frac{2\pi}{n}, x_2 = \frac{4\pi}{n}, x_j = \frac{2j\pi}{n}, x_{n-1} = \frac{2(n-1)\pi}{n}
\]

The Euler's form of complex numbers are,

\[
	f(x) = e^{inx} = \cos{nx} + i \sin{nx}
\]


The discrete Fourier representation, has sampled values,

\[
	f_j = f(\frac{2j \pi}{n}) = exp(in \frac{2j\pi}{n}) = e^{2j\pi i} = 1
\]
Since n is equally spaced... This implies that sampling at n equally spaced samples \textbf{cannot} detect periodic signals of any frequency. The Fourier representation always gives the same sample vector $(1,1,\dots, 1)^T$. 

This means that upon equally spaced sampling, $e^{i(k+n)x}$ and $e^{ikx}$ are equal. So we need to only use the first n periodic complex exponential functions.
\end{document}
